\pagenumbering{arabic}
%\setcounter{page}{1} 

\section{JOGOS}

\subsection{BENEFÍCIOS}
Desenvolvedores de jogos web podem rapidamente satisfazer as
necessidades de seus jogadores, mantendo-os leais a tecnologia HTML5
\autocite{developingEffect}.

A maioria dos desenvolvedores demonstra interesse para o HTML5.
(referenciar)

O tempo de desenvolvimento de uma aplicação em HTML5 é 67\% menor
(referenciar) que aplicações nativas. Isso mostra o custo efetivo de
aplicações baseadas em HTML5.

A real vantagem de aplicações em HTML5 é o suporte horizontal entre
as plataformas - que é a maior razão por trás do custo efetivo.
(HASAN et al, 2012)

\subsection{O MERCADO}

A maior dificuldade em capturar uma base de usuários é que o mercado
de dispositivos móveis é muito fragmentado e não existe uma única
plataforma popular. (HASAN, 2012)

Segundo \cite{HTML5CrossPlatformGameDevelopment}: 80\% do tempo total
gasto usando dispositivos móveis é para a utilização de aplicativos,
e 32\% é para jogar vídeo games.


\subsection{JOGOS E MULTIPLATAFORMA}

Funcionalidades foram disponibilizadas de diversas fontes e não foram
construídas de forma consistente com as demais. Além
disso, devida a única característica da Web, erros de implementação
se tornam frequentes, e muitas vezes se tornam o padrão, pois outras
funcionalidades dependem destas primeiras antes que elas estejam
estáveis. (W3C manual)

Enquanto o HTML é desenvolvido muitas das funcionalidades
disponibilizadas são testadas em apenas um pequeno conjunto de
navegadores para um pequeno conjunto de versões (referência 2). Isso
acarreta em suporte inconsistente. A forma mais segura de garantir
suporte é testando em todas as versões alvo, todavia essa solução
não é prática. (ref. 2)

Os desenvolvedores de navegadores podem interpretar/implementar
as especificações erroneamente aumentando os problemas de
compatibilidade.

\section{ESTE TRABALHO}

Este projeto propõe analisar as limitações do HTML5 quanto relativo
a construção de jogos multiplataforma. Através de revisão
bibliográfica e da criação de um protótipo de jogo multiplataforma.

% Um tratado completo sobre o assunto requiriria um comparativo entre
% jogos desenvolvidos nativamente e jogos em HTML5.

%Mencionar a que areas  da computacao este trabalho se inclui (jogos, etc).

