\pagenumbering{arabic}
%\setcounter{page}{1} 


Desenvolvedores de jogos web podem rapidamente satisfazer as
necessidades de seus jogadores, mantendo-os leais a tecnologia HTML5
\autocite{developingEffect}.

Mais de 80\% do tempo total gasto utilizando dispositivos móveis é na
utilização de aplicativos, e 32\% deste tempo é para jogar vídeo
games \autocite{HTML5CrossPlatformGameDevelopment}.

Mais de 70\% das pessoas que jogam vídeo games são adultos
\autocite{gamebenefits}.

A maioria dos desenvolvedores demonstra interesse para o HTML5.
(referenciar) E muito pouco investimento é necessário para
começar a desenvolver jogos utilizando as tecnologias da WEB
\autocite{html5mostwanted}.

O tempo de desenvolvimento de uma aplicação em HTML5 é 67\% menor que
aplicações nativas \autocite[pp. 460]{html5Tradeoffs}. Isso mostra o
custo efetivo de aplicações baseadas em HTML5.

A real vantagem de aplicações em HTML5 é o suporte horizontal entre
as plataformas - que é a maior razão por trás do custo efetivo
\autocite{html5Tradeoffs}.

A maior dificuldade em capturar uma base de usuários é que o mercado
de dispositivos móveis é muito fragmentado e não existe uma única
plataforma popular.\autocite{html5Tradeoffs}.

Alguns jogadores de Hattrick tem participado do jogo por mais de 
10 anos \autocite{gameCommunities}

Funcionalidades foram disponibilizadas de diversas fontes e não foram
construídas de forma consistente com as demais. Além disso,
devida a única característica da Web, erros de implementação se
tornam frequentes, e muitas vezes se tornam o padrão, pois outras
funcionalidades dependem destas primeiras antes que elas estejam
estáveis. (W3C manual)

Os desenvolvedores de navegadores podem interpretar/implementar
as especificações erroneamente aumentando os problemas de
compatibilidade.

\section{Limitações}

A utilização da WEB é gigantesca, mas criar jogos
dinâmicos e de tempo real não é o primeiro objetivo da WEB
\autocite{html5mostwanted}.

\section{ESTE TRABALHO}

Este projeto propõe analisar as limitações do HTML5 quanto relativo
a construção de jogos multiplataforma. Através de revisão
bibliográfica e da criação de um protótipo de jogo multiplataforma.

% Um tratado completo sobre o assunto requiriria um comparativo entre
% jogos desenvolvidos nativamente e jogos em HTML5.

%Mencionar a que areas  da computacao este trabalho se inclui (jogos, etc).

\end{draft}

%cite the sections of the work and what can be found in each of them
