\pagenumbering{arabic}
%\setcounter{page}{1} 

Desde a década de 90 jogos digitais se tornaram uma forma dominante
de recriação \autocite{gameDesignPatterns}. Devido a fatores como o
barateamento da construção de hardware e sua crescente capacidade a massificação
dos dispositivos móveis inteligentes se tornou possível. Os
computadores deixaram de ser apenas uma ferramenta científica e de
negócios, tornando-se uma plataforma de diversão. Atualmente jogos
são utilizados em uma vasta gama de dispositivos \autocite[pp.
6]{crossPlatformMobileGameDevelopment}.

\autocite{HTML5CrossPlatformGameDevelopment} afirma que mais de 80\% do
tempo total gasto utilizando dispositivos móveis é na utilização de
aplicativos, e 32\% deste tempo é para jogar vídeo games.

Não obstante, devido a variedade de dispositivos, é difícil para
os criadores de jogos alcançarem todos os jogadores possivelmente
interessados em suas criações. \citet{html5Tradeoffs} afirma que a
maior dificuldade em capturar uma base de usuários é que o mercado
de dispositivos móveis é muito fragmentado e não existe uma única
plataforma popular.

Esta característica força os criadores de jogos a suportarem diversas
plataformas. \autocite{htmlSurvey} afirma que 81\% dos aplicativos
mobile rodam em pelo menos dois sistemas operacionais. Para atingir
múltiplas plataformas os desenvolvedores utilizam de variadas
estratégias, sendo uma delas o HTML5.

Desde o lançamento da versão 5, o HTML deixou de ser apenas uma
plataforma para visualização de documentos na Internet e vem
conquistando sua posição como uma forma de desenvolver jogos
para múltiplas plataformas. Isso se dá pelas características de
multimídia adicionadas nesta versão do HTML. E pelo suporte horizontal
a múltiplas plataformas que o HTML provê \autocite{html5Tradeoffs}.
Além destes, outros benefícios são observáveis na construção de
jogos em HTML.

Muito pouco investimento é necessário para começar a desenvolver
jogos utilizando as tecnologias da WEB \autocite{html5mostwanted}.
Desenvolvedores de site podem reaproveitar o conhecimento direcionando-o
para o desenvolvimento de jogos. O interesse por parte dos
desenvolvedores também é grande \autocite{htmlSurvey} cita que cerca
de 59\% dos desenvolvedores estão muito interessados em desenvolver
aplicativos em HTML5.

As vantagens financeiras do desenvolvimento de aplicações
multiplataforma em HTML também são substanciais. O tempo de
desenvolvimento de uma aplicação em HTML5 é 67\% menor que
aplicações nativas \autocite[p. 460]{html5Tradeoffs}.

Não obstante, apesar utilização da WEB ser gigantesca, criar
jogos dinâmicos e de tempo real não é seu primeiro objetivo
\autocite{html5mostwanted}. O processo de desenvolvimento de
aplicações HTML5 está em constante fluxo de aperfeiçoamento e novos
métodos, técnicas, e ferramentas estão aparecendo todo o tempo
\autocite{crossPlatformMobileGame}.

Neste contexto vê-se a necessidade de uma revisão das tecnologias do
HTML para jogos, suas fraquezas e especialidades. Este trabalho propõe
analisar as limitações do HTML5 quanto relativo a construção de
jogos multiplataforma através de revisão bibliográfica e da criação
de um protótipo de jogo multiplataforma. O protótipo escolhido foi
um jogo de matemática simples onde o usuário pode escolher se uma
questão matemática, e seu resultado dado estão corretos.

Com as informações coletas de maneira prática e teórica foram
registradas as limitações e uma análise simples sobre elas foi
efetuada.
