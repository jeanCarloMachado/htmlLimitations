\pagenumbering{arabic}
\setcounter{page}{10}

Desde a década de 90, jogos digitais se tornaram uma "forma dominante
de recriação" \autocite{gameDesignPatterns}. Devido a fatores
como o barateamento da construção de hardware e sua crescente
capacidade a massificação dos dispositivos móveis inteligentes
se tornou possível. Os computadores deixaram de ser apenas
ferramentas científica e de negócios, tornando-se plataformas de
diversão. Atualmente jogos são utilizados em uma vasta gama de
dispositivos \citet[p. 6]{crossPlatformMobileGameDevelopment}.
\citet{HTML5CrossPlatformGameDevelopment} afirma que mais de 80\% do
tempo total gasto utilizando dispositivos móveis é na utilização de
aplicativos, e 32\% deste tempo é para jogar videogames.

Entretanto, devido a variedade de dispositivos, é difícil para
os criadores de jogos alcançarem todos os jogadores possivelmente
interessados em suas criações. \citet{html5Tradeoffs} afirma que a
maior dificuldade em capturar uma base de usuários é que o mercado
de dispositivos móveis é muito fragmentado e não existe uma única
plataforma popular. Esta característica força os criadores de jogos a
suportarem diversas plataformas. \citet{htmlSurvey} afirma que 81\% dos
aplicativos mobile rodam em pelo menos dois sistemas operacionais. Para
atingir múltiplas plataformas os desenvolvedores utilizam de variadas
estratégias, sendo uma delas o HTML5.

Desde o lançamento da versão 5, o HTML deixou de ser apenas uma
plataforma para visualização de documentos na Internet e vem
conquistando sua posição como uma forma de desenvolver jogos
para múltiplas plataformas. Isso se dá pelas características de
multimídia adicionadas nesta versão do HTML. E pelo suporte horizontal
a múltiplas plataformas que o HTML provê \autocite{html5Tradeoffs}.
Além destas características positivas, outros benefícios são
observáveis na construção de jogos em HTML.

Muito pouco investimento é necessário para começar a desenvolver
jogos utilizando as tecnologias da Web \autocite{html5mostwanted}.
Desenvolvedores de sites podem reaproveitar o conhecimento
direcionando-o para o desenvolvimento de jogos. O interesse por
parte dos desenvolvedores também é grande \autocite{htmlSurvey}
cita que cerca de 59\% dos desenvolvedores estão muito interessados
em desenvolver aplicativos em HTML5. As vantagens financeiras do
desenvolvimento de aplicações multiplataforma em HTML também são
substanciais. O tempo de desenvolvimento de uma aplicação em HTML5 é
67\% menor que aplicações nativas \autocite[p. 460]{html5Tradeoffs}.

Contudo, apesar utilização da Web ser gigantesca, criar
jogos dinâmicos e de tempo real não é seu primeiro objetivo
\autocite{html5mostwanted}. O processo de desenvolvimento de
aplicações HTML5 está em constante fluxo de aperfeiçoamento e novos
métodos, técnicas, e ferramentas estão aparecendo todo o tempo
\autocite{crossPlatformMobileGame}.

Neste contexto vê-se a necessidade de uma revisão das tecnologias do
HTML para jogos, suas fraquezas e especialidades. Este trabalho propõe
analisar as limitações do HTML5 relativo a construção de jogos
multiplataforma através de revisão bibliográfica e da criação de um
protótipo de jogo multiplataforma. O protótipo escolhido foi um jogo
de matemática simples onde o usuário pode escolher se uma questão
matemática, e seu resultado dado, estão corretos. Com as informações
coletadas, de maneira prática e teórica, foram registradas as limitações
e uma análise sobre elas foi efetuada.

\section{PROBLEMA}
%{{
A carência de definições concretas sobre a viabilidade da atual
versão do HTML quando aplicado ao desenvolvimento de jogos, e o senso
comum, acostumado com soluções nativas, acabam por praticamente
monopolizar a construção de jogos nativos.

Neste contexto este trabalho busca responder a seguinte pergunta: quais os
reais problemas e limitações comuns no desenvolvimento de jogos
multiplataforma em HTML5?

%}}}
\section{OBJETIVOS}

Os objetivos gerais e específicos do trabalho serão descritos a seguir.

\subsection{Objetivo Geral}

Identificar possíveis limitações no processo de desenvolvimento
de jogos multiplataforma oriundas do atual estado de definição e
implementação do HTML5.

Não é objetivo deste trabalho comparar o HTML com outras tecnologias
de desenvolvimento de jogos, como Flash Player, Silverlight ou
alternativas Desktop. Ou então demonstrar os pontos fortes do HTML5,
apesar de haver revisão das tecnologias, o artefato final deste
trabalho são as limitações do HTML.

\subsection{Objetivos Específicos}

O primeiro objetivo é estudar e resumir as seguintes tecnologias:

\begin{itemize}
\item Tecnologias de renderização;
\item Formas de disponibilizar jogos;
\item Eventos de entrada;
\item Armazenamento;
\item Offline.
\end{itemize}

Estas foram selecionadas baseando-se em recomendações da literatura
e mercado. Por exemplo, \citet{browserGamesTechnologyAndFuture} cita
que vídeo, áudio, drag-and-drop, funcionalidades de gráficos e
armazenamento offline são aspectos importantes do HTML5 para o
desenvolvimento de jogos. Já o site da Mozilla \citet{gamesIntroduction}
contém uma lista de tecnologias que julga relevantes ao desenvolvimento
de jogos na Web as quais são fortemente relacionadas aos itens acima
mencionados.

Outro objetivo é a criação de um protótipo para melhor compreensão
das tecnologias da Web e detecção de suas limitações. O protótipo
deve ser criado sem utilização de bibliotecas ou frameworks de
terceiros para eliminar a possibilidade de limitações do HTML serem
tratadas ou ocultas por estas tecnologias.

Definiu-se que o protótipo deve funcionar nas plataformas Android
5 e nos navegadores Desktop Google Chrome e Firefox em suas últimas
versões (47 e 43 respectivamente), para realizar a característica
multiplataforma do trabalho. Optamos por Android, e não IOS, pois
o primeiro contém a vasta maioria do mercado de dispositivos
inteligentes, e por termos maior experiência na já mencionada
plataforma.

O objetivo final é a criação de uma lista de limitações relacionadas
aos assuntos elencados que influenciem a construção de jogos.

\section{JUSTIFICATIVA}

Tendo em vista que este trabalho busca mapear possíveis problemas
do desenvolvimento multiplataforma em HTML, ele serve para apoiar
e justificar decisões relativas ao desenvolvimento de jogos
multiplataforma.

Muitas pessoas no mercado de software são da opinião de que o
desenvolvimento nativo para jogos é a melhor opção, quando outras
formas de aplicativos são mais adequados para a Web \citet[p.
21]{aSeriousContender}. Este trabalho, por indicar problemas concretos
do desenvolvimento Web, serve para verificar essa opinião.

Grande parte dos desenvolvedores estão familiarizados com as
tecnologias da Web ou apontam interesse na tecnologia. A revisão
tecnológica que este trabalho contém pode servir como um indicativo
de sobre o que é preciso estudar para começar a desenvolver jogos na
Web.

\section{METODOLOGIA}
%{{{
\thispagestyle{myheadings}

O primeiro passo consiste na definição das plataformas alvo do trabalho.
Estas devem ser relevantes mercadologicamente ao desenvolvimento de
jogos em HTML5.

Segue-se com a elaboração de uma lista com os recursos relevantes
aos jogos que, sofrem ou são comummente ligados à
limitações multiplataforma. Segue-se uma pesquisa para aprofundar
teoricamente cada um dos recursos, possivelmente elegendo novos.

Com um baseamento teórico substancial, o próximo passo é a criação
do protótipo de um jogo multiplataforma que utilize recursos
analisados.

Com o protótipo concebido, o passo que segue é a enumeração, e
descrição das limitações detectadas no processo de desenvolvimento e
testes do jogo. Este detalhamento deve responder perguntas como:

\begin{itemize}
\item Quais as limitações são comuns desenvolvimento de jogos em HTML5?
\item Em quais plataformas?
\item Sob quais circunstâncias?
\item Quais limitações podem ser contornadas? Como?
\end{itemize}

%}}}

\section{TRABALHOS RELACIONADOS}
%{{{
\citet{crossPlatformMobileGame} elaborou uma revisão de aspectos do
HTML5 através da construção de um jogo. O autor foca muito nos
aspectos de criação de jogos e feedback do desenvolvimento. Troca
de tecnologias e não especificamente nas limitações conforme o meu
trabalho. Em outras palavras seu escopo é mais genérico e não tão
preciso quanto este

\citet{aSeriousContender} realizou uma pesquisa através de questionário
e protótipo sobre a viabilidade de aplicativos em HTML5, concluindo que
no geral desenvolvimento de aplicativos em HTML5 são opções viáveis
e lucrativas. Seu trabalho difere a este por não focar no contexto dos
jogos, não observando muitas das nuances e necessidades específicas
para o desenvolvimento de jogos. Outra diferença substancial é que o
autor foca apenas na viabilidade não ressaltando as limitações da
plataforma.

\citet{crossPlatformMobileGameDevelopment} revisa algumas tecnologias
da web e constrói um jogo protótipo para aprender um framework que
possibilita a construção de jogos multiplataforma. Não obstante
o autor foca em um framework de compilação múltipla, não usando
diretamente as tecnologias da Web. O autor também não foca na experiência do
desenvolvimento ou em coletar limitações como este projeto se propõe.

\citet{viabilityBusinessApplications} estuda a viabilidade de
aplicações comerciais multiplataforma em HTML5 através da
construção de um aplicativo comercial em Sencha Touch. É construída
uma lista de recursos interessantes no desenvolvimento comercial
de aplicações e cada um destes recursos é revisado depois do
desenvolvimento, assemelhando-se muito a metodologia deste trabalho. Em
seu estudo o autor utiliza uma ferramenta de desenvolvimento em C\# que
compila nativamente o que se distancia da proposta deste trabalho de
avaliar as limitações com a construção de um protótipo diretamente
em HTML. O autor também não se foca em tecnologias dos jogos,
outrossim aplicações genéricas.

%}}}

