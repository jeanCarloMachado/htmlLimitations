\pagenumbering{arabic}
%\setcounter{page}{1} 

O mercado de jogos em dispositivos móveis é substancial.
Mais de 80\% do tempo total gasto utilizando dispositivos móveis é na
utilização de aplicativos, e 32\% deste tempo é para jogar vídeo
games \autocite{HTML5CrossPlatformGameDevelopment}.

O público alvo dos jogos é fortemente focado na parcela adulta.
\autocite{gamebenefits} afirma que mais de 70\% das pessoas que jogam vídeo games são adultos.

Não obstante, é difícil alcançar todos os jogadores
\cite{html5Tradeoffs} afirma que a maior dificuldade em capturar uma
base de usuários é que o mercado de dispositivos móveis é muito
fragmentado e não existe uma única plataforma popular.

\autocite{htmlSurvey} afirma que 81\% dos aplicativos mobile rodam em
pelo menos dois sistemas operacionais.

Desde o lançamento da versão 5 do HTML, o vem conquistando sua
posição como uma forma de desenvolver jogos para múltiplas
plataformas. Pelas características de muiltimídia adicionadas nesta
versão do HTML5 e pelo suporte horizontal a múltplas plataformas que o
HTML provê \autocite{html5Tradeoffs}.

Muito pouco investimento é necessário para começar a desenvolver
jogos utilizando as tecnologias da WEB \autocite{html5mostwanted}.
Desenvolvedores de site podem reaproveitar o conhecimento direcionando-o
para o desenvolvimento de jogos. O interesse por parte dos
desenvolvedores é grande. \autocite{htmlSurvey} cita que cerca de 59\%
dos desenvolvedores estão muito interessados em desenvolver aplicativos
em HTML5.
As vantagens técnicas financeiras também são substanciais.
O tempo de desenvolvimento de uma aplicação em HTML5 é 67\% menor que
aplicações nativas \autocite[pp. 460]{html5Tradeoffs}. Isso mostra o
custo efetivo de aplicações baseadas em HTML5.

Alguns jogadores de Hattrick tem participado do jogo por mais de 
10 anos \autocite{gameCommunities}

A utilização da WEB é gigantesca, mas criar jogos
dinâmicos e de tempo real não é seu primeiro objetivo \autocite{html5mostwanted}.

O processo de desenvolvimento de aplicações HTML5 está em constante
fluxo de aperfeiçoamento e novos métodos, técnicas, e ferramentas
estão aparecendo todo o tempo \autocite{crossPlatformMobileGame}.

Neste contexto vê-se a necessidade de uma revisão das tecnologias, suas fraquezas e especialidades.

Este trabalho propõe analisar as limitações do HTML5 quanto relativo
a construção de jogos multiplataforma. Através de revisão
bibliográfica e da criação de um protótipo de jogo multiplataforma.

O protótipo escolhido foi um jogo de matemática simples onde o
usuário pode escolher se uma questão matemática, e seu resultado dado
estão corretos. Com as informações coletas de maneira prática e
teórica foram registradas as limitações e uma análise simples sobre
elas foi efetuada.



