Neste trabalho revisamos tecnologias relevantes no desenvolvimento
de jogos multiplataforma em HTML5 e algumas das limitações a elas
relacionadas. Para tanto, elaborou-se uma pesquisa bibliográfica
dos assuntos relevantes. Em seguida, criou-se um protótipo de jogo
para avaliar experimentalmente algumas das tecnologias e detectar
limitações. Apesar de simples, o protótipo desenvolvido ajudou
substancialmente na validação de algumas limitações previamente
encontradas e na detecção de alguns problemas adicionais os quais
foram citados acima.

As limitações encontradas através do processo mencionado foram
registradas e avaliadas. Os resultados obtidos nos levam a crer que,
apesar de o número de limitações ser substancial, a grande maioria
destas limitações podem ser contornadas pelo programador ou já estão
no processo de serem solucionadas pelas especificações e navegadores.
As tecnologias da Web vem evoluindo a grande passo e número de pontos
positivos da abordagem HTML para desenvolvimento de jogos é muito
grande em comparação às limitações.

É seguro dizer que o HTML ainda não está pronto para substituir todas
as estratégias de desenvolvimento de jogos, nem pode atender todas as
categorias de jogos. Não obstante, as tecnologias da Web podem ser
aplicadas ao desenvolvimento de jogos em um leque crescente de casos.

Existem várias funcionalidades interessantes para jogos que ainda não
estão prontas ou, por hora, não funcionam em todos os navegadores
como: WebVR, ES6, HTTP/2, Gamepad, entre outras. Entretanto a maioria
destas tecnologias já podem ser utilizadas pelo programador, requerendo
apenas um pouco de diligência.

No contexto multiplataforma, provavelmente o maior desafio é prover
interatividade em meio a uma coleção de dispositivos diferentes.
Todavia, através da criação de aplicações incrementais,
adicionando recursos conforme a capacidade dos dispositivos, e
utilizando os recursos de retrocompatibilidade que fazem parte dos
alicerces do HTML. É possível fornecer a melhor experiência que cada
plataforma comporta.

Das limitações existentes, provavelmente as mais difíceis de resolver
são aquelas relacionadas as próprias filosofias da Web, como a
impossibilidade de controlar recursos via script por levantar problemas
em potencial sobre a segurança do usuário. Contudo mesmo estes problemas
podem ser solucionados dado o interesse da comunidade.

De uma visão macroscópica, se considerarmos o histórico do HTML,
principalmente após o lançamento do HTML5, as limitações do HTML5
são passageiras. A figura \ref{fig:htmlSupport} corrobora com esta
hipótese, o crescimento do suporte das tecnologias nesta imagem
apresenta uma curva, em traços gerais, exponencial. Ao que tudo indica
o futuro dos jogos em HTML5 parece brilhante.

\section{Trabalhos Futuros}

Visto que a especificação do HTML é viva, checar suas limitações
é uma tarefa que poderia ser feita de tempos em tempos. EMACScript 6
e Web Assembly podem mudar completamente o cenário de desenvolvimento
de jogos Web assim que se tornarem opções viáveis, comercialmente
falando.

Seria interessante que assuntos como o WebGL e WebVR, levemente
abordados neste trabalho, fossem estudados com profundidade, visto que
são deveras importantes para a criação de jogos cada vez mais
iterativos. O suporte a mais versões de navegadores e plataformas,
bem como outras metodologias de desenvolvimento, também poderiam ser
estudadas.

A utilização de plugins foi intencionalmente ignorada deste
trabalho. Conquanto, em jogos comerciais, estas ferramentas são
imprescindíveis. Trabalhos que analisem a construção de um jogo sob
a perspectiva comercial, utilizando plugins, na forma mais aproximada
o possível da realidade do mercado, seriam igualmente interessantes.
