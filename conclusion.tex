\begin{draft}
Apesar dos problemas mencionados neste trabalho, tomando uma perspectiva
mais abrangente as tecnologias de desenvolvimento de jogos evoluiu muito
e existem muitas coisas boas para se dizer à respeito.

A manipulação de erros é bastante consistente nos navegadores mas 
incrivelmente não faz parte da especificação \autocite{howBrowsersWork}.
Não pude testar todos os métodos e ferramentas e versões à
disposição, um trabalho completo demandaria esforços conjuntos de
muitos indivíduos ou um período de tempo bem mais extenso. 

Se uma empresa deseja produzir jogos nativos elas precisarão de vários
desenvolvedores. Eu sozinho fui capaz de produzir um jogo em tempo
razoável trabalhando com a plataforma WEB.

Por não utilizar frameworks e bibliotecas estou me distanciando
dos casos da vida real.

Só poderemos considerar o HTML como uma especificação pronta quando
for possível fazer tudo o que se faz nativamente com os dispositivos
através de uma API WEB padronizada.

Muitas das limitações do HTML5 são contornáveis através de JavaScript.

O futuro dos jogos em HTML5 parece brilhante.

Neste trabalho revisamos tecnologias relevantes no desenvolvimento de
jogos.

Este trabalho surgiu da minha vontade de aprender mais sobre a WEB....

O suporte ao HTML vem crescendo com o tempo, o site HTML5Test
\url{http://html5test.com/about.html}, oferece um placar atualizado
dinamicamente, conforme utilização dos navegadores, sobre os recursos
do HTML.

\subsection{TRABALHOS FUTUROS}

Trabalhos que explorem os benefícios mercadológicos do HTML5 em comparação com alternativas nativas.
EMACSCRIPT 7.

HTTP2 também traz boas perspectivas em relação a performance.

\end{draft}
