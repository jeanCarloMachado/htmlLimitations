\begin{draft}

Neste trabalho revisamos tecnologias relevantes no desenvolvimento
de jogos multiplataforma em HTML5 e algumas das limitações a elas
relacionadas.

Para tanto, elaborou-se uma pesquisa bibliográfica
dos assuntos relevantes. Em seguida, criou-se um protótipo de jogo
para avaliar experimentalmente algumas das tecnologias e detectar
limitações. Apesar de simples, o protótipo desenvolvido ajudou
bastante na validação de algumas limitações previamente encontradas
e na detecção de alguns problemas adicionais os quais foram citados
previamente.

As limitações encontradas através do processo mencionado, foram
registradas e avaliadas. Os resultados obtidos nos levam a crer que,
apesar de o número de limitações ser substancial, a grande maioria
destas limitações podem ser contornadas ou já estão no processo de
serem.

As tecnologias da WEB vem evoluindo a grande passo e número de pontos
positivos desta abordagem HTML para desenvolvimento de jogos é muito
grande em comparação as limitações.


É seguro dizer que ainda o HTML não está pronto para substituir todas
as estratégias de desenvolvimento de jogso. Não obstante as tecnolgias
da WEB podem ser aplicadas ao desenvlvimento de jogos em uma vasta gama
de casos.

Existem funcionalidades interessantes para jogos que não funcionam em
todos os navegadores. Por exemplos a possibilidade de criar um ícno da
aplicação WEB através do Safari.



Se uma empresa deseja produzir jogos nativos elas precisarão de vários
desenvolvedores. Eu sozinho fui capaz de produzir um jogo em tempo
razoável trabalhando com a plataforma WEB.

Por não utilizar frameworks e bibliotecas estou me distanciando
dos casos da vida real.

Só poderemos considerar o HTML como uma especificação pronta quando
for possível fazer tudo o que se faz nativamente com os dispositivos
através de uma API WEB padronizada.



Muitas das limitações do HTML5 são contornáveis através de JavaScript.

O futuro dos jogos em HTML5 parece brilhante.


Este trabalho surgiu da minha vontade de aprender mais sobre a WEB....

O suporte ao HTML vem crescendo com o tempo, o site HTML5Test
\url{http://html5test.com/about.html}, oferece um placar atualizado
dinamicamente, conforme utilização dos navegadores, sobre os recursos
do HTML.

Apesar de ser difífil de prover a mesma experiência em duas plataformas diferentes, é possível utilizar algumas plataformas com funcionalidades limitadas enquanto outras recebem a expeirência completa dos jogos \autocite[pp. 1]{currentStateCrossPlatform}.

\subsection{TRABALHOS FUTUROS}
A manipulação de erros é bastante consistente nos navegadores mas 
incrivelmente não faz parte da especificação \autocite{howBrowsersWork}.
Não pude testar todos os métodos e ferramentas e versões à
disposição, um trabalho completo demandaria esforços conjuntos de
muitos indivíduos ou um período de tempo bem mais extenso. 

Trabalhos que explorem os benefícios mercadológicos do HTML5 em comparação com alternativas nativas.
EMACSCRIPT 7.

HTTP2 também traz boas perspectivas em relação a performance.

Um trabalho com criação de protótipos em WebGL seria interessante.

A utilização de plugins foi interacionalmente ignorada trabalhos
que analisem a construção de um jogo comercial utilizando plugins,
da forma mais aproximada possível da realidade mercadológica seria
igualmente interessante.

Não obstante, otimizações de performance dependem do ambiente em
que estão sendo feitas. E aquelas que hoje tem um impacto positivo
hoje podem se tornar inúteis, ou mesmo prejudiciais, amanhã
\autocite[pp.131]{html5mostwanted}.

Um exemplo disso se dá com a introdução do HTTP/2. Minificar arquivos
pode não ser mais benéfico, visto que todos passam pelo mesmo canal
de comunicação, não havendo mais problemas de bloqueio de máximas
requisições.

\end{draft}
