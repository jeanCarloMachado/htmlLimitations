
Neste trabalho revisamos tecnologias relevantes no desenvolvimento
de jogos multiplataforma em HTML5 e algumas das limitações a elas
relacionadas.

Para tanto, elaborou-se uma pesquisa bibliográfica
dos assuntos relevantes. Em seguida, criou-se um protótipo de jogo
para avaliar experimentalmente algumas das tecnologias e detectar
limitações. Apesar de simples, o protótipo desenvolvido ajudou
bastante na validação de algumas limitações previamente encontradas
e na detecção de alguns problemas adicionais os quais foram citados
previamente.

As limitações encontradas através do processo mencionado foram
registradas e avaliadas. Os resultados obtidos nos levam a crer que,
apesar de o número de limitações ser substancial, a grande maioria
destas limitações podem ser contornadas pelo programador ou já estão
no processo de serem solucionadas pelas especificações e navegadores.

As tecnologias da Web vem evoluindo a grande passo e número de pontos
positivos da abordagem HTML para desenvolvimento de jogos é muito
grande em comparação as limitações.

É seguro dizer que o HTML ainda não está pronto para substituir todas
as estratégias de desenvolvimento de jogos. Não obstante, as tecnologias
da Web podem ser aplicadas ao desenvolvimento de jogos em uma gama
crescente de casos.

No contexto multiplataforma, apesar de ser difícil de prover a
mesma experiência em duas plataformas diferentes, é possível
utilizar algumas plataformas com funcionalidades limitadas enquanto
outras recebem a experiência completa dos jogos \autocite[p.
1]{currentStateCrossPlatform}.

%provavelmente os problemas da Web mais difíceis de resolver são 
%os que conflitam com os princípios da mesma.

Da mesma maneira, existem várias funcionalidades interessantes para
jogos que ainda não estão prontas ou, por hora, não funcionam
em todos os navegadores como: WebVR, ES6, HTTP/2, Gamepad, entre
outras. Mas estes problemas, se considerarmos o histórico do HTML -
principalmente após o lançamento do HTML5, são passageiros. A figura
\ref{fig:htmlSupport} corrobora com esta hipótese, o crescimento do
suporte das tecnologias nesta imagem apresenta uma curva aparentemente
exponencial. Ao que tudo indica o futuro dos jogos em HTML5 parece
brilhante.

\section{Trabalhos Futuros}

Visto que a especificação do HTML é viva, checar suas limitações
é uma tarefa que deveria ser feita de tempos em tempos. Tecnologias
como o EMACScript 6 e Web Assembly podem mudar completamente o cenário
de desenvolvimento de jogos Web assim que se tornarem opções viáveis,
comercialmente falando.

Seria interessante que assuntos como o WebGL e WebVR, levemente
abordados neste trabalho, fossem estudado com profundidade visto que
são deveras importantes para a criação de jogos cada vez mais
iterativos. O suporte a mais versões de navegadores e plataformas
bem como outras metodologias de desenvolvimento também poderiam ser
estudadas.

A utilização de plugins foi intencionalmente ignorada deste
trabalho. Conquanto, em jogos comerciais, estas ferramentas são
imprescindíveis. Trabalhos que analisem a construção de um jogo sob
a perspectiva comercial, utilizando plugins, na forma mais aproximada
possível da realidade mercadológica, seriam igualmente interessantes.

