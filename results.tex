Grande parte da detecção de funcionalidades é feita através de JavaScript, isso força os desenvolvedores a criarem pelo menos parte da marcação em JavaScript, isso pode ser um fator limitante para o uso generalizado de HTML5 \autocite{diveIntohtml}.

Abaixo constam as limitações encontradas durante a pesquisa e concepção do jogo.

Muitos dos problemas dos jogos multiplataforma não são específicos
dos jogos, mas aplicam-se a todos tipos de software  \parencite{currentStateCrossPlatform}.

\section{LIMITAÇÕES}

Apesar da grande maioria dos recursos dos dispositivos estar presente
em HTML5 ainda existem muitas funcionalidades faltando para este tipo
de aplicação. Por exemplo, não podemos mudar a imagem de fundo do
dispositivo, ou adicionar toques etc. Similarmente, existem muitas
APIs de nuvem como os serviços de impressão do ICloud ou Google
cloud que estão disponíveis para aplicações nativas mas não para
HTML5. Outros serviços utilitários como o C2DM do Google que está
disponível para desenvolvedores Android para utilizar serviços de push
também não estão disponíveis para o HTML5. (HASAN, 2012)

\subsection{VERSÕES}
A grande maioria dos dispositivos atualmente no mercado utilizam
obsoletas de seus softwares. Isso dificulta o desenvolvimento. Se a
tecnologia de tradução para o navegador utilizar o a classe Webview do
Android - como o Apache Cordova faz - as versões mais antigas podem ser
penalizadas com problemas de performance ou falta de recursos.

\subsection{OFFLINE}

Refresh duplo para ver assets cacheados. Ver:
http://buildnewgames.com/game-asset-management/

\subsection{AUDIO}
Api de som quebra quando executado diversas vezes.
Os navegadores variam na disponibilização de formatos aceitáveis
Somente um áudio pode ser tocado no Navegador do Android
Não é possível trocar o volume no IOS.
Alguns navegadores favorecem formatos ogg (vorbis) e outros, como o
Safari, favorecem o MP3.

O maior problema com as API's de áudio e de vídeo do HTML5 é
a disputa entre os codecs dos navegadores. Por exemplo, Mozilla e
Opera suportam Theora, já o Safari suporta H.264 que também é
suportado pelo IE9. Ambos, Iphone e Android suportam H.264 em seus
navegadores. A W3C recomenda OggVorbis e OggTheora para áudio e vídeo
respectivamente. (HASAN et al, 2012)

\subsection{SVG}

Segundo \cite{html5mostwanted} a grande desvantagem do SVG é que quão
maior o documento mais lenta a renderização.

\subsection{VIDEO}

Codecs

4. ASSETS

Trafegar muitos assets deixa o sistema lento.

 Contorno
Utilizando páginas de carregamento e/ou cache;

5. UI

É muito custoso desenvolver uma interfaces que pareçam nativas
para cada dispositivo sem a utilização de plugins e ferramentas
especializadas. Em termos gerais, trabalhar com proporções é
positivo. Não obstante
há casos, como o dos botões de certo e errado que a proporções ficam
exageradas, nesses casos a utilizada de max-width é uma solução
conveniente.

6. PERFORMANCE

De acordo com uma pesquisa, para um usuário uma tarefa é instantânea
se ele leva até 0.1 segundos para ser executada. Se a tarefa toma
aproximadamente um segundo então a demora será notada mas o
usuário não se incomodará com ela. Entretanto, se a tarefa leva
aproximadamente 10 segundos para terminar o usuário então começa a
ficar aborrecido e esse é o limite que algum feedback deve ser dado
para um usuário.

ACELERAÇÃO DE GPU

7. Acelerômetro

8. IMPLEMENTAÇÃO INCONSISTENTE DE APIs

9.  TAMANHO DE TELA
Em alguns casos o tamanho das telas pode ser um fator limitante – como
no caso de jogos de estratégia. Jogadores com telas menores podem sair
em desvantagem. 9. CÂMERA

10 . JavaScript
Ciclo de vida demorado pois necessita que todos os consumidores da
especificação entrem em consenso e implementem a.

11  Detecção de recursos
Muitas das detecções de recursos é feita via JavaScript, isso
força os desenvolvedores a fazer ao menos parte da marcação em
JavaScript\autocite{diveIntohtml}.

Desktop/Firefox
Desktop/Google Chrome
Smatphone/Android
