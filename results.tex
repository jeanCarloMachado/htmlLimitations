
Abaixo constam as limitações encontradas durante a pesquisa e concepção do jogo.


Muitos dos problemas dos jogos multiplataforma não são específicos
dos jogos, mas aplicam-se a todos tipos de software  \parencite{currentStateCrossPlatform}.


Alguns problemas são inerentes das tecnologias, outras dos dispostivos ou da característica multiplataforma.

\section{LIMITAÇÕES}

Apesar da grande maioria dos recursos oferecidos nativamente nos dispositivos estar presente
em HTML5 ainda existem algumas funcionalidades faltando para este tipo
de aplicação.

\cite{html5Tradeoffs} cita algumas limitações no contexto geral.

Não podemos mudar a imagem de fundo do
dispositivo, ou adicionar toques etc. Similarmente, existem muitas
API's de nuvem como os serviços de impressão do ICloud ou Google
Cloud que estão disponíveis para aplicações nativas mas não para
HTML5. Outros serviços utilitários como o C2DM do Google que está
disponível para desenvolvedores Android para utilizar serviços de push
também não estão disponíveis para o HTML5.

\cite{browserGamesTechnologyAndFuture} apresenta as seguintes limitações 
no contexto de jogos de navegador

\begin{itemize}
\item Não podem ser instalados em um dispositivo móvel como um aplicativo separado
\item Não tem acesso a funcionalidades específicas dos dispositivos e recursos como notificações, use de hardware nativo ou comunicação entre aplicativos.
\end{itemize}


\subsection{VERSÕES}
A grande maioria dos dispositivos atualmente no mercado utilizam
obsoletas de seus softwares. Isso dificulta o desenvolvimento. Se a
tecnologia de tradução para o navegador utilizar o a classe Webview do
Android - como o Apache Cordova faz - as versões mais antigas podem ser
penalizadas com problemas de performance ou falta de recursos.


Segundo \cite{crossPlatformMobileGame}
\begin{quote}
Enquanto o HTML é desenvolvido muitas das funcionalidades
disponibilizadas são testadas em um pequeno conjunto de
navegadores para um pequeno conjunto de versões. Isso
acarreta em suporte inconsistente. A forma mais segura de garantir
suporte é testando em todas as versões alvo, todavia essa solução
não é prática.
\end{quote}


\subsection{OFFLINE}

Refresh duplo para ver assets cacheados. Ver:
http://buildnewgames.com/game-asset-management/

Quando o download offline falha o browser emite um evento mas não há
indicação de qual problema aconteceu. Isso pode tornar a depuração
tinda complicada que o usual \autocite{diveIntohtml}.

\subsection{SVG}

Segundo \cite{html5mostwanted} a grande desvantagem do SVG é que quão
maior o documento mais lenta a renderização.

\subsection{Canvas}

Os aspectos negativos do canvas é que a performance varia de plataformas para plataformas e não existe implementação nativa para animações \autocite{html5mostwanted}.

\subsection{WebGL}

Como WebGL é baseada na versão otimizada para dispositivos móveis do OpenGL não é possível utilizar muitos recursos especiais disponível para os ambientes desktop.

Segundo \cite{html5mostwanted} um dos problemas do WebGL é sua alta
curva de aprendizagem e o fato de não ter suporte para o Internet
Explorer. Entretanto o suporte foi adicionado na última versão do
Internet Explorer (11). Não obstante a dificuldade de utilização
ainda persiste, forçando a maioria dos desenvolvedores a utilizarem
abstrações criadas por bibliotecas de terceiros.

Para ver um programa em WebGL é necessário um navegador recente, uma placa gráfica recente e um sistema operacional que suporte a tecnologia \autocite{html5mostwanted}

\subsection{VIDEO}

\subsection{ASSETS}

Trafegar muitos assets deixa o sistema lento.  Pode-se contornar este problema utilizando páginas de carregamento e/ou cache;

Outro problema relativo aos assets é descrito por \cite{howBrowsersWork}
scripts requerendo informações de estilo durante o processo de 
parsing. Se o estilo ainda não foi carregado o script vai utilizar 
informações erradas, causando uma série de problemas.

\subsection{Codecs}

\subsection{ÁUDIO}

Segundo \cite{html5mostwanted} a limitação do elemento de audio do HTML5 é que seu propósito é para executar apenas um som, como o som de fundo dentro de um jogo.

A API de som no protótipo nem sempre emite o som quando o evento é disparado
sucessivamente. Confirmando a afirmação de \cite{html5mostwanted} a API de som é boa se você deseja apenas tocar alguma música, mas se você está lançando eventos em um jogo ela é problemática.

Os navegadores variam na disponibilização de formatos aceitáveis
Somente um áudio pode ser tocado no Navegador do Android

Não é possível trocar o volume no IOS.

Alguns navegadores favorecem formatos ogg (vorbis) e outros, como o
Safari, favorecem o MP3.

Segundo \cite{html5Tradeoffs}
\begin{quote}
O maior problema com as API's de áudio e de vídeo do HTML5 é
a disputa entre os codecs dos navegadores. Por exemplo, Mozilla e
Opera suportam Theora, já o Safari suporta H.264 que também é
suportado pelo IE9. Ambos, Iphone e Android suportam H.264 em seus
navegadores. A W3C recomenda OggVorbis e OggTheora para áudio e vídeo
respectivamente.
\end{quote}

\subsection{INTERFACE GRÁFICA}

É muito custoso desenvolver uma interfaces que pareçam nativas
para cada dispositivo sem a utilização de plugins e ferramentas
especializadas. Em termos gerais, trabalhar com proporções é
positivo. Não obstante
há casos, como o dos botões de certo e errado que a proporções ficam
exageradas, nesses casos a utilizada de max-width é uma solução
conveniente.

\subsection{PERFORMANCE}

De acordo com uma pesquisa, para um usuário uma tarefa é instantânea
se ele leva até 0.1 segundos para ser executada. Se a tarefa toma
aproximadamente um segundo então a demora será notada mas o
usuário não se incomodará com ela. Entretanto, se a tarefa leva
aproximadamente 10 segundos para terminar o usuário então começa a
ficar aborrecido e esse é o limite que algum feedback deve ser dado
para um usuário.

Otimizações de performance dependem do ambiente em que estão        .
sendo feitas E aquelas que hoje tem um impacto positivo hoje podem     .
se tornar inúteis, ou mesmo prejudiciais, amanhã \autocite[pp.       .
131]{html5mostwanted}                                                  .

\subsection{Acelerômetro}

\subsection{IMPLEMENTAÇÃO INCONSISTENTE DE APIs}

\subsection{TAMANHO DE TELA}
Em alguns casos o tamanho das telas pode ser um fator limitante – como
no caso de jogos de estratégia. Jogadores com telas menores podem sair
em desvantagem.

\subsection{JavaScript}

O JavaScript, por ser uma tecnologia desenvolvida por consenso, tem um
ciclo de vida de atualizações demorado; pois necessita que todos os
consumidores da especificação entrem em consenso e implementem a.

Apesar da performance ter notavelmente melhorado, ainda é geralmente
menos eficiente produzir animações em JavaScript do que utilizando
transições e animações do CSS, que por sua vez são mais otimizados
e acelerados via hardware \autocite{html5mostwanted}.

Erros numéricos resultam no valor NaN (\textit{not a number}).
Todas as operações com NaN como operadores irão retornar outro
NaN. Isso torna a depuração de erros desnecessariamente complexa
\autocite{html5mostwanted}.

\subsection{Detecção de recursos}

Grande parte da detecção de funcionalidades é feita através de
JavaScript, isso força os desenvolvedores a criarem pelo menos parte da
marcação em JavaScript, isso pode ser um fator limitante para o uso
generalizado de HTML5 \autocite{diveIntohtml}.

Desktop/Firefox
Desktop/Google Chrome
Smatphone/Android

