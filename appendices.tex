\chapter{Diagrama de classes}

A figura \ref{fig:fullDiagram} é o diagrama de classes detalhado do protótipo.
\begin{figure}
    \centering
    \includegraphics[width=0.8\textwidth,natwidth=610,natheight=642]{ClassesFullView.png}
	\caption{Diagrama de classes completo}
    \label{fig:fullDiagram}
\end{figure}


\chapter{Bibliotecas relevantes no desenvolvimento de jogos WEB}

A quantidade de ferramentas WEB á disposição dos usuário é
enorme. Segundo \autocite{html5mostwanted} desenvolvedores da WEB são
geralmente de mente aberta e desenvolveram uma variedade de bibliotecas e frameworks pela internet. Dessa forma, é uma tarefa praticamente
impossível ter habilidade em todas as ferramentas disponíveis.
Não obstante é importante que os desenvolvedores ao menos conheça
as ferramentas mais importantes disponíveis para que no momento
necessário saibam qual aprender.

\cite{creatingFun} ressalta a importância de sermos moderados
quanto a escolha de bibliotecas no contexto WEB multiplataforma.
\begin{quote}
Muitos desenvolvedores da WEB utilizam bibliotecas como jQuery e o
Prototype de modo que se vejam livres de terem que lidar com
partes triviais do desenvolvimento WEB, como selecionar e
manipular elementos do DOM. Muitas vezes essas bibliotecas incluem
várias funcionalidades que não são utilizadas. É recomendável
cautela para verificar se realmente é necessário adicionar 50-100k
de bibliotecas, ou se alguma coisa mais simples e menor não trará
os mesmo benefícios, especialmente quando desenvolvendo
multiplataforma onde uma rápida conexão a internet nem sempre é
garantida.
\end{quote}

Esta preocupação aumenta no contexto de desenvolvimento de jogos.
Ainda segundo \cite{creatingFun} o site MicroJS https://microjs.com
oferece uma coleção de micro bibliotecas focadas em áreas
particulares em detrimento de grandes bibliotecas cheias de
funcionalidades.

\section{WebGL}

CocoonJS é uma aplicativo híbrido que preenche a fraca implementação
de WebGL nos dispositivos móveis possibilitando se desenvolver em
WEBGL, CSS. Conta com suporte a dispositivos legados à partir do
Android 2.3 e IPhone 5.

Treejs é um framework popular para o desenvolvimento em WebGL.
Consistem em uma abstração sobre WebGL que permite os autores se
focarem na criação de conteúdo para WEB, ao invés de dispenderem
tempo manipulando os detalhes da WebGL. Possibilita trabalhar com
efeitos, luzes, cenas e outras abstrações em detrimento de shaders,
vértices, e conceitos mais primitivos.

\section{Frameworks de jogos}

Motores de jogos são bibliotecas que agregam várias funcionalidades
usualmente úteis para o desenvolvimento de jogos \autocite[pp.
5]{browserGamesTechnologyAndFuture}. Elas podem incluir controle
de usuário, cena, áudio, física, etc. Também servem como uma camada de 
segurança adicional \autocite{browserGamesTechnologyAndFuture}.

Outrossim,
motores de jogos não são amplamente difundidos no mercado de jogos de
navegadores \autocite{browserGamesTechnologyAndFuture}.
Com o intuito de simplificar o processo para os desenvolvedores,
auxiliando-os a focarem-se apenas nas soluções que estão
desenvolvendo, foram criados os frameworks para desenvolvimento de
jogos. Alguns frameworks reconhecidos são:

enchant.js: dentre suas funcionalidades constam: orientação à, orientado à eventos, contém um motor de animação,
suporta WebGL e Canvas, etc three.js: considerada leve, renderiza, WebGL e Canvas, arquitetura procedural
limeJs: bom para 2d
quintus: especialista em jogos de plataforma 2D

\section{CROSSWALK}

Crosswalk empacota os fontes juntamente com uma versão do Chromium, a
versão Open-source do Google Chrome. Isso faz com que o software se
comporte da mesma forma para todas as versões de dispositivos Android.

\section{PHONEGAP}

PhoneGap é uma plataforma que permite o desenvolvimento de aplicativos híbridos baseados nas tecnologias da WEB.

\section{PHONEGAP CLOUD}

Este serviço possibilita que se faça upload de um arquivo compactado
contendo os fontes – ou apontando para um repositório no GitHub –
que no tempo desta pesquisa não estava funcionando; e se gere o APK
para o Android nativamente.

\chapter{AMBIENTES PARA DESENVOLVIMENTO HTML5}

Na pesquisa efetuada sobre estes frameworks full-stack foram
identificadas as seguintes tecnologias:

Segundo \cite[pp. 29]{gtw}
\begin{quote}
O GWT é um framework essencialmente para o lado do cliente (client
side) e dá suporte à comunicação com o servidor através de RPCs
(\textit{Remote Procedure Calls}). Ele não é um framework para
aplicações clássicas da web, pois deixa a implementação da
aplicação web parecida com implementações em desktop.
\end{quote}

Com o GWT se programa em Java e exporta-se com código otimizado para
as plataformas alvo. Sua licença é Apache 2 e, como o framework é em Java,
pode ser rodado em todos os sistemas operacionais.

Construct 2: é um editor na nuvem focado para usuários sem
conhecimento prévio em programação orientado a comportamento;

PlayCanvas: é uma plataformas para a construção de jogos 3D
na nuvem, desenvolvida com foco em performance. Permite a hospedagem,
controle de versão e publicação dos aplicativos nela criados,
possibilita também a importação de modelos 3D de softwares populares
como: Maya, 3ds Max e Blender;

O Intel® HTML5 Development Environment  fornece uma solução na nuvem,
completa para o desenvolvimento em plataforma cruzada, com serviços de
empacotamento, serviços para a criação e testes de aplicativos com
montagem de interfaces \textit{drag and drop} (Intel XDK) e bibliotecas para a
construção de jogos utilizando aceleração de hardware, o que garante
até duas vezes mais performance que aplicativos mobile baseados em
Web tradicionais. 

Esta solução é gratuita, open-source e funciona através de um
plugin para o Google Chrome, ou seja, o desenvolvimento também é
multiplataforma e devido ao fato de os binários ficarem hospedados
na nuvem, possibilitou a Intel criar compiladores para cada uma das
plataformas disponibilizadas pelo PhoneGap, que é o framework polyfill
utilizado na solução.

O website https://codepen.io e https://jsfiddle.net permitem a construção e visualização de aplicações utilizando as tecnologias da WEB. Apesar de ser possível, criação de aplicações utilizando estas tecnologias não é muito comum, servindo principalmente para visualizar trechos de códigos de exemplos na WEB.

\chapter{ALTERNATIVAS AO JAVASCRIPT}

Abaixo seguem algumas tecnologias que servem de alternativa ao
JavaScript.

\section{TYPESCRIPT}

Conhecido como uma versão estendida do JavaScript que compila para
JavaScript normal.

\section{DART}

Google. DartVM é uma máquina virtual que está embebido no Google
Chrome. Significante melhorias em performance quando comparado
ao JavaScript. Existe o dart2js que compila código em Dart para
JavaScript.

\chapter{CONVERSORES PARA HTML}

Além da possibilidade de escrever em HTML, pode-se optar pela
alternativa de utilizar-se um conversor de linguagens.

\chapter{METODOLOGIAS DE CRIAÇÃO DE SOFTWARE PARA GAMES}

Como o jogo é um software complexo demanda-se a utilização de
metodologias de engenharia de software, dentre os processos de software
mais conhecidos academicamente destacamos:

OpenUP: este é bem detalhado e de característica iterativa e
incremental. Gerando assim, um levantamento mais apurado dos riscos,
requisitos e outros detalhes do sistema e a criação incremental do
sistema, com requisitos maleáveis;

Cascata: processo antigo, caracteriza-se por ser pouco maleável aos
requisitos mapeados posteriormente ao processo de análise;

Processo ágil - SCRUM: sua utilização é flexível e sendo
um método ágil especifica pouca documentação, ou como dizem,
somente a documentação necessária, este processo é bem conhecido e
aceito na comunidade de desenvolvimento de software. Suas principais
características são: divisão do processo de desenvolvimento através
uma série de iterações chamadas sprints. Cada sprint consiste
tipicamente em duas a quatro semanas. É bem aplicado a projetos que
mudam constantemente e que demandam rápidas adaptações;

Processo ágil – XP: tem muitas características similares ao SCRUM
por este também ser um processo ágil. Dentre suas especifidades
destaca-se: versões frequentes, pequenos ciclos de desenvolvimento que
buscam aumentar a produtividade, introduzem checkpoints onde os clientes
podem agregar novas funcionalidades;

\chapter{SISTEMAS DE BUILDING}

Aquivos JavaScript são requisitados do servidor assincronamente. Isso
pode levar a tempos de requisição pouco desejáveis. Uma saída seria
escrever o código em apenas um arquivo mais isso leva a gerência de
código bagunçada. A saída mais comum entre desenvolvedores é utilizá
ruma ferramenta que junta todos os arquivos e disponibiliza apenas um
para o usuário.

Utiliza o conceito de streams para aplicar todas as modificações sobre
um arquivo de uma vez só.

