%{{{
\documentclass[
11pt,
a4paper,
portuges,
draft
]{report}

\usepackage[portuges]{babel}
\usepackage[utf8]{inputenc}
\usepackage[T1]{fontenc}
\usepackage{fancyhdr}
\pagestyle{fancy}
\fancyhf{}
\renewcommand{\headrulewidth}{0pt}
\fancyhead[R]{\thepage}
\usepackage{graphicx}
\graphicspath{ {asset/} } 
\usepackage[style=authoryear,backend=bibtex,sorting=nty]{biblatex}
\addbibresource{tcc.bib}

\addto\captionsportuges{
\renewcommand{\contentsname}{Sumário}
\renewcommand{\chaptername}{}
}

\usepackage[
top=3cm,
left=3cm,
bottom=2cm,
right=2cm
% inner=1.5in, % The inner margin (beside binding)
% outer=1in, % The outer margin (opposite binding)
% headheight=20pt, % Header height
% headsep=.25in, % Header separation
% includehead,
% includefoot
]{geometry}

\newcommand{\supervisor}{Mr. Rafael Jaques}
\title{Limitações do HTML5 \\ no desenvolvimento de jogos multiplataforma}
\author{Jean Carlo Machado}
\newcommand{\university}{\uppercase{Instituto Federal de Educação, Ciência e Tecnologia do Rio Grande do Sul}}
\newcommand{\campus}{Campus Bento Gonçalves}
\newcommand{\website}{http://jeancarlomachado.com.br}
\date{Bento Gonçalves, Dezembro 2015}

\begin{document}
%}}}

\maketitle
\tableofcontents
\listoffigures
\listoftables

\chapter{CONTEXTUALIZAÇÃO}%{{{

\section{JOGOS}

\subsection{BENEFÍCIOS}

\subsection{O MERCADO}
A maior dificuldade em capturar uma base de usuários é que o mercado
de dispositivos móveis é muito fragmentado e não existe uma única
plataforma popular. (HASAN, 2012)

\subsection{JOGOS E MULTIPLATAFORMA}

\subsection{HTML E MULTIPLATAFORMA}

Desenvolvedores de jogos web podem rapidamente satisfazer as
necessidades de seus jogadores, mantendo-os leais a tecnologia HTML5
(ZHANG, 2012).

> A maioria dos desenvolvedores demonstra interesse para o HTML5.

> O tempo de desenvolvimento de uma aplicação em HTML5 é 67% menor
que aplicações nativas. Isso mostra o custo efetivo de aplicações
baseadas em HTML5. A real vantagem de aplicações em HTML5 é o suporte
horizontal entre as plataformas - que é a maior razão por trás do
custo efetivo. (HASAN et al, 2012)

\subsection{LIMITAÇÕES DE JOGOS MULTIPLATAFORMA COM HTML5}

> Funcionalidades foram disponibilizadas de diversas fontes e não foram
construídas de forma especialmente consistente com as demais. Além
disso, devida a única característica da Web, erros de implementação
se tornam frequentes, e muitas vezes se tornam o padrão, pois outras
funcionalidades dependem destas primeiras antes que elas estejam
estáveis. (W3C manual)

Enquanto o HTML é desenvolvido muitas das funcionalidades
disponibilizadas são testadas em apenas um pequeno conjunto de
navegadores para um pequeno conjunto de versões (referência 2). Isso
acarreta em suporte inconsistente. A forma mais segura de garantir
suporte é testando em todas as versões alvo, todavia essa solução
não é prática. (ref. 2)

Os desenvolvedores de navegadores podem interpretar/implementar
as especificações erroneamente aumentando os problemas de
compatibilidade.

Nem todos os recursos disponíveis através das SDK's nativas estão
presentes através do HTML5.

\section{ESTE TRABALHO}

Este projeto propõe analisar as limitações do HTML5 quanto relativo
a construção de jogos multiplataforma. Através de revisão
bibliográfica e da criação de um protótipo de jogo multiplataforma.

Um tratado completo sobre o assunto requiriria um comparativo entre
jogos desenvolvidos nativamente e jogos em HTML5.

\subsection{O JOGO}

Para a análise prática das limitações foi escolhido um jogo de
matemática simples. Consistindo na geração de equações com um
candidato de resposta. Cabe ao usuário informar se o resultado apontado
pelo jogo está correto ou não.

Porquê escolhi esse tipo de jogo?
%}}}

\chapter{PROBLEMA}
%{{{

A carência de definições concretas sobre a viabilidade da atual
versão do HTML5 - aplicado ao desenvolvimento de jogos. O senso comum, 
acostumado com soluções nativas, acabam por monopolizar à construção de jogos nativos.

Os custos introduzidos no ciclo vida de um jogo, para diversas
plataformas, é muito alto para ser considerado trivial. Cerca de 65%
mais altos (segundo trabalho 2)
%}}}

\section{OBJETIVOS}
%{{{

\subsection{OBJETIVO GERAL}
%{{{

Identificar possíveis limitações no processo de desenvolvimento
de jogos multiplataforma oriundas do atual estado de definição e
implementação do HTML5.

Não é objetivo deste trabalho demonstrar os pontos fortes do HTML5,
apenas suas limitações. Também não é objetivo deste trabalho
comparar o HTML com outras tecnologias de desenvolvimento de jogos, c
omo Flash Player, Silverlight ou alternativas Desktop.
%}}}

\subsection{OBJETIVOS ESPECÍFICOS}

Estudar as limitações de desenvolvimento de jogos nas plataformas de
dispositivos inteligentes Android e navegadores Desktop Google Chrome
e Firefox. Os navegadores foram testados em suas últimas versões.

Optamos por Android, e não IOS, pois o primeiro contém
a vasta maioria do mercado de dispositivos inteligentes, e por termos
maior experiência na já mencionada plataforma.

Pretende-se também estudar os seguintes tópicos do desenvolvimento de
jogos, relativos ao HTML5:

\cite{browserGamesTechnologyAndFuture} cita que video, audio, drag and drop, funcionalidades de gráficos e armazenamento off-line são aspectos importantes do HTML5 para o desenvolvimento de jogos.

Com base nos estudos efetuados e na experiência adquirida elencamos os seguintes itens que serão abordados neste estudo.

\begin{itemize}
\item Performance
\item Depuração
\item Diferenças em tamanho de tela
\item Canvas
\item Empacotadores HTML5
\item Eventos de entrada
\item Vibração
\item Acelerômetro
\item Armazenamento
\item Disponibilização de assets (controle de tamanhos, cache, etc)
\item Aplicações offline
\item CSS media queries
\end{itemize}

Elaborar uma lista de limitações e correlacionar os dados de acordo
com as plataformas.

%}}}

\chapter{JUSTIFICATIVA}
%{{{

Tendo em vista que este trabalho busca mapear possíveis problemas
do desenvolvimento multiplataforma em HTML ele serve para apoiar
e justificar decisões relativas ao desenvolvimento de jogos
multiplataforma.

Tem potencial apontar os pontos chave que necessitam
de melhorias no HTML, colateralmente colaborando para a
melhoria da própria especificação.

A opinião comum tende para soluções nativas em detrimento do
desenvolvimento de jogos, este trabalho pretende desafiar esta
concepção. %REFERENCIAR

Muitos desenvolvedores estão familiarizados
com as tecnologias da WEB ou apontam interesse na tecnologia.

Estimular e avançar o estudo da implementação da Open Web;

%}}}

\chapter{REVISÃO BIBLIOGRÁFICA}
%{{{

\section{JOGOS}
%{{{
Segundo LEMES (2009, pág. 126) jogo digital constitui-se em uma
atividade lúdica composta por uma série de ações e decisões,
limitada por regras e pelo universo do game, que resultam em uma
condição final.

Essa característica interativa e a dependência comandos
sobre uma interface digital, que faz com que o projeto digital desta
natureza não seja um filme ou uma animação, e sim um game.

Video games melhoram as funções cognitivas, melhoram as capacidades criativas, e 
motivam uma visão positiva diante a falha. \cite{gamebenefits}

Jogos em plataformas móveis trazem um novo conjunto de desafios para
produtores de jogos. Um destes desafios é fornecer feedback suficiente
para o jogador pois o dispositivo é limitado em proporções, som, tela
etc.

Já jogos multiplataforma em HTML5 tem a dificuldade adicional
de ter que comportar, na mesma base de código, o feedback adequando para cada
plataforma móvel.

A interface tem que ser o mais intuitiva o possível. No caso de
dispositivos móveis, quanto menos gestos necessários melhor. Tornar
previsível causa e efeito é uma boa característica para os jogos.
Os desenvolvedores tem que evitar fazer o jogo para eles mesmos.
E pela falta de crítica os designs tendem a ser ruins. Afinal o
que os jogadores querem? LEMES (2009, pg XX) aponta alguns fatores procurados
pelos usuários de jogos: Desafio, socializar, experiência solitária,
respeito e fantasia.

Mencionar algum jogo (como WOW) e como ele faz para prender a
atenção dos usuários. Candy Crush saga 

\subsection{GÊNEROS}

Segundo \cite{gamebenefits}:

\begin{quote}
    Pela diversidade  em itens e gêneros e a vasta quantidade de dimensões que os video games se encontram, uma taxonomia dos jogos contemporâneos é extremamente difícil de desenvolver (muitos já tentaram).
\end{quote}

- Adventure Ação RPG Estratégia Simuladores Esportes Luta Casuais
- 'God' Games Educacionais Puzzle Online / Massive Multiplayer

\subsection{MECÂNICA}

A mecânica é composta pelas regras do jogo. Quais as ações
disponíveis aos usuários, é fortemente influenciada pela categoria do
jogo em questão.

\section{ARQUITETURA MULTIPLATAFORMA}

Designers de jogos tem as seguintes possibilidades arquiteturais
quando em face de desenvolver um novo jogo: Criar um jogo web,
um jogo híbrido, ou nativo. As opções serão descritas abaixo.

\subsection{JOGOS WEB}

Um jogo web é um jogo que utiliza o HTML e
ferramentas correlacionadas para sua construção. Este tipo de jogo
é o que será abordado neste trabalho.

Entre seus pontos positivos pode-se listar:

\begin{itemize}
\item Necessitam de apenas uma base de código e pode rodar em todas as
plataformas;
\item Contém a mais vasta gama de desenvolvedores e muitos
interessados em aprendê-la;
\item Seus custos são inferiores, aos do desenvolvimento nativo devido a 
inexistência de duplicação da base de código;
Os pontos negativos dessa abordagem são o principal foco deste trabalho.
Mas a um nível macroscópico podemos listar:
\item Programas que rodam na web são geralmente mais lentos que os
nativos;
\item Por falta de especificação ou incompletude de implementação.
\end{itemize}

\subsection{JOGOS HÍBRIDOS}

Jogos híbridos são jogos geralmente desenvolvidos com tecnologias da
web. Mas rodam dentro de um contêiner nativo – possibilitando o
acesso à chamadas do sistema, recursos de hardware, eliminando muitias
das dificuldades da web.

\subsection{DESENVOLVIMENTO DE JOGOS NATIVOS}

Habilita a melhor experiência de usuário pois permite utilizar ao
máximo os recursos e funcionalidades dos aparelhos. 

Porém, devido a cada plataforma conter seu próprio sistema operacional, 
com seus próprios *SDK's* totalmente incompatíveis, os desenvolvedores são
forçados a desenvolver uma versão do jogo para cada plataforma alvo.

Além da replicação dos fontes, esta abordagem requer mais pessoas, e
maior custo com possivelmente parte do mercado não atendido de qualquer
forma.

%}}}

\section{WEB}
%{{{

\subsection{OPEN WEB}

A OWP (\textit{Open Web Platform}), uma coleção de tecnologias livres,
amplamente utilizadas e padronizadas. 
Quando uma tecnologia se torna amplamente popular, através da
adoção de grandes empresas e desenvolvedores ela se torna candidata a
adoção pela OWP.

Mais do que um conjunto de tecnologias Open Web, é um conjunto de
filosofias as quais a web se baseia.

Neste conjunto inclui-se:

\begin{itemize}
\item Descentralização;
\item Transparência;
\item Relevância;
\item Imparcialidade;
\item Consenso;
\item Disponibilidade;
\item Manutibilidade;
\end{itemize}

A tecnologia chave que inaugurou e alavancou este processo é o HTML.
%}}}

\section{HTML}
%{{{

HTML (\textit{Hyper Text Markup Language}) é uma linguagem de marcação
que define a estrutura semântica do conteúdo das páginas da
web, criada por Tim Berners Lee e oficialmente mantida pela W3C.

O "Hyper Text" refere-se a links que conectam páginas umas as
outras, fazendo a Web como conhecemos hoje \cite{mdn2015}.

HTML foi especificado baseado-se em SGML (Standard Generalized Markup Language),
abraçando assim suas premissas:

\begin{itemize}
\item Deve ser declarativo, descrevendo estrutura e outros atributos;
\item Deve ser ao invés de definir o processamento a ser efetuado no
\item Deve ser documento: rigoroso de modo que as mesmas técnicas de
\item Deve ser mineração de dados em objetos e bancos de dados possam ser
\end{itemize}

Os criadores de navegadores constantemente introduziam
elementos de apresentação com o *blink*, *<i>* itálico, *<b>* bold,
que eventualmente acabavam por serem inclusos na especificação. Foi
somente nas últimas versões que elementos de apresentação voltaram
a ser proibidos reforçando as propostas chave HTML como uma linguagem
de conteúdo semântico, abrindo espaço para a expansão de outras
tecnologias como o CSS para responder as demandas de apresentação.

A atual versão do HTML é o HTML5, desenvolvido como um trabalho em
conjunto entre a WHATWG e a W3C, seu rascunho foi proposto em 2008 e
ratificado em 2014. O HTML5 introduziu elementos interativos que
viabilizaram a construção de jogos para a plataforma como: Canvas,
áudio, vídeo.

Cada elemento HTML informa algo ao navegador sobre a informação que
reside entre o abrir e fechar da tag. <!-- ducket pág 20 -->

Um elemento é o abrir fechar de uma tag e todo o conteúdo que dentro
dele reside.<!-- ducket pág 24 -->

O HTML5 é muitas vezes interpretado como um conceito guarda chuva para
designar as tecnologias da web HTML, CSS3 e JavaScript.

%}}}

\section{CSS}
%{{{

É uma linguagem de folhas de estilo, criada por Håkon Wium Lie em
1994, com intuito de definir a apresentação de páginas HTML.

>  Possibilitam ligação tardia *late biding*. Essa característica
é atrativa para os publicadores por dois motivos. Primeiramente pois
permite o mesmo estilo em várias publicações, segundo pois os
publicadores podem focar-se no conteúdo. Lie

O CSS é dividido em módulos, contendo aproximadamente 50 deles, cada
qual evoluindo separadamente.

Sua última versão, o CSS3, introduziu várias funcionalidades
relevantes para jogos, como *media-queries*:possibilitam regras para
tamanhos de tela, transformações 3D e animações.

Os navegadores interpretam CSS através da tag ``<style>``.

<!-- Cascading Style Sheets, PhD thesis, by Håkon Wium Lie – provides
an authoritative historical reference of CSS pág. 23 -->

<!-- Muitos navegadores também suportam aceleração de GPU (Unidade de
processamento gráfico) para elementos que tenham transformações 3D.
-->

Flow de documento, ordem e posição em que os elementos tem que
aparecer na página. Modelo de caixa o que encapsula o conteúdo em um
elementos. -->

<!-- falar do suporte a variáveis do CSS -->

\subsection{Transições}

É uma forma de adicionar animações de um estado para outro, em uma página web.

%}}}

\section{JAVASCRIPT}
%{{{

EMACScript, melhor conhecido como JavaScript, criada por Brendan Eich
em 1992, é a linguagem da Web. Devido a tremenda popularidade entre
comunidade de desenvolvedores a linguagem foi abraçada pela W3C e
atualmente é um dos componentes da *Open Web Platform*.

As definições da linguagem são descritas na especificação ECMA-262.
Esta possibilitou o desenvolvimento de outras implementações além da
original - *SpiderMonkey* - como o Rhino, V8 e TraceMonkey; bem como
outras linguagens similares como JScript da Microsoft e o ActionScript
da Adobe.

JavaScript é uma linguagem de script. Segundo a Ecma Internacional
2012:

\begin{quote}
Uma linguagem de script é uma linguagem de programação que é
usada para manipular e automatizar os recursos presentes em um dado
sistema. Nesses sistemas funcionalidades já estão disponíveis
através de uma interface de usuário, uma linguagem de script é
um mecanismo para expor essas funcionalidades para um programa
protocolado.
\end{quote}

A intenção original era utilizar o JavaScript para dar suporte aos já
bem estabelecidos recursos do HTML, como para validação, alteração
de estado de elementos, etc. Em outras palavras, a utilização do
JavaScript era opcional e as páginas da web deveriam continuar
operantes sem a presença da linguagem.

Com a construção de projetos Web cada vez mais complexos,
as responsabilidades delegadas ao JavaScript aumentaram a ponto que a
grande maioria dos sistemas web não funcionarem sem ele. Não obstante,
JavaScript não evoluiu ao passo da demanda e muitas vezes carece de
definições expressivas, completude teórica, e outras características
de linguagens de programação mais bem estabelecidas, como o C++ ou
Java (Barnett, 2013). A nova versão do JavaScript, o JavaScript 6, é
um esforço nessa direção. JavaScript 6 ou *EMACScript Harmonia*,
contempla vários conceitos de orientação a objetos como classes,
interfaces, herança, tipos, etc.

Estes esforços de padronização muitas vezes não são rápidos
o suficiente para produtores de software web, demora-se muito até
obter-se um consenso sobre quais as funcionalidades desejadas em
determinada versão e seus detalhes de implementação. Outrossim, uma
vez definidas as especificações, é necessário que os distribuidores
do JavaScript implementem o especificado.

Alternativamente, existe uma vasta gama de conversores de código -
*transpilers* - para JavaScript; possibilitando programar em linguagens
formais e posteriormente gerar código JavaScript. Não obstante,
essa alternativa tem seus pontos fracos, necessita-se de mais tempo
de depuração , visto que o JavaScript gerado não é conhecido
pelo desenvolvedor, e provavelmente o código gerado não será tão
otimizado, nem utilizará os recursos mais recentes do JavaScript.

Mesmo com suas fraquezas amplamente conhecidas, JavaScript está
presente em praticamente todo navegador atual. Sendo uma espécie de
denominador comum entre as plataformas. Essa onipresença torna-o
integrante vital no processo de desenvolvimento de jogos multiplataforma
em HTML5. Vários títulos renomeados já foram produzidos que fazem
extensivo uso de JavaScript, são exemplos: Candy Crush Saga, Angry
Birds, Dune II, etc.

Jogos Web são escritos na arquitetura cliente servidor, JavaScript pode
rodar em ambos estes contextos, para tanto, sua especificação não
define recursos de plataforma. Distribuidores do JavaScript complementam
a o JavaScript com recursos específicos para suas plataformas alvo.
Por exemplo, para servidores, define-se objetos de terminal, acesso a
arquivos e dispositivos, etc. No contexto de cliente, são definidos
objetos como janelas, frames, DOM, etc.

Para o navegador o código JavaScript geralmente é disposto no elemento
``script`` dentro de arquivos HTML. Quando os navegadores encontram esse
elemento eles fazem a requisição para o servidor e injetam o código
retornado no documento, e a não ser que especificado de outra forma,
iniciam sua execução.

%}}}

\section{DOCUMENT OBJECT MODEL (DOM)}
%{{{
É uma convenção que especifica como elementos HTML interegem.

A primeira versão do DOM, (DOM Legacy) foi parcialmente especificada no HTML4.

O DOM é também uma forma de manter estado em páginas HTML.

Os navegadores contam com as engines de layout para parsear HTML em DOM.

A atual versão do DOM é a versão 3    


%}}}

\subsection{CANVAS}
%{{{
1. Troca de tamanhos via Canvas vs DOM
2. Aceleração de GPU
3. API de Áudio (referência 2)

A nova tag <canvas> define um layer gráfico em documentos HTML que pode
ser desenhado através de JavaScript.

Permite desenhar diagramas, gráficos e animações [7]. É baseado em
bitmap.

Muitas vezes lento. Algumas soluções tentam arrumar isso através da
utilização de GPU Apache Cordova utiliza o FastCanvas.

CocoonJS é uma aplicativo híbrido que preenche a fraca implementação
de OPENGL nos dispositivos móveis possibilitando se desenvolver em
WEBGL.
%}}}
\section{WEBGL}
%{{{
Baseado no OpenGL.

WebGL não foi utilizada no trabalho apesar de ser de grande
relevância no processo de jogos pois ainda não está completamente
especificada e a dificuldade e escopo do projeto aumentariam muito se
tivessem de incluir um jogos 3D. Versão da especificação atual?
%}}}
\section{VIDEO}
%{{{
Antes do HTML5 era impossível adicionar vídeos nas páginas sem a utilização de algum plugin.

Os navegadores não concordam em qual formato de vídeo suportar.
Uma tag vídeo pode apontar para vários arquivos em diversos formatos, e os navegadores que suportarem determinado irão escolhê-lo.

Codec - é o algorítmo usado para encodificar o video em um conjunto de bits \cite{diveIntohtml}.

Um formato de vídeo é a combinação de várias tecnologias.
%}}}
\section{AUDIO}
%{{{
Áudio é um componente vital para oferecer grande satisfação aos
usuários de jogos. Provê feedback e imerge o usuário. Efeitos de som
e música podem servir como mecanismo. Jogadores tem baixa tolerância a
volume, deve ser utilizado com cautela.
\subsection{TAG AUDIO}

A tag <audio> define um som dentro de um documento HTML. Quando o
elemento é renderizado pelos navegadores, ele carrega o conteúdo
que pode ser reproduzido pelo player de audio do navegador. Existem
muitas discrepâncias entre os formados aceitáveis pelos navegadores.
É um tanto limitada quanto comparada ao áudio de múltiplos canais
disponibilizados por SDKs nativas.

\subsection{API DE AUDIO}

É uma interface de audio experimental para JavaScript. Provê maior
flexibilidade na manipulação de audio. Essa tecnologia é muito mais
nova do que a tag áudio.

FORMATOS DE ÁUDIO

%}}}

\section{CACHE}

Aplicações offline.

Algumas tecnologias desta classe são:

\section{OFFLINE E ARMAZENAMENTO}

> Uma das grades limitações do HTML era a ausência de capacidade
de armazenamento de dados. Armazenamento no lado do cliente é um
requerimento básico para qualquer aplicação moderna. Essa área
era ode as aplicações nativas detinham grande vantagem sobre as
aplicações web. O HTML5 solucionou este problema introduzindo várias
formas de armazenamento de dados. (HASAN et all, 2012)

\subsection{LOCAL STORAGE}

Também conhecido como WebStorage na especificação do HTML5. Provê
uma forma de armazenar os dados como chave valor dentro do navegador. Os
dados são persistido mesmo que o navegador seja fechado.

\section{ENTRADA DE COMANDOS}

Na construção da grande maioria dos jogos é muitas vezes
imprescindível alta flexibilidade na gestão de entrada de dados.
Este fator muito se amplia na criação de jogos multiplataforma,
seja através de teclado, tela sensível ou sensor de movimentos, o
importante é oferecer a melhor experiência possível por plataforma.
O HTML5 trata todos estes casos abstratamente na forma de eventos, os
quais podem ser escutados através de listeners. Os eventos básicos
são: keydown (tecla baixa), keyup (tecla solta) e keypress (tecla
pressionada).

Para detectar suporte aos mais variados recursos do HTML5 no navegador
do cliente existem duas possibilidades. Pode-se implementar testes para
cada funcionalidade utilizada abordando os detalhes de implementação
de cada uma ou então fazer uso de alguma biblioteca especializada
neste processo, o Modernizr é uma opção open-source deste tipo de
biblioteca, este gera uma lista de booleanos sobre grande variedade dos
recursos HTML5, dentre estes, geolocalização, canvas, áudio, vídeo e
local storage.

\subsection{WEB SQL}

Um banco de dados SQLite embebido no navegador. Permite
tabelas relacionais. O tamanho padrão do banco de dados é 5 megabytes
e pode ser estendido pelo usuário.

\subsection{DISPONIBILIZAÇÃO DA APLICAÇÃO}

Links com manifestos
\section{INSTALAÇÃO}

Este método é benéfico pois possibilita ao usuário a mesma
experiência ao adquirir uma aplicação normal. Este tipo de
aplicação é comummente referido como "híbrido".

\section{O JOGO}

Finalizada a revisão bibliográfica será descrito os detalhes da implementação do jogo.

Devido ao fato deste trabalho explorar as limitações dos jogos em
HTML5, optei por evitar a utilização de plugins e ferramentas de
terceiros que pudessem ocultar alguma limitação.

Escolhi a simplicidade para não precisar ficar muito tempo aprendendo
as coisas em detrimento do refinamento da pesquisa.

\section{MECÂNICA}

O jogo consiste em simplesmente em uma tela que apresenta equações e
um possível resultado. Cabe ao jogador decidir se o resultado está
certo ou errado. O tempo é um fator levado em consideração, quão
mais rápido o jogador acertar se a afirmação está correta ou não,
mais pontos ele receberá.

Argumentos à favor da escolha do game: Tem profundidade, permite a
adição de novos recursos no futuro;
É facilmente traduzível em tamanhos de telas diferentes e tipos de
entrada de dados diferentes;

\section{IMPLEMENTAÇÃO}

Não tenho grande experiência com o desenvolvimento de jogos nem com
o desenvolvimento em HTML5. Também para não interferir na pesquisa
busquei não me distanciar do que é considerado padrão em ferramentas
e métodos.
Comecei escrevendo o aplicativo para o Navegador do desktop pois era o
que estava mais acessível no momento. Mais tarde descobri que de fato
é assim que de desenvolve.

\section{ANDROID}
%{{{

É um sistema operacional *open-source* desenvolvido pela Google.
Utiliza o kernel Linux .
Softwares para Android são geralmente escritos em Java e executados
através da máquina virtual Dalvik.

É similar a máquina virtual Java, mas roda um  .
formato de arquivos diferenciado (dex), otimizados para consumir pouca .
memória, que são agrupados em um único Android Package (apk) Android.
permite a renderização de documentos HTML através de sua própria   .
API WEBVIEW. Ou através do navegador disponibilizado por padrão, ou  .
outros de terceiros como o Google Chrome, Firefox, Opera, etc          .

No quesito jogos para dispositivos móveis é preferível disponibilizar
os jogos através da interface nativa pois dá a sensação de
continuidade para com os demais aplicativos instalados no dispositivo.

%}}}


\section{NAVEGADORES}
%{{{

Aplicações do lado do cliente geralmente se comunicam com um
servidor através de documentos em HTTP. Quado o navegador recebe um
destes pacotes em HTML ele começa o processo de renderização. A
renderização pode requisitar outros arquivos a fim de completar a
experiência desenvolvida para o endereço em questão.

Nos navegadores os usuários necessitam localizar a página que desejam,
sabendo o endereço, ou pesquisando em buscadores. Isso é um processo
árduo para a plataformas móveis pois necessitam maior interação
do usuários e não são “naturais” se comparado ao modo normal
de consumir aplicativos nestas mesmas plataformas – simplesmente
adquirindo o aplicativo na loja e abrindo-o no sistema operacional.
Alguns contornos para este problema serão descritos nas tecnologias
offline.

Para transformar as instruções retornadas pelo servidor em algo útil
para o usuário final os navegadores geralmente fazem uso de bibliotecas
externas capazes de interpretar HTML5 e gerar o conteúdo iterativo.

\subsection{BIBLIOTECAS WEB}

O Google Chrome utilizá o Webkit para renderizar seu conteúdo HTML5. O
webkit foi criado pela Apple baseando-se no motor de renderização do
Konkeror do projeto KDE. Safari e Opera também fazem uso do Webkit. V8
para JavaScript.

O motor de renderização do HTML5 do Firefox é o XXX. O motor de
JavaScript é o.


\section{ TRABALHOS SIMILARES}

(Referência 2) Faz uma revisão de aspectos do HTML5 através da
construção de um jogo. O autor foca muito nos aspectos de criação
de jogos e feedback do desenvolvimento. Troca de tecnologias e não
especificamente nas limitações conforme o meu trabalho. Em outras
palavras seu escopo é mais genérico e não tão preciso quanto este


%}}}
%}}}

\chapter{METODOLOGIA}
%{{{

O primeiro passo consiste em definir as plataformas alvo do trabalho.
Estas devem ser relevantes mercadologicamente ao desenvolvimento de jogos em HTML5.

Segue-se com a construção de uma lista com os recursos relevantes
aos jogos que, sofrem ou são comummente ligados à
limitações multiplataforma. Segue-se uma pesquisa para aprofundar
teoricamente cada um dos recursos, possivelmente elegendo novos.

Com um baseamento teórico substancial, o próximo passo é a criação
do protótipo de um jogo multiplataforma que utilize recursos
potencialmente limitados. Para ser considerado pronto, o protótipo deve
ser testado, e estar funcional, com adaptações ou não, em cada uma
das plataformas alvo definidas.

Com o protótipo concebido, o passo que segue é a enumeração, e
descrição das limitações detectadas no processo de desenvolvimento e
testes do jogo. Este detalhamento deve responder as seguintes perguntas:

\begin{itemize}
\item Quais as limitações foram encontradas no jogo?
\item Em quais plataformas?
\item Sob quais circunstâncias?
\item As limitações puderam ser contornadas?
\item Algum efeito colateral das limitações no jogo?
\item Qual a categoria do problema: usabilidade, funcionalidade,
manutibilidade, portabilidade ou performance? (segundo ISO)
\end{itemize}

Como recomendação geral, busca-se abster-se da utilização de plugins pois estes muitas
vezes escondem limitações do HTML em si.
%}}}

\chapter{RESULTADOS}
%{{{

Durante a construção do jogo utilizei a estratégia de declarar todos
os objetos relativos ao window e limitar o escopo. Isso se demonstrou
uma boa forma de separar as responsabilidades.

Abaixo constam as limitações encontradas durante a pesquisa e concepção do jogo

Muitos dos problemas dos jogos multiplataforma não são específicos
dos jogos, mas aplicam-se a todos tipos de software.  \parencite{currentStateCrossPlatform}


\section{LIMITAÇÕES}

\subsection{}

Apesar da grande maioria dos recursos dos dispositivos estar presente
em HTML5 ainda existem muitas funcionalidades faltando para este tipo
de aplicação. Por exemplo, não podemos mudar a imagem de fundo do
dispositivo, ou adicionar toques etc. Similarmente, existem muitas
APIs de nuvem como os serviços de impressão do ICloud ou Google
cloud que estão disponíveis para aplicações nativas mas não para
HTML5. Outros serviços utilitários como o C2DM do Google que está
disponível para desenvolvedores Android para utilizar serviços de push
também não estão disponíveis para o HTML5. (HASAN, 2012)

1.  VERSÕES
A grande maioria dos dispositivos atualmente no mercado utilizam
obsoletas de seus softwares. Isso dificulta o desenvolvimento. Se a
tecnologia de tradução para o navegador utilizar o a classe Webview do
Android - como o Apache Cordova faz - as versões mais antigas podem ser
penalizadas com problemas de performance ou falta de recursos.

2. OFFLINE

Refresh duplo para ver assets cacheados. Ver:
http://buildnewgames.com/game-asset-management/

3. AUDIO
Api de som quebra quando executado diversas vezes.
Os navegadores variam na disponibilização de formatos aceitáveis
Somente um áudio pode ser tocado no Navegador do Android
Não é possível trocar o volume no IOS.
Alguns navegadores favorecem formatos ogg (vorbis) e outros, como o
Safari, favorecem o MP3.

> O maior problema com as API's de áudio e de vídeo do HTML5 é
a disputa entre os codecs dos navegadores. Por exemplo, Mozilla e
Opera suportam Theora, já o Safari suporta H.264 que também é
suportado pelo IE9. Ambos, Iphone e Android suportam H.264 em seus
navegadores. A W3C recomenda OggVorbis e OggTheora para áudio e vídeo
respectivamente. (HASAN et al, 2012)

3. VIDEO

Codecs

4. ASSETS

Trafegar muitos assets deixa o sistema lento.

 Contorno
Utilizando páginas de carregamento e/ou cache;

5. UI

É muito custoso desenvolver uma interfaces que pareçam nativas
para cada dispositivo sem a utilização de plugins e ferramentas
especializadas. Em termos gerais, trabalhar com proporções é
positivo. Não obstante
há casos, como o dos botões de certo e errado que a proporções ficam
exageradas, nesses casos a utilizada de max-width é uma solução
conveniente.

6. PERFORMANCE

De acordo com uma pesquisa, para um usuário uma tarefa é instantânea
se ele leva até 0.1 segundos para ser executada. Se a tarefa toma
aproximadamente um segundo então a demora será notada mas o
usuário não se incomodará com ela. Entretanto, se a tarefa leva
aproximadamente 10 segundos para terminar o usuário então começa a
ficar aborrecido e esse é o limite que algum feedback deve ser dado
para um usuário.

ACELERAÇÃO DE GPU

7. Acelerômetro

8. IMPLEMENTAÇÃO INCONSISTENTE DE APIs

9.  TAMANHO DE TELA
Em alguns casos o tamanho das telas pode ser um fator limitante – como
no caso de jogos de estratégia. Jogadores com telas menores podem sair
em desvantagem. 9. CÂMERA

10 . JavaScript
Ciclo de vida demorado pois necessita que todos os consumidores da
especificação entrem em consenso e implementem a.

Desktop/Firefox
Desktop/Google Chrome
Smatphone/Android
%}}}

\chapter{CONCLUSÕES}
%{{{

Não pude testar todos os métodos e ferramentas e versões à
disposição, um trabalho completo demandaria esforços conjuntos de
muitos indivíduos ou um período de tempo bem mais extenso. 

Se uma empresa deseja produzir jogos nativos elas precisarão de vários
desenvolvedores. Eu sozinho fui capaz de produzir um jogo em tempo
razoável trabalhando com a plataforma web.

Por não utilizar frameworks e bibliotecas estou me distanciando
dos casos da vida real. 

Só poderemos considerar o HTML como uma especificação pronta quando for possível fazer tudo o que se faz nativamente com os dispositivos através de uma API web padronizada.

Baixa fricção quer dizer que você pode ir de um site para outro sem
ter que instalar, o serviço está em demanda, você não é obrigado a
tê-lo em sua tela inicial.


O futuro dos jogos em HTML5 parece brilhante.


Neste trabalho revisamos tecnologias relevantes no desenvolvimento de jogos.

\subsection{TRABALHOS FUTUROS}

Trabalhos que explorem os benefícios mercadológicos do HTML5 em comparação com alternativas nativas.
EMACSCRIPT 7

%}}}

\printbibliography

\appendix

\chapter{ANEXOS}
%{{{

\subsection{ CONVERSORES PARA HTML5}
Além da possibilidade de escrever em HTML, pode-se optar pela
alternativa de utilizar-se um conversor de linguagens.

\subsection{ METODOLOGIA DE DESENVOLVIMENTO DE SOFTWARE PARA A CONSTRUÇÃO DE GAMES}

Como o jogo é um software complexo demanda-se a utilização de
metodologias de engenharia de software, dentre os processos de software
mais conhecidos academicamente destacamos:

- OpenUP: este é bem detalhado e de característica iterativa e
incremental. Gerando assim, um levantamento mais apurado dos riscos,
requisitos e outros detalhes do sistema e a criação incremental do
sistema, com requisitos maleáveis;

- Cascata: processo antigo, caracteriza-se por ser pouco maleável aos
requisitos mapeados posteriormente ao processo de análise;

- Processo ágil - SCRUM: sua utilização é flexível e sendo
um método ágil especifica pouca documentação, ou como dizem,
somente a documentação necessária, este processo é bem conhecido e
aceito na comunidade de desenvolvimento de software. Suas principais
características são: divisão do processo de desenvolvimento através
uma série de iterações chamadas sprints. Cada sprint consiste
tipicamente em duas a quatro semanas. É bem aplicado a projetos que
mudam constantemente e que demandam rápidas adaptações;

- Processo ágil – XP: tem muitas características similares ao SCRUM
por este também ser um processo ágil. Dentre suas especifidades
destaca-se: versões frequentes, pequenos ciclos de desenvolvimento que
buscam aumentar a produtividade, introduzem checkpoints onde os clientes
podem agregar novas funcionalidades;

\subsection{ AMBIENTES PARA DESENVOLVIMENTO HTML5}

Na pesquisa efetuada sobre estes frameworks full-stack foram
identificadas as seguintes tecnologias:

    - segundo (PRADO, 2012) o GWT é um framework essencialmente para
o lado do cliente (client side) e dá suporte à comunicação com
o servidor através de RPCs Remote Procedure Calls (ou procedimento
de chamadas remotas). Ele não é um framework para aplicações
clássicas da web, pois deixa a implementação da aplicação web
parecida com implementações em desktop. Este é utilizado em muitos
produtos de grande porte como o Google Adwords e Google Wallet. Outra
característica interessante é que a plataforma opera sobre a licença
Apache versão 2;

    - construct 2 - é um editor na nuvem focado para usuários sem
    - conhecimento prévio em programação orientado a comportamento;
    - PlayCanvas - é uma plataformas para a construção de jogos 3D
na nuvem, desenvolvida com foco em performance. Permite a hospedagem,
controle de versão e publicação dos aplicativos nela criados,
possibilita também a importação de modelos 3D de softwares populares
como: Maya, 3ds Max e Blender;

    - o ambiente HTML5 da Intel, este fornece uma solução na nuvem,
completa para o desenvolvimento em plataforma cruzada, com serviços de
empacotamento, serviços para a criação e testes de aplicativos com
montagem de interfaces puxa e arrasta (Intel XDK) e bibliotecas para a
construção de jogos utilizando aceleração de hardware, o que garante
até duas vezes mais performance que aplicativos mobile baseados em
Web tradicionais. Esta solução é gratuita, open-source e funciona
através de um plugin para o Google Chrome, ou seja, o desenvolvimento
também é multiplataforma e devido ao fato de os binários ficarem
hospedados na nuvem, possibilitou a Intel criar compiladores para cada
uma das plataformas disponibilizadas pelo PhoneGap, que é o framework
polyfill utilizado na solução.

\subsection{ HTTP}

\subsection{Frameworks de jogos}

Com o intuito de simplificar o processo para os desenvolvedores,
auxiliando-os a focarem-se apenas nas soluções que estão
desenvolvendo, foram criados os frameworks para desenvolvimento de
jogos. Não obstante, o intuito deste trabalho é desenvolver um jogo
sem auxilio de frameworks pois estes muitas vezes escondem possíveis
limitações, desenvolvendo soluções próprios.

- enchant.js: dentre suas funcionalidades constam: orientação à     ;
- objetos, orientado à eventos, contém um motor de animação,       ;
- suporta WebGL e Canvas, etc three.js: considerada leve, renderiza    ;
- WebGL e Canvas, arquitetura procedural                               ;
- quintus: bom para plataformas 2D
- limeJs: bom para 2d

\subsection{ INTERFACE E ESCOLHAS DE DESIGN}

\subsection{PROGRESSÃO CONTÍNUA}

\subsection{JAVASCRIPT NÃO OBSTRUTIVO}

\subsection{Arquitetura Cliente Servidor}

\subsection{NODEJS}

Permite rodar JavaScript fora do navegador. Utiliza um modelo dirigido
à eventos sem bloqueio, tornando-o rápido e eficiente.

\subsection{ALTERNATIVAS AO JAVASCRIPT}

Abaixo seguem algumas tecnologias que servem de alternativa ao
JavaScript.

\subsection{TYPESCRIPT}

Conhecido como uma versão estendida do JavaScript que compila para
JavaScript normal.

\subsection{DART}

Google. DartVM é uma máquina virtual que está embebido no Google
Chrome. Significante melhorias em performance quando comparado
ao JavaScript. Existe o dart2js que compila código em Dart para
JavaScript.

\section{SISTEMAS DE BUILDING}

Aquivos JavaScript são requisitados do servidor assincronamente. Isso
pode levar a tempos de requisição pouco desejáveis. Uma saída seria
escrever o código em apenas um arquivo mais isso leva a gerência de
código bagunçada. A saída mais comum entre desenvolvedores é utilizá
ruma ferramenta que junta todos os arquivos e disponibiliza apenas um
para o usuário.

Utiliza o conceito de streams para aplicar todas as modificações sobre
um arquivo de uma vez só.

\subsection{SOURCE MAPS}

Para encontrar os arquivos minificados a fim de ajudar o desenvolvedor a
debugar a aplicação.

%}}}

\subsection{ASM.JS}
%{{{
Asm.js é um subconjunto da sintaxe do JavaScript a qual permite grandes
aumentos de performance quando em comparação com JavaScript normal.
No contexto dos jogos performance é usualmente um recurso estimável,
asm.js consegue-o supra utilizando recursos que permitam otimizações
antes do tempo *ahead of time optimizations*. Entretanto, não é
trivial escrever código em asm.js e geralmente a geração de código
asm.js é feita através da transpilação de outras linhagens como C.

> Muita da performance adicional em relação ao JavaScript é devido
a consistência de tipo e a não existência de um coletor de lixo
(memória é gerenciada manualmente através de um grande vetor). Esse
modelo simples desprovido de comportamento dinâmico, sem alocação
e desalocação de memória, apenas um bem definido conjunto de
operações de inteiros e flutuantes possibilita grade performance e
abre espaço para otimizações.
%}}}

\section{CROSSWALK}
%{{{
Crosswalk empacota os fontes juntamente com uma versão do Chromium, a
versão Open-source do Google Chrome. Isso faz com que o software se
comporte da mesma forma para todas as versões de dispositivos Android.

\section{PHONEGAP}
\subsection{PHONEGAP CLOUD}

Este serviço possibilita que se faça upload de um arquivo compactado
contendo os fontes – ou apontando para um repositório no GitHub –
que no tempo desta pesquisa não estava funcionando; e se gere o APK
para o Android nativamente.
%}}}
\end{document}
