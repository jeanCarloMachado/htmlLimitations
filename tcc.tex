%{{{
\documentclass[
12pt,
a4paper,
portuges,
%titlepage,
% draft
]{report}

\usepackage[
top=3cm,
left=3cm,
bottom=2cm,
right=2cm
% inner=1.5in, % The inner margin (beside binding)
% outer=1in, % The outer margin (opposite binding)
% headheight=20pt, % Header height
% headsep=.25in, % Header separation
% includehead,
% includefoot
]{geometry}
\usepackage[portuges]{babel}
%times new roman package
\usepackage[utf8]{inputenc}
\usepackage[T1]{fontenc}
\usepackage{mathptmx}
\usepackage{titling}
\usepackage{fancyhdr}
\usepackage[hyphens]{url}
\usepackage{anyfontsize}
%indent first paragraphs of sections
\usepackage{indentfirst}
\usepackage{csquotes}
\usepackage{float}

\input glyphtounicode
\pdfgentounicode=1

\usepackage{xcolor,mdframed}
\usepackage{longtable}
\usepackage{listings}

%limitations schema
\usepackage{cleveref}
\newcounter{limitation}
\newcommand{\limitationname}{LMT}
\newcommand{\limitation}[1]{%
  \refstepcounter{limitation}%
  \limitationname\ \thelimitation%
  \label{limitation:#1}%
}
\crefname{limitation}{limitation}{limitations}
\Crefname{limitation}{LMT}{LMT}


%table settings
\setlength{\arrayrulewidth}{1mm}
\setlength{\tabcolsep}{18pt}
\renewcommand{\arraystretch}{1.5}
\setcounter{tocdepth}{4}
\setcounter{secnumdepth}{4}

%set first paragraph indentation length
\setlength{\parindent}{1.27cm}


\pagestyle{myheadings}
%ovewrite plain to my headings style
\makeatletter
  \let\ps@plain\ps@myheadings
\makeatother

\usepackage{titlesec}

\usepackage[titletoc,title]{appendix}

\usepackage[final]{graphicx}
\graphicspath{ {asset/} }
%sort by alphabetic label, name, year, volume, title 
\usepackage[style=abnt, citestyle=authoryear,sorting=anyvt,backend=bibtex, natbib]{biblatex}
\addbibresource{tcc.bib}
\usepackage[font=footnotesize,format=plain,labelfont=bf,up,textfont=normal,up,justification=justified,singlelinecheck=false]{caption}
\addto\captionsportuges{
	\renewcommand{\contentsname}{\hspace*{\fill}\normalfont\fontsize{14}{17}\bfseries SUMÁRIO \hspace*{\fill}}
    \renewcommand{\listfigurename}{\hspace*{\fill}\normalfont\fontsize{14}{17}\bfseries LISTA DE FIGURAS \hspace*{\fill}}
    \renewcommand{\listtablename}{\hspace*{\fill}\normalfont\fontsize{14}{17}\bfseries LISTA DE QUADROS\hspace*{\fill}}
    \renewcommand{\chaptername}{}
    \renewcommand{\figurename}{Figura}
    \renewcommand{\tablename}{Quadro}
}


\newtheorem{lm}{LIM}

\usepackage{mfirstuc}
\renewcommand*{\mkbibnamelast}[1]{\makefirstuc{#1}}%
\renewcommand*{\mkbibnameprefix}[1]{\makefirstuc{#1}}%

\AtBeginBibliography{% 
  \renewcommand*{\mkbibnamelast}[1]{\MakeUppercase{#1}}%
    \renewcommand*{\mkbibnameprefix}[1]{\MakeUppercase{#1}}%
}

\usepackage{setspace}

\titleformat{\chapter}{\normalfont\fontsize{14}{17}\bfseries}{\thechapter}{1em}{}
\titleformat{\section}{\normalfont\fontsize{12}{15}\bfseries}{\thesection}{1em}{}
\titleformat{\subsection}{\normalfont\fontsize{12}{15}\bfseries}{\thesubsection}{1em}{}
\titleformat{\subsubsection}{\normalfont\fontsize{12}{15}\bfseries}{\thesubsubsection}{1em}{}


\usepackage{titletoc}
\usepackage{tocloft}
\renewcommand{\cftchapleader}{\bfseries\cftdotfill{\cftdotsep}} % for chapters
\renewcommand{\cftchapfont}{\normalfont\bfseries}                          %Bold chapterentries
\renewcommand{\cftsecfont}{\normalfont}                          %Bold chapterentries
\makeatletter
\patchcmd{\l@section}{#1}{\MakeUppercase{#1}}{}{}
\makeatother
\renewcommand{\cftsubsecfont}{\normalfont\bfseries}                          %Bold chapterentries
\renewcommand{\cftsubsecleader}{\bfseries\cftdotfill{\cftdotsep}} % for chapters
\setlength{\cftchapindent}{0cm}
\setlength{\cftsecindent}{0cm}
\setlength{\cftsubsecindent}{0cm}
\setlength{\cftsubsubsecindent}{0cm}

\newlength{\mylen}

\renewcommand{\cfttabpresnum}{\tablename\enspace}
\renewcommand{\cfttabaftersnum}{:}
\settowidth{\mylen}{\cfttabpresnum\cfttabaftersnum}
\addtolength{\cfttabnumwidth}{\mylen}

\renewcommand{\cftfigpresnum}{\figurename\enspace}
\renewcommand{\cftfigaftersnum}{:}
\settowidth{\mylen}{\cftfigpresnum\cftfigaftersnum}
\addtolength{\cftfignumwidth}{\mylen}

\renewcommand{\cftfigpresnum}{\figurename\enspace}
\renewcommand{\cftfigaftersnum}{:}
\settowidth{\mylen}{\cftfigpresnum\cftfigaftersnum}
\addtolength{\cftfignumwidth}{\mylen}


\usepackage{chngcntr}
\counterwithout{figure}{chapter}
\counterwithout{table}{chapter}

\renewcommand\cfttoctitlefont{\normalfont\fontsize{14}{17}}
\renewcommand\cftloftitlefont{\normalfont\fontsize{14}{17}}
%\renewcommand\cftfigfont{\normalfont\fontsize{10}{12}}
%\renewcommand\cftfigpagefont{\normalfont\fontsize{10}{12}}

% \DefineBibliographyStrings{portuguese}{%
% %  backrefpage = {<newtext>},
% %  backrefpages= {<newtext>},
%   urlseen = {Disponível em},
%   url = {Disponível em}
% }
% \DeclareFieldFormat{url}{\bibstring{url}\space\url{#1}}
% \DeclareFieldFormat{urlseen}{\bibstring{urlseen}\space\url{#1}}

%quotes configuration {{{
\usepackage{relsize,etoolbox}% http://ctan.org/pkg/{relsize,etoolbox}
\AtBeginEnvironment{quote}{\footnotesize}% Step font down one size relative to current font.
\usepackage{etoolbox}
\renewenvironment{quote}{\list{}{\leftmargin=4cm\rightmargin=0cm}\item[]}{\endlist}
\expandafter\def\expandafter\quote\expandafter{\quote\singlespacing}
%}}}
%footnotes configuration {{{
%seems to be ok
%}}}

%{{{ VARIABLES
\title{\uppercase{Limitações do HTML \\ no desenvolvimento de jogos multiplataforma}}
\author{\uppercase{Jean Carlo Machado}}
\newcommand{\supervisor}{Mr. Rafael Jaques}
\newcommand{\university}{\uppercase{Instituto Federal de Educação, Ciência \\ e Tecnologia do Rio Grande do Sul \\ Campus Bento Gonçalves}}
\newcommand{\locale}{Bento Gonçalves, dezembro de 2015.}
\newcommand{\website}{http://jeancarlomachado.com.br}
\newcommand{\source}[1]{\caption*{\textbf{Fonte:} \footnotesize{{#1}}\hfill} }
\usepackage[textfont={footnotesize}]{caption}

%}}}

\makeatletter
\newcommand{\setappendix}{Apêndice~\thechapter -~}
\newcommand{\setchapter}{\thechapter~}
\titleformat{\chapter}{\normalfont\fontsize{14}{17}\bfseries}{
%\titleformat{\chapter}{\bfseries\LARGE}}}
%{{{Title Page

\begin{titlepage}
    \begin{center}
        {\fontsize{14}{18}\selectfont \university}
        \vfill
        {\fontsize{16}{19}\selectfont \thetitle }
        \vfill
        {\fontsize{12}{15}\selectfont \theauthor}
        \vfill
        {\locale}
    \end{center}
\end{titlepage}

%}}}
%{{{Folha de rosto

\begin{titlepage}
    \begin{center}
        {\fontsize{14}{18}\selectfont \theauthor}
        \vfill
        {\fontsize{16}{19}\selectfont \thetitle }
        \vfill
        \hfill
        \parbox[s]{8cm}{
        \singlespacing
            Monografia apresentada junto ao Curso
        de Tecnologia em Análise e Desenvolvimento de Sistemas no
    Instituto Federal de Educação, Ciência e Tecnologia do Rio
Grande do Sul - Campus Bento Gonçalves, como requisito parcial à
obtenção do título de Tecnólogo em Análise e Desenvolvimento de Sistemas.
        \\
        Orientador: Prof. Esp. Rafael Jaques
        }
        \vfill
        {\bfseries \locale}
    \end{center}
\end{titlepage}

%}}}
%{{{ Resumo
\onehalfspacing
\renewcommand{\abstractname}{\Large\bfseries RESUMO}
\begin{abstract}
{
Jogos estão cada vez mais diversos, sociais e presentes no nosso dia
a dia e a miríade de dispositivos que podem comportá-los oferece
desafios e oportunidades. O HTML é uma ferramenta que possibilita
a construção de jogos para múltiplas plataformas; entretanto, as
tecnologias do HTML mudam constantemente e seu suporte é variado. Neste
contexto, este trabalho busca estudar as limitações do HTML quando
aplicado ao desenvolvimento de jogos multiplataforma. Para tal, foi
elaborada uma revisão bibliográfica do assunto através de artigos
científicos, livros e teses. E desenvolveu-se um protótipo de jogo
de matemática, onde o usuário pode escolher a veracidade de data
equação. Com a experiência adquirida e os dados coletados na revisão
procedeu-se com a criação de uma lista de limitações presentes no
atual estado do HTML.
}
\\
\\
{\bfseries Palavras-chave:} Jogos, HTML, Limitações,
Multiplataforma
\end{abstract}

\newpage

\renewcommand{\abstractname}{\Large\bfseries ABSTRACT}
\begin{abstract}
{
Games are increasingly diverse, social and present in our every day
lives. The great amount of devices that can handle this games offers
challenges and opportunities. HTML is a set of tools that enables the
construction of multiplatform games; nevertheless, the set of tools
HTML is composed of changes rapidly and their support is miscellaneous.
Given this situation, this work aims to study the limitations of HTML
when applied to multiplatform game development. For that, we elaborated
a bibliographic research through articles, books and thesis. And a math
game prototype was developed, where the user can choose the veracity of
a given equation. With the expertise we got from it, and the data we
collected, we composed a list of limitations that are present currently
on HTML.
}
\\
\\
{\bfseries Keywords:} Games, HTML, Limitations, Multiplatform
\end{abstract}

%}}}
%LISTS
%{{{
%30 is ok
{\listoffigures}
\clearpage
{\listoftables}
\clearpage
{\tableofcontents}
%\listoftables
%}}}k
\chapter{INTRODUÇÃO}%{{{
\pagenumbering{arabic}
%\setcounter{page}{1} 

Desde a década de 90 jogos digitais se tornaram uma forma dominante
de recriação \autocite{gameDesignPatterns}. Devido a fatores
como a massificação dos dispositivos móveis inteligentes os
computadores deixaram de ser apenas uma ferramenta científica
e de negócios, tornando-se uma plataforma de diversão. Atualmente jogos
são utilizados em uma vasta gama de dispositivos \autocite[pp.
6]{crossPlatformMobileGameDevelopment}.

\autocite{HTML5CrossPlatformGameDevelopment} afirma que mais de 80\% do
tempo total gasto utilizando dispositivos móveis é na utilização de
aplicativos, e 32\% deste tempo é para jogar vídeo games.

Não obstante, devido a variedade de dispositivos, é difícil para
os criadores de jogos alcançarem todos os jogadores possivelmente
interessados em suas criações. \cite{html5Tradeoffs} afirma que a
maior dificuldade em capturar uma base de usuários é que o mercado
de dispositivos móveis é muito fragmentado e não existe uma única
plataforma popular.

Esta característica força os criadores de jogos a suportarem diversas
plataformas. \autocite{htmlSurvey} afirma que 81\% dos aplicativos mobile rodam em
pelo menos dois sistemas operacionais.
Para atingir múltiplas plataformas os desenvolvedores utilizam de variadas estratégias, 
sendo uma delas o HTML5.

Desde o lançamento da versão 5 do HTML, ele conquistando sua posição
como uma forma de desenvolver jogos para múltiplas plataformas.

Pelas características de multimídia adicionadas nesta
versão do HTML5 e pelo suporte horizontal a múltiplas plataformas que o
HTML provê \autocite{html5Tradeoffs}.

Muito pouco investimento é necessário para começar a desenvolver
jogos utilizando as tecnologias da WEB \autocite{html5mostwanted}.
Desenvolvedores de site podem reaproveitar o conhecimento direcionando-o
para o desenvolvimento de jogos. 

O interesse por parte dos desenvolvedores também é grande.
\autocite{htmlSurvey} cita que cerca de 59\% dos desenvolvedores estão
muito interessados em desenvolver aplicativos em HTML5.

As vantagens financeiras do desenvolvimento de aplicações multiplataforma
em HTML também são substanciais. O tempo de desenvolvimento de
uma aplicação em HTML5 é 67\% menor que aplicações nativas
\autocite[pp. 460]{html5Tradeoffs}. 

Alguns jogadores de Hattrick tem participado do jogo por mais de 
10 anos \autocite{gameCommunities}

Não obstante, apesar utilização da WEB ser gigantesca, criar
jogos dinâmicos e de tempo real não é seu primeiro objetivo
\autocite{html5mostwanted}. O processo de desenvolvimento de
aplicações HTML5 está em constante fluxo de aperfeiçoamento e novos
métodos, técnicas, e ferramentas estão aparecendo todo o tempo
\autocite{crossPlatformMobileGame}.

Neste contexto vê-se a necessidade de uma revisão das tecnologias,
suas fraquezas e especialidades. Este trabalho propõe analisar
as limitações do HTML5 quanto relativo a construção de jogos
multiplataforma. Através de revisão bibliográfica e da criação
de um protótipo de jogo multiplataforma. O protótipo escolhido foi
um jogo de matemática simples onde o usuário pode escolher se uma
questão matemática, e seu resultado dado estão corretos.

Com as informações coletas de maneira prática e teórica foram
registradas as limitações e uma análise simples sobre elas foi
efetuada.




%}}}

%}}}
\chapter{REVISÃO BIBLIOGRÁFICA}
%{{{
\section{JOGOS}
%{{{

Segundo \citet{indieGamesLemes}, jogo digital constitui-se em uma
atividade lúdica composta por uma série de ações e decisões,
limitadas por regras e pelo universo do game, que resultam em uma
condição final.

Apesar da definição clara, uma taxonomia dos jogos não é tarefa
trivial. Pela diversidade em itens e gêneros e a vasta quantidade de
dimensões que os vídeos games se encontram, uma categorização dos
jogos contemporâneos é extremamente difícil de desenvolver (muitos
já tentaram) \autocite[p. 60]{gamebenefits}.

Para corroborar com essa complexidade, \citet{gamebenefits} afirmam que
a natureza dos jogos tem mudado drasticamente na última década, se
tornando cada vez mais complexos, diversos, realísticos e sociais em
sua natureza.

Com as inovações nas tecnologias relacionadas ao HTML, novas
mecânicas e gêneros de jogos podem ser explorados na WEB. Sendo
que cada gênero acompanha um conjunto de desafios específicos.
Jogos de FPS (tiro em primeira pessoa) requerem menor latência de
rede, já jogos de RPG podem requerer vastas quantidades de cache
\autocite{html5mostwanted}.

\subsection{Benefícios}

Assim como as tecnologias, um grupo emergente de pesquisas sobre os
benefícios do jogos vem se desenvolvendo \citet{gamebenefits}. Apesar
de a quantidade de estudos sobre os males dos jogos ser muito maior do
que os estudos sobre seus benefícios, a quantidade de benefícios já
correlacionados aos jogos é substancial.

\citet{gamebenefits} demostram que vídeo games melhoram as funções
cognitivas, as capacidades criativas, e motivam uma visão positiva
diante a falha. Também segundo \citet{gamebenefits}, postura positiva
em relação a falha correlaciona-se com melhor performance acadêmica.

Benefícios em habilidades práticas também são observados em
usuários de jogos. Jogadores de jogos de tiro demonstram maior
alocação de atenção, maior resolução espacial no processamento
visual e melhor habilidades de rotação \autocite{gamebenefits}.

Estes aspectos positivos muitas vezes não recebem a atenção devida.
Habilidades espaciais derivadas de jogar jogos de tiro comercialmente
disponíveis são comparáveis aos efeitos de um curso universitário
que busca melhorar as mesmas habilidades \autocite{gamebenefits}. Estas
habilidades derivadas dos jogos são centrais para muitas áreas de
interesse humano. \citet{gamebenefits} afirmam que habilidades espaciais
estão diretamente relacionadas com o sucesso em ciência, tecnologia,
engenharia e matemática.

Outra outro ponto importante dos jogos é seu aspecto social. Apesar de a
mídia ter criado uma perspectiva negativa sobre jogos, especialmente
os violentos, a realidade é mais complexa do que se pensa. Jogadores
de jogos violentos, cuja jogabilidade encoraje a cooperatividade, são
mais prováveis de exibir comportamento altruísta fora do contexto dos
jogos, do que jogadores de jogos não violentos \autocite{gamebenefits}.

Além dos aspectos benéficos, adiante serão abordados alguns aspectos
técnicos e gerenciais dos jogos.

\subsection{MECÂNICA}

\citet{html5mostwanted} ressalta a importância do planejamento antes do
desenvolvimento de jogos. Ao criar jogos deve-se planejar o que se
pretende atingir e como chegar lá antes de se escrever qualquer
código, definições quanto a mecânica é um passo vital neste planejamento.

A mecânica é composta pelas regras do jogo. Quais as ações
disponíveis aos usuários, e seu funcionamento, é fortemente
influenciada pela categoria do jogo em questão. Dedicar-se na
elaboração de uma mecânica é tarefa quintessencial para a
construção de um jogo de sucesso.

A mecânica não pode ser simplesmente boa, nos melhores jogos ela é
intuitiva. O processo de o jogador entender por si a mecânica do jogo
é um componente vital para a sua satisfação. Se ele não entender as
regras do jogo quase que instantaneamente muitas pessoas vão perder o
interesse e desistir rapidamente \autocite{crossPlatformMobileGame}.

Ainda sobre a importância da mecânica \citet{html5mostwanted}, afirma
que se os gráficos e áudio são espetaculares mas a jogabilidade
é chata o jogador vai parar de jogar. A substância do jogo é sua
mecânica, então não invista muito em visual ao menos que isso
desempenhe um papel essencial no jogo.

Os desenvolvedores tem que evitar fazer o jogo para eles mesmos.
Falta de crítica antes do desenvolvimento também tende a gerar jogos ruins.
Bons jogos são aqueles que ao menos suprem as expectativas dos usuário.
\citet{indieGamesLemes} aponta alguns fatores procurados pelos usuários
de jogos: desafio, socializar, experiência solitária, respeito e
fantasia. Jogos que conseguem integrar o maior número destes aspectos 
em sua mecânica serão os mais apreciados.

A elaboração da mecânica em jogos desenvolvidos profissionalmente
pode ser integrada dentro de um processo de engenharia de software
O sumário também conta com um resumo sobre metodologias de
desenvolvimento de software contextualizada para a criação jogos.

Depois dos preparativos efetuados, pode-se começar a construção
do jogo. 

\subsection{Laço}

Jogos digitais geralmente operam através um laço que executa uma série
de tarefas a cada frame, construindo a ilusão de um mundo animado
\autocite[p. 31]{gwt}. Da perspectiva da programação, a parte
principal de um jogo é o laço onde o jogo é executado \autocite[p.
17]{crossPlatformMobileGameDevelopment}. A cada iteração do 
laço o processamento de entrada de dados do usuário e
alterações da cena tem de ser computados, tornando a otimização deste processo
importantíssima para que o jogo não se torne lento.

Na seção Otimizações para jogos em JavaScript será abordada as
opções da WEB na hora de construir um laço de jogo.

Outra decisão técnica importante na hora de desenvolver jogos é a 
seleção das plataformas alvo mais adequadas. Neste trabalho o foco será 
em  HTML5 para desktops e dispositivos móveis inteligentes. Não obstante,
na sessão a seguir serão descritas, genericamente, as plataformas que os
desenvolvedores podem selecionar ao desenvolver jogos, juntamente com 
seus benefícios e fraquezas.

\section{JOGOS MULTIPLATAFORMA}

Jogos multiplataforma são jogos que rodam em mais de uma plataforma.
Cada plataforma contém sua própria API (\textit{Application
Programming Interface}), sendo que se um aplicativo foi criado para
uma plataforma ele não vai poder ser utilizado em outra pois as
APIs são diferentes \autocite{crossPlatformMobileGameDevelopment}.
Além da API, dispositivos de múltiplas plataformas variam em seus
recursos, capacidades e qualidade. Devido a essas características,
desenvolvedores de jogos que almejem múltiplas plataformas, deparam-se
com uma nova gama de oportunidades e desafios.

Um destes desafios é fornecer \textit{feedback} suficiente para o
jogador pois muitas vezes o dispositivo é limitado em proporções,
som, tela. Devido a estes requerimentos, tendências como WEB design
responsivo (RWD) emergiram. Requerendo que os desenvolvedores busquem
criar interfaces cada vez mais fluídas e intuitivas o possível. As
tecnologias da Web também tiveram que acompanhar a mudança. CSS3 media
queries, tamanhos relativos, são exemplos de tecnologias desenvolvidas
com o foco em multiplataforma.

Outro desafio multiplataforma é suportar os vários ecossistemas
de software com tecnologias diferenciadas e versões de
software divergentes. Para que um jogo multiplataforma tenha
sucesso é necessário definir com cautela suas tecnologias.
\citet{html5mostwanted} afirma que: o estágio mais complicado e crucial
do desenvolvimento de jogo é a escolha das tecnologias utilizadas.

Designers de jogos tem as seguintes possibilidades quando em face
de desenvolver um jogo multiplataforma: criar um jogo web, um jogo
híbrido, ou nativo. As opções serão descritas a seguir.

\subsection{JOGOS WEB}

Um jogo web é um jogo que utiliza o HTML e ferramentas correlacionadas
para sua construção e disponibilização. Não obstante, o objetivo
primário da WEB nunca foi o desenvolvimento de jogos. Por muito
tempo os títulos famosos de jogos da web residiam em jogos como
\textit{Traviam}, desprovidos de animações, compostos basicamente por
formulários, imagens e textos. Durante esse período o interesse em
jogos WEB residia principalmente nas de casualidade, flexibilidade, e o
fator social dos jogos.

Mais recentemente é que a WEB começou a ser vista como de fato um
ambiente com interatividade para criação de jogos dos mais variados
gêneros. Publicar jogos baseados em texto é uma atividade cada vez
mais rara, podendo-se concluir que interface gráfica se tornou uma
funcionalidade mandatória \autocite{browserGamesTechnologyAndFuture}.
Jogos como BrowserQuest, Angry Birds, entre outros títulos expandiram
os conceitos do que é possível se fazer utilizando as ferramentas
da WEB. Segundo \cite[p. 28]{gwt} um bom exemplo que tem alcançado
bastante sucesso entre o público são os jogos adicionados no logotipo
do Google, chamados doodles.

Quando comparados a outras abordagens de criar jogos, jogos da WEB
contém vários aspectos positivos. Talvez o mais reconhecível
é o fato de com uma única base de código poder suprir uma gama
praticamente inesgotável de dispositivos. 

Comparável a base de clientes está a quantidade de desenvolvedores
WEB, os quais podem reaproveitar grande parcela do conhecimento
adquirido através do desenvolvimento de páginas WEB na criação de
jogos.

Sua distribuição também é superior ao estilo convencional de
aplicações desktop \autocite{browserGamesTechnologyAndFuture}. 
Por serem criados à partir das tecnologias da WEB, jogos na WEB, se
beneficiam de uma arquitetura construída para um ambiente social, sendo
relativamente mais fácil criar experiências sociais.

A performance é um ponto negativo da WEB, é difícil chegar a
performance comparável a abordagem nativa. Não obstante, esse
problema é cada vez menor o hardware dos dispositivos são cada vez
melhores e as tecnologias tecnologias de software também avançam
substancialmente.

Existem inconsistências nas implementações das especificações
WEB o que leva a comportamento inesperados em alguns caos, sendo
necessário desenvolver regras específicas para dispositivos e versões
de navegadores.

Outro aspecto negativo da WEB é que nem todas as funcionalidades dos
dispositivos estão especificados com as tecnologias da WEB e muitas
vezes os recursos do dispositivo ficam sub utilizados.

Além dos jogos web, há a possibilidade de criar jogos nativos e híbridos.

\subsection{DESENVOLVIMENTO DE JOGOS NATIVOS}

Uma aplicação nativa é uma aplicação que foi desenvolvida para ser
utilizada em uma plataforma ou dispositivo específico \autocite[p.
7]{aSeriousContender}. Aplicativos nativos tendem a oferecer uma
experiência mais próxima com a do resto do sistema operacional o qual
está rodando. Potencialmente softwares nativos são mais rápidos
que suas alternativas da WEB, visto que interagem com o dispositivo
através do sistema operacional. Diferentemente dos jogos WEB, que
necessitam que o navegador interaja com o sistema operacional, para
por sua ver interagir com o dispositivo. Por terem acesso total ao
dispositivo, aplicativos nativos podem aproveitar o hardware da melhor
forma possível e oferecer ao usuário a melhor experiência possível
\autocite[p. 7]{aSeriousContender}.

Desenvolver jogos nativos tende a ser mais caro que a alternativa da WEB,
visto que é necessário duplicar funcionalidades em sistemas distintos
e manter um profissional por ambiente suportado. As versões nativas
são totalmente incompatíveis entre si, impossibilitando reúso em múltiplas plataformas.

A alternativa hibrida fica em meios ao desenvolvimento nativo e WEB e
será descrita abaixo.

\subsection{JOGOS HÍBRIDOS}

A alternativa híbrida é uma tentativa de beneficiar-se das melhores
características da abordagem nativa e o melhor do desenvolvimento
WEB. Muitas vezes desenvolve-se aplicações híbridas utilizando as
tecnologias da WEB só que ao invés de disponibilizar a aplicação
através de um navegador a aplicação WEB é instalada como um
aplicativo normal. A aplicação roda em uma WebWiew que é um
componente do sistema operacional capaz de rodar as tecnologias da
WEB. Desta forma o aplicativo WEB conversa diretamente com o sistema
operacional, não necessitando da intervenção de um software terceiro
para mediar a interação.

Outra possibilidade híbrida é escrever o software em uma linguagem
e gerar binários para as plataformas alvo. Utilizando o Xamarin é
possível desenvolver em C\# e compilar para diversas plataformas
nativamente. Através dessa abordagem é possível beneficiar-se de um
aplicativo nativo e eliminar grande parte da duplicação geralmente
imposta \footnote{Os frameworks multiplataforma dos apêndices contém
tecnologias similares ao Xamarin como o Titanium}.

Visto que a estratégia híbrida geralmente tem acesso ao sistema
operacional é possível criar APIs para acessar recursos não sempre
disponíveis para a plataforma WEB. Soluções como o PhoneGap
adotam essa estratégia para possibilitar granular controle sobre os
dispositivos através de APIs JavaScript implementadas nativamente.

Outro benefício da estratégia híbrida em relação a WEB é que
ela permite empacotar as aplicações com um experiência exata a dos
softwares nativos. Sendo imperceptível para o usuário final a
diferença ente um aplicativo híbrido e um nativo.

Não obstante, segundo \cite[p. 8]{aSeriousContender} a diferença
entre o que é possível com a estratégia híbrida e a WEB está
diminuindo devido ao grande esforço da comunidade WEB para prover
novas especificações.

%}}}
\section{WEB}
%{{{

\subsection{OPEN WEB}

A OWP (\textit{Open Web Platform}) é uma coleção de tecnologias
livres, amplamente utilizadas e padronizadas. Quando uma tecnologia
se torna amplamente popular, através da adoção de grandes empresas
e desenvolvedores, ela se torna candidata a adoção pela OWP. Os
benefícios de utilizar tecnologias da OWP são vários. \cite[p.
3]{svgTime} cita que o tecnologias padronizadas tem um maior ciclo de
vida e são mais fáceis de mudar. Da mesma maneira, as tecnologias
da WEB são benéficas devido sua grande adoção, permitindo que
aplicações baseadas nelas tenham impacto na maior quantidade de
clientes possível.

Não obstante, mais do que tecnologias a OWP é um conjunto de
filosofias as quais a WEB se fundamenta \autocite{openWebDefinition}. Entre outras, a Open Web
busca transparência, imparcialidade nos processos de criação
e padronização de novas tecnologias. Retro compatibilidade com
as especificações anteriores. Consenso entre o mercado e o meio
acadêmico, nunca um distanciando-se muito do outro \footnote{Mais
informações sobre a Open WEB podem ser encontradas no seguinte
endereço http://tantek.com/2010/281/b1/what-is-the-open-web}.

Várias pessoas, empresas e comunidades estão interessadas neste
processo, cada qual com seus próprios conjunto de ideias sobre como
a WEB deveria funcionar. Mas para que a crescente quantidade de
dispositivos possa acessar a riqueza que o HTML5 permite, padrões
precisam ser definidos \autocite[p. 5]{aSeriousContender}.

A W3C é uma comunidade responsável por boa parte das
especificações da web como: HTML (em conjunto com a WHATWG), CSS,
entre outras\footnote{Uma lista completa das especificações mantidas pela
W3C pode ser encontrada em: http://www.w3.org/TR/}. Outros grupos
detém responsabilidade por outras tecnologias da OWP, como a ECMA,
responsável pelo JavaScript; ou Kronos, responsável pelo WebGL.

Na W3C o processo de desenvolvimento de especificações consiste
na elaboração de rascunhos (\textit{working drafts}), criados por grupos
de trabalhos (\textit{workin groups}) de especialistas no
assunto, que passam por vários passos de revisão até se tornarem
recomendações. As recomendações podem ser implementadas com
segurança de que a especiação não mudará substancialmente.

Apesar do processo da W3C ser rigoroso, está longe de perfeito. A
especificação final do HTML4 contava com quatro erros publicados
via errata \autocite{HTML5}. Não obstante, o cenário é animador
\citet{html5mostwanted} cita que as tecnologias da Open WEB tem
evoluído desde os princípios da internet e já provaram sua robusteza
e estabilidade enquanto outras tecnologias crescem e morrem ao redor
dela.

A tecnologia chave que inaugurou e alavancou este processo é o HTML.
%}}}
\section{HTML}
%{{{

HTML (\textit{Hyper Text Markup Language}) é uma linguagem de
marcação que define a estrutura semântica do conteúdo das páginas
da web. Criada por Tim Berners Lee em 1989 no CERN. HTML é a tecnologia
base para a criação de páginas web e aplicativos online. A parte
denominada: "\textit{Hyper Text}", refere-se a links que conectam
páginas HTML umas as outras, fazendo a Web como conhecemos hoje
\autocite{mdn2015}.

A última versão do HTML é o HTML5, iniciado pela WHATWG e
posteriormente desenvolvido em conjunto com a W3C. Seu rascunho foi
proposto em 2008 e ratificado em 2014. Após 2011, a última chamada
de revisão do HTML5, a WHATWG decidiu renomear o HTML5 para HTML
\autocite{htmlIsTheNewHtml5}. Não obstante, o termo HTML5 permanece em
utilização pela W3C.

Além da nomenclatura, exitem pequenas diferenças nas especificações
da W3C e WHATWG. A W3C vê a especificação do HTML5 como algo fechado,
inclusive já iniciou o desenvolvimento do HTML 5.1. Já a WHATWG vê o
HTML5 como uma especificação viva. A postura da W3C tende a criar uma
especificação estável, já a da W3C reflete mais a realidade dos
navegadores, que nunca implementam uma versão completamente. A Mozilla
utiliza a especificação da WHATWG no desenvolvimento do Firefox e
recomenda a da W3C para sistemas que requeiram maior estabilidade. Neste
trabalho optamos pela nomenclatura da WHATWG, utilizamos o termo HTML em
detrimento a HTML5, sempre que semanticamente viável.

HTML foi especificado baseando-se no padrão SGML (\textit{Standard Generalized
Markup Language}).

Alguns benefícios do SGML são:
\begin{itemize}
    \item Documentos declaram estrutura, diferentemente de aparência
, possibilitando otimizações nos ambientes de uso (tamanho de tela,
etc);
    \item São portáveis devido a definição de tipo de documento
(\textit{document type declaration}).
\end{itemize}

Apesar de o SGML especificar a não definição de aparência, os criadores de
navegadores constantemente introduziam elementos de apresentação como o
piscar, itálico, e negrito, que eventualmente acabavam por serem inclusos
na especificação. Foi somente nas últimas versões que elementos de
apresentação voltaram a ser proibidos reforçando as propostas chave
do HTML como uma linguagem de conteúdo semântico, incentivando a
utilização de outras tecnologias como o CSS para responder as demandas de
apresentação.

Além do HTML, existe o XHTML, que é uma iniciativa de utilização de
XML nas páginas da web. O XML é um padrão mais rigoroso que SGML e
resulta em páginas sem problemas de sintaxe e tipografia. 
Alguns estimam que 99\% das paginas HTML de hoje
contenham ao menos um erro de estrutura \autocite{diveIntohtml}.
Uma das maiores vantagem do XML é que sistemas sem erros de sintaxe
que podem ser facilmente interpretadas por outras tecnologias como
sistemas de indexação, buscadores, etc.

Para transformar o HTML em algo visível os navegadores utilizam motores
de renderização. O primeiro passo efetuado por esses sistemas é
decodificar o documento HTML para sua representação em memória. Este
processo dá-se através da análise (\textit{parsing}) e posterior
tokenização, que é a separação do HTML em palavras chave que o
interpretador pode utilizar. Diferentemente do XHTML, HTML não pode
ser decodificado através de tokenização tradicional. Deve-se ao HTML
ser amigável ao programador, aceitando erros de sintaxe, dependente
de contexto, buscando entregar a melhor aproximação possível. 
Segundo \citet{howBrowsersWork} essa é a maior razão do HTML ser tão popular - 
ele perdoa os erros e torna a vida dos autores da WEB mais fácil. Esta
característica deu origem a uma especificação para renderizar HTML
(\textit{HTML parser}).

Antes do HTML5 várias versões foram propostas, algumas radicais
em seus preceitos. O XHTML 2.0, por exemplo, quebrava com toda
a compatibilidade das versões anteriores e acabou por sendo descontinuado.
Outrossim, a maioria das versões HTML de grande sucesso foram versões de
retrospectiva (\textit{retro-specs}). Versões que não tentavam
idealizar a linguagem, buscando alinhar-se com os requerimentos do
mercado \autocite{diveIntohtml}. Não obstante, a ideia que a melhor forma
de ajustar o HTML é substituindo ele por outra coisa ainda aparece de tempos
em tempos \autocite{diveIntohtml}.

Uma página HTML consiste em elementos que podem ter seu comportamento
alterado através de atributos. Um elemento é o abrir fechar de
uma tag e todo o conteúdo que dentro dele reside \autocite[p.
10--11]{htmlAndCssDucket}. Por exemplo, na figura \ref{fig:htmlSample} o elemento
meta (<meta>) tem um atributo \textit{charset}, que especifica o formato de 
codificação do documento.

\begin{figure}[H]
\centering
\begin{verbatim}
<}!DOCTYPE HTML>
<html lang="en-US">
<head>
	<meta charset="UTF-8">
	<title></title>
</head>
<body>
    <video>
        <span>Seu navegador não suporta vídeo</span>
    </video>
</body>
</html>
\end{verbatim}
\caption{Exemplo de documento HTML}
\label{fig:htmlSample}
\end{figure}

Na sua versão inicial, o HTML contava com 18 elementos; atualmente
existem aproximadamente cem \autocite{diveIntohtml}. Não obstante, foi
no HTML5 que a maior parte dos elementos que viabilizam a construção
de jogos foram adicionados.

Uma das características do HTML que o torna tão popular é seu
interesse em manter manter a retrocompatibilidade. Interpretadores
HTML atingem isso ignorando os elementos que não conhecem, tratando
seu vocabulário exclusivamente. Esse mecanismo permite que os
desenvolvedores incluam marcação de reserva dentro dos elementos
que podem não ser suportados. O elemento \textit{span} na figura
\ref{fig:htmlSample} só aparecerá para o usuário caso seu navegador
não suporta a tag vídeo.

Além da convencional linguagem de marcação, HTML é muitas vezes
interpretado como um conceito guarda chuva para designar as tecnologias
da web. Segundo \citet{diveIntohtml} algumas dessas tecnologias (como geolocalização) estão em especificações separadas mas são tratadas como HTML5 também. Outras
tecnologias foram removidas do HTML5 estritamente falando, mas são tratados
como HTML5 (como a API de armazenamento de dados).

\begin{figure}[H]
    \centering
    \includegraphics[width=0.8\textwidth,natwidth=610,natheight=642]{html5.jpg}
    \caption{Suíte HTML}
    \source{http://64vision.com/HTML5-what-is-it}
\end{figure}

Uma tecnologia fortemente entrelaçada com o HTML é o DOM.
Tendo uma relação próxima de um para um com a marcação
\autocite{howBrowsersWork}. DOM permite a interação entre documentos
HTML e as demais tecnologias da WEB de uma forma fácil e padronizada.

\subsection{DOM}
%{{{

O modelo de documento de objetos (\textit{Document Object Model}) é
a representação em memória de uma árvore de elementos HTML. Esta
representação é definida por um conjunto de objetos, unicamente
identificados e dispostos em forma de grafo, que busca facilitar a
manipulação de elementos através de JavaScript.

A primeira versão do DOM, DOM nível zero, foi especificada no
HTML 4 e permitia manipulação parcial dos elementos. Foi somente
com a especificação do JavaScript em 1998 que o DOM nível 1 foi
especificado, permitindo a manipulação de qualquer elemento. DOM
nível 2 e 3 seguiram com melhorias nas consultas aos elementos e CSS.

\begin{figure}[H]
\centering
\begin{verbatim}
    var elementos = document.querySelector( ".main, #sceen"  );
    var elementosB = document.querySelectorAll( "a.minhaClasse, p"  );
\end{verbatim}
\caption{Exemplo de utilização de seletores do DOM em JavaScript}
\label{fig:selectorsSample}
\end{figure}

A API de seletores (\textit{querySelector}) do DOM permite alto
nível de precisão e performance para buscar elementos. A figura
\ref{fig:selectorsSample} exemplifica a utilização dos seletores
do DOM em um documento JavaScript. O método \textit{querySelector}
seleciona o primeiro elemento em conformidade com o padrão
especificado. Já o método \textit{querySelectorAll} seleciona todos os
elementos que estão em acordo com o padrão especificado.

DOM também conta com uma API de eventos que possibilita que, através
de JavaScript, se saiba quando algum evento interessante aconteceu.
Cada evento é representado por um objeto baseado na interface
\textit{Event} e pode ter campos e funções adicionais para prover
maior informação do evento ocorrido \autocite{devdocs}.

Os eventos podem ser criados pelo usuário ou serem lançados pelo
navegador, possibilitando uma manipulação consistente dos mais
variados aspectos de uma aplicação.

Manipulação de entrada de comandos, e muitas varias outras APIs do
HTML, como o IndexedDB, se dá através de eventos, tornando o assunto
relevante aos jogos \footnote{O site http://devdocs.io/dom\_events/
contém uma lista dos eventos lançados automaticamente pelos
navegadores}.
%}}}

Fortemente entrelaçado com o HTML e o DOM está o CSS que possibilita
customizar a apresentação do markup possibilitando experiencias muito
mais ricas do que o conteúdo bruto.
%}}}
\section{CSS}
%{{{
CSS (\textit{Cascading Style Sheets}) é uma linguagem de folhas de
estilo criada por Håkon Wium Lie em 1994 com intuito de definir a
apresentação de páginas HTML. CSS, juntamente com JavaScript e HTML,
compõem as tecnologias centrais no desenvolvimento WEB tornando-se
parte da OWP; sua especificação é atualmente mantida pela W3C.

O termo \textit{Cascading} refere-se ao fato de regras serem
herdadas pelos filhos de um elemento, eliminando grande parcela de
duplicação antes necessária para estilizar uma página. Segundo
\citet{html5mostwanted} pode-se expressar regras gerais que são
"cascateadas" para muitos elementos, e então sobrescrever os elementos
específicos conforme a necessidade.

Segundo \citet[p. 23--24]{CascadingStyleSheets}:
\begin{quote}
CSS possibilita a ligação tardia (\textit{late biding}) com
páginas HTML. Essa característica é atrativa para os publicadores
por dois motivos. Primeiramente pois permite o mesmo estilo em várias
publicações, segundo pois os publicadores podem focar-se no conteúdo
ao invés de se preocuparem-se com detalhes de apresentação.
\end{quote}

Esta ligação tardia permitiu diferenciação entre apresentação e
estrutura, sendo neste caso o CSS responsável pela apresentação. Esta
característica é uma das ideias pioneiras do SGML, motivo que tornou a
utilização do CSS tão conveniente para o desenvolvimento WEB.
Antes do CSS era impossível ter estilos diferenciados para diferentes
tipos de dispositivos, limitando a aplicabilidade dos documentos.
Com CSS também tornou-se possível que o usuário declare suas próprias
folhas de estilo, um recurso importante para acessibilidade.

Estruturalmente falando, CSS é formado por um conjunto de regras,
dentro de uma tag HTML denominada \textit{style}, que são agrupadas
por seletores em blocos de declaração. Os elementos selecionados são
denominados o assunto do seletor \autocite{cssSelectors}. Seletores tem
o intuito de definir quais partes do documento HTML serão afetadas por
determinado bloco de declaração.

CSS é dividido em módulos, que representam conjuntos de
funcionalidades, contendo aproximadamente 50 deles. Cada módulo evolui
separadamente, esta abordagem é preferível pois permite uma maior
quantidade de entrega de novas funcionalidades. Visto que novos recursos
não dependem da aceitação de outros para serem disponibilizados.
Além do módulos, CSS também é organizado por perfis e níveis.

Os perfis do CSS organizam a especificação por dispositivo de
utilização. Existem perfis para dispositivos móveis, televisores,
impressoras, etc. A aplicabilidade das regras do CSS varia dependendo do
perfil. O conteúdo do elemento \textit{strong}, por exemplo, pode ser
traduzido em uma entonação mais forte em um leitor de telas, já em um
navegador convencional pode ser apresentado como negrito.

Já os níveis organizam o CSS por camadas de abstração. Os níveis
inferiores representam as funcionalidades vitais do CSS, os níveis
superiores dependem dos inferiores para construir as funcionalidades
elaboradas.

A primeira especificação do CSS, CSS1 (ou nível 1) foi lançada em
1996. Em 1997 foi lançado o CSS2 com o intuito de ampliar a completude
do CSS1. Em 1998 iniciou-se o desenvolvimento do CSS3 que ainda continua
em 2015. Além do nível 3 existem módulos de nível 4 no CSS, não
obstante o termo CSS3 ainda é o mais utilizado.

Apesar da clara evolução das versões do CSS, esse processo nem
sempre é linear. Em 2005 o grupo de trabalho do CSS decidiu aumentar a
restrição de suas especificações rebaixando o CSS 2.1, Seletores do
CSS3 e Texto do CSS3 de recomendações para rascunhos.

\begin{figure}[H]
    \centering
    \includegraphics[width=0.8\textwidth,natwidth=610,natheight=642]{cssModules.png}
    \caption{Os módulos do CSS}
    \source{https://commons.wikimedia.org/wiki/File:CSS3\_taxonomy\_and\_status-v2.png}
\end{figure}

A última versão do CSS, o CSS3, introduziu várias funcionalidades
relevantes para jogos, como \textit{media-queries}, transições,
transformações 3D, entre outros.

\subsection{Media Queries}

Media Queries permitem aplicar regras a dispositivos específicos,
dependendo de suas capacidades, como resolução, orientação, tamanho
de tela, entre outros. A especificação prevê a possibilidade de
condicionalmente carregar arquivos JavaScript ou CSS, ou utilizar
seletores dentro do CSS de acordo com regras de Media Queries.

Esse carregamento condicional  permite implementar fluidez e
adaptabilidade de layout para diferentes resoluções. Que segundo
\citet{HTML5CrossPlatformGameDevelopment} é o mais importante aspecto do
desenvolvimento de jogos multiplataforma com as tecnologias da WEB.

\begin{figure}[H]
\centering
\begin{verbatim}
@media only screen and (min-width: 1024px) {
    background-color: green;
}
\end{verbatim}
\caption{Exemplo de Media Query}
\label{fig:MediaQuery}
\end{figure}

A figura \ref{fig:MediaQuery} demostra a aplicação de uma regra via
seletor Media Query, aplicando o a cor de fundo para dispositivos com no
mínimo 1024 pixels de resolução.

CSS nível 4 permite a utilização de media queries (\textit{Custom
Media Queries}) criados pelo usuário, com regras e definições
customizadas. A figura \ref{fig:MediaQueryCustom} demostra as novas
possibilidades de definição de media queries tanto em CSS como em
JavaScript.

\begin{figure}[H]
\centering
\begin{verbatim}
@custom-media --narrow-window (max-width: 30em);

<script>
CSS.customMedia.set('--foo', 5);
</script>

\end{verbatim}
\caption{Exemplo de media queries customizados}
\label{fig:MediaQueryCustom}
\source{https://developer.mozilla.org/en-US/docs/Web/CSS/MediaQueries}
\end{figure}

%falar de tamanhos absolutos vs relativo
%Unidades vw e vh para tamanho do viewport

\subsection{Transições}

Transições são uma forma de adicionar animações em uma página
web. Estas animações são compostas por um estado inicial e um final.
A especificação de transições permite grande controle sobre seus
estados, habilitando o desenvolvedor a controlar o tempo de execução,
os estados intermediários, e efeitos aplicados uma transição.

Para utilizar transições, assim como em uma máquina de estados,
precisamos identificar estados e ações. Estados são seletores do CSS
e ações são modificações realizadas entre esses dois seletores CSS
\autocite{html5mostwanted}.

Transições são interessantes em jogos, especialmente pois muitos
navegadores suportam aceleração de GPU (Unidade de processamento
gráfico) para estas operações. Isso garante grandes benefícios de
performance sobre implementações diretamente em JavaScript.

Segundo \citet{html5mostwanted}, transições nos permitem construir jogos
degradáveis pois os interpretadores de CSS são amigáveis; se eles
encontrarem propriedades desconhecidas eles simplesmente as ignoram e
continuam a funcionar.

\begin{figure}[H]
\centering
\begin{verbatim}
div {
    width: 100px;
    height: 100px;
    background: red;
    transition: width 2s;
}

div:hover {
    width: 300px;
}

\end{verbatim}
\caption{Exemplo de transição}
\label{fig:CSSTransition}
%\soruce{http://www.w3schools.com/CSS/css3\_transitions.asp}
\end{figure}

A figura \ref{fig:CSSTransition} demostra a utilização de uma
transição de tamanho em uma \textit{div} quando o mouse está sobre o
elemento. No período de 2 segundos a largura da \textit{div} vai de 100
pixels ara 300 pixels.

Atualmente um conjunto finito de propriedades podem ser animadas
com transições, e essas lista tende a mudar com o tempo, cabe ao
desenvolvedor assegurar-se que determinada propriedade está disponível
\autocite{mdnTransitions}.

\subsection{Transformações 3D}

Transformações é outra tecnologia do CSS3 que permite grande
flexibilidade na construção de jogos. Transformações permitem que
elementos sejam traduzidos, rotacionados, escalados e distorcidos em um
espaço de duas dimensões \autocite{html5mostwanted}.

A transformação demonstrada na figura \ref{fig:CSSTransform} escala o
tamanho do elemento com a classe (\textit{test}) para vinte porcento a
mais do seu tamanho original. Perceba também os comandos repetidos com
o prefixo ms e WebKit. Esse tipo de abordagem é comum para tecnologias
que não passam de rascunhos na especificação.

Assim como transições, as transformações são muitas vezes aceleradas
via GPU incrementando a performance de animações criadas com a tecnologia.

\begin{figure}[H]
\centering
\begin{verbatim}
<style>
.test:hover
{
        -webkit-transform: scale(1.2);
        -ms-transform: scale(1.2);
        transform: scale(1.2);
}
</style>
\end{verbatim}
\caption{Exemplo de transformação}
\label{fig:CSSTransform}
\end{figure}

\subsection{CSS 4}

Apesar de o termo CSS 4 ser bastante utilizado, o grupo de trabalho do CSS
não considera mais a existência de versões, como foi até o CSS3.
Não obstante existem recursos cuja especificação está avançada e não estavam presentes
no CSS 3 quando este foi lançado, dentre estas funcionalidades inclui-se:

\begin{itemize}
\item Suporte a variáveis no CSS
\item Media queries customizadas
\item Funções de cores como: color(), hwb() e gray()
\item Suporte a filtros
\end{itemize}

Recursos recentes do CSS muitas vezes não estão presentes nos
navegadores, não obstante muitos deles são interessantes no contexto
de desenvolvimento de jogos, como o suporte a variáveis.

O projeto cssnext http://cssnext.io/ é uma iniciativa para permitir a
utilização dos recentes recursos do CSS mesmo sem os mesmos estarem
implementados nos navegadores. O projeto funciona compilando o código
não suportado em algo compatível com versões para as versões
implementadas pelos navegadores.

Além da apresentação, recurso vital para jogos, e aplicativos web em
geral, é a iteratividade. Com as tecnologias da WEB esta iteratividade
é atingida através do JavaScript.
%}}}
\section{JAVASCRIPT}
%{{{

EMACScript, melhor conhecida como JavaScript, criada por Brendan Eich em
1992, é a linguagem de script da Web. Devido a tremenda popularidade
entre comunidade de desenvolvedores a linguagem foi abraçada pela W3C e
atualmente é um dos componentes da Open Web.

As definições da linguagem são descritas na especificação ECMA-262.
Esta possibilitou o desenvolvimento de outras implementações além da
original (SpiderMonkey) como o Rhino, V8 e TraceMonkey; bem como
outras linguagens similares como JScript da Microsoft e o ActionScript
da Adobe.

Segundo a \citet{ecmaSpecificaton}:
\begin{quote}
Uma linguagem de script é uma linguagem de programação que é
usada para manipular e automatizar os recursos presentes em um dado
sistema. Nesses sistemas funcionalidades já estão disponíveis
através de uma interface de usuário, uma linguagem de script é
um mecanismo para expor essas funcionalidades para um programa
protocolado.
\end{quote}

No caso de JavaScript na web, os recursos manipuláveis são o conteúdo
da página, elementos HTML, elementos de apresentação,
a própria janela do navegador e variados outros recursos que tem
suporte adicionado por novas especificações.

A intenção original era utilizar o JavaScript para dar suporte aos já
bem estabelecidos recursos do HTML, como para validação, alteração
de estado de elementos, etc. Em outras palavras, a utilização do
JavaScript era opcional e as páginas da web deveriam continuar
operantes sem a presença da linguagem.

Não obstante, com a construção de projetos Web cada vez mais complexos, as
responsabilidades delegadas ao JavaScript aumentaram a ponto que a
grande maioria dos sistemas web não funcionarem sem ele.
JavaScript não evoluiu ao passo da demanda e muitas vezes carece de
definições expressivas, completude teórica, e outras características
de linguagens de programação mais bem estabelecidas, como C++ ou
Java \autocite{crossPlatformMobileGame}.

A última do JavaScript, o JavaScript 6, é um esforço nessa direção.
JavaScript 6 ou EMACScript Harmonia, contempla vários conceitos de
orientação a objetos como classes, interfaces, herança, tipos, etc.
Não obstante o suporte ao JavaScript 6 é apenas parcial em todos
os navegadores. O site http://kangax.github.io/compat-table/es6/
apresenta um comparativo de suporte das funcionalidades do JavaScript.

Segundo \citet{ecmaSupport}, o suporte no início de 2015 era o seguinte:

\begin{itemize}
    \item Chrome: 30\%
    \item Firefox: 57\%
    \item Internet Explorer : 15\%
    \item Opera: 30\%
    \item Safari: 19\%
\end{itemize}

Estes esforços de padronização muitas vezes não são rápidos
o suficiente para produtores de software web, demora-se muito até
obter-se um consenso sobre quais as funcionalidades desejadas em
determinada versão e seus detalhes de implementação. Além da
espera por especificações, uma vez definidas, é necessário que os
navegadores especificado.

O projeto babel https://github.com/babel/babel é um compilador de
JavaScript 6 para JavaScript 5. Permitindo que, mesmo sem suporte, os
desenvolvedores possam usufruir dos benefícios da utilização do
JavaScript 6 durante o tempo de desenvolvimento, gerando código em
JavaScript 5 para rodar nos navegadores.

Alternativamente, existe uma vasta gama de conversores de código
(\textit{transpilers}) para JavaScript; possibilitando programar
em outras linguagens posteriormente gerando código JavaScript
\footnote{Uma lista das tecnologias para converter código HTML pode ser encontrada nos apêndices}.
Entretanto, essa alternativa tem seus pontos fracos, necessita-se
de mais tempo de depuração, visto que o JavaScript gerado não é
conhecido pelo desenvolvedor, e provavelmente o código gerado não
será tão otimizado, nem utilizará os recursos mais recentes do
JavaScript.

Mesmo com suas fraquezas amplamente conhecidas, JavaScript está
presente em praticamente todo navegador atual. Sendo uma espécie de
denominador comum entre as plataformas. Essa onipresença torna-o
integrante vital no processo de desenvolvimento de jogos multiplataforma
em HTML5. Vários títulos renomeados já foram produzidos que fazem
extensivo uso de JavaScript, são exemplos: Candy Crush Saga, Angry
Birds, Dune II, etc.

Jogos Web são geralmente escritos na arquitetura cliente servidor,
JavaScript pode rodar em ambos estes contextos, para tanto, sua
especificação não define recursos de plataforma. Distribuidores do
JavaScript complementam a o JavaScript com recursos específicos para
suas plataformas alvo. Por exemplo, para servidores, define-se objetos como:
console, arquivos e dispositivos; no contexto de cliente,
são definidos objetos como: janelas, quadros, DOM, etc.

Para o navegador o código JavaScript geralmente é disposto no elemento
script dentro de arquivos HTML. Quando os navegadores encontram esse
elemento eles fazem a requisição para o servidor e injetam o código
retornado no documento, e a não ser que especificado de outra forma,
iniciam sua execução.

\subsection{JAVASCRIPT 7}

Antes da finalização da especificação 6, algumas funcionalidades
do JavaScript 7 já haviam sido propostas. Na página
https://github.com/tc39/ecma262 pode-se conferir os itens propostos e
seu estágio de evolução. A figura \ref{fig:ecma7} é a tabela de
funcionalidades sugeridas e seu estágio no caminho da especificação.

Alguns dos recursos esperados para o JavaScript 7 são: guards,
contratos e concorrência no laço de eventos \autocite{ecma7}.

\begin{figure}[H]
    \centering
    \includegraphics[width=0.8\textwidth,natwidth=610,natheight=642]{ecma7.png}
	\caption{Propostas do ECMA 7}
	\source{https://github.com/tc39/ecma262}
    \label{fig:ecma7}
\end{figure}

\subsection{ASM.JS}% o correto é asm.js

Asm.js é um subconjunto da sintaxe do JavaScript a qual permite
grandes benefícios de performance quando em comparação com
JavaScript normal. Entretanto, não é trivial escrever código em
asm.js e geralmente a criação de código asm.js é feita através
da conversão de outras linhagens como C. O projeto Emscripten
https://github.com/kripken/emscripten pode ser utilizado para gerar
código em asm.js é utilizado pelo motor de jogos Unity 3D e Unreal.

No contexto dos jogos performance é um fator de extrema importância
asm.js se destaca por utilizar recursos que permitam otimizações
antes do tempo (\textit{ahead of time optimizations}). Grade parcela
da performance adicional, em relação ao JavaScript, é devido a
consistência de tipo e a não existência de um coletor de lixo
(\textit{garbage collector}) a memória é gerenciada manualmente
através de um grande vetor. Esse modelo simples desprovido de
comportamento dinâmico, sem alocação e desalocação de memória,
apenas um bem definido conjunto de operações de inteiros e flutuantes
possibilita grade performance e abre espaço para otimizações.

O desenvolvimento do asm.js iniciou-se no final de 2013 não obstante a
maioria dos navegadores não implementam ou implementam parcialmente o
rascunho. O motor JavaScript da Mozilla, SpiderMonkey, é a exceção,
implementando a grande maioria dos recursos do asm.js.

\subsection{WEB Assembly}
%{{{
Web Assembly é uma tecnologia que pretende definir um formato de
máquina da Web. A tecnologia ainda está em seus estágios iniciais
de desenvolvimento, nem conta com um grupo de trabalho. Não obstante,
sabe-se que WEB Assembly irá permitir que outras linguagens além do
JavaScript gerem código binário que rode nos navegadores com grande
ganhos de performance e flexibilidade.

Além da versão binária, otimizada para performance, uma versão em
texto também está prevista, ideal para desenvolvimento e depuração.
Bibliotecas aplicações que requeiram grande performance como
motores de física, simulações e jogos em geral vão se beneficiar
substancialmente com o Web Assembly.

A iniciativa do Web Assembly está sendo desenvolvida pelo Google,
Microsoft, Mozilla, entre outros, tornando a proposta uma possibilidade
promissora. Seu objetivo não é substituir o JavaScript, outrossim
habilitar que aplicações que necessitem de grande performance possam
ser inclusas na WEB. A ideia do WEB Assembly é uma continuação
do trabalho do asm.js, uma forma de trazer performance similar a
nativa eliminando grande parte das abstrações que o traz JavaScript
embutidas.

Visto que os desenvolvedores de motores JavaScript terão que colocar
o código do Web Assembly na mesma base que o do JavaScript as
expectativas são de que o JavaScript consiga aproveitar partes
da implementação do Web Assembly incrementando a performance do
JavaScript.

A aplicabilidade do Web Assembly em jogos em produção ainda
é praticamente nula. Até então apenas um polyfill do Web
Assembly está disponível e pode ser encontrado no seguinte link
https://github.com/Web\_Assembly/polyfill-prototype-1. Mas conforme a
especificação evolui a probabilidade é que as empresas interessadas
implementem a especificação em seus navegadores e os desenvolvedores
de jogos comecem a integrar a tecnologia em suas aplicações.

%}}}
\subsection{Web Animations}
%{{{

Web Animations é uma especificação em rascunho que define uma forma
imperativa de manipular animações através de JavaScript. Como
demonstrado na figura \ref{fig:webAnimations} a tecnologia vai permitir
manipular as animações de elementos do DOM, com a possibilidade de
filtrar por tipo de animação, alterar a taxa de animações, o tempo
de execução, entre outras propriedades de uma forma dinâmica -
através de scripts.

Visto que Web Animations lida diretamente com o DOM, animações podem
ser aplicadas para SVG além de CSS, servindo como uma tecnologia para
unificar animações.

Grande controle sobre animações é desejável para os
jogos, não obstante, visto que a especificação é muito
nova somente o Google Chrome a implementa. A biblioteca
\url{https://github.com/web-animations/web-animations-js} serve como
polyfill para os demais navegadores.

\begin{figure}[H]
    \centering
    \begin{verbatim}
elem.getAnimations().filter(
  animation =>
    animation.effect instanceof 
    KeyframeEffectReadOnly &&
    animation.effect.getFrames().some(
      frame => frame.hasOwnProperty('transform')
    )
).forEach(animation => {
  animation.currentTime = 0;
  animation.playbackRate = 0.5;
});
    \end{verbatim}
	\caption{Exemplo de utilização de WEB Animatios}
	\source{http://www.w3.org/TR/WEB-animations/}
    \label{fig:webAnimations}
\end{figure}

O site \url{http://web-animations.github.io/web-animations-demos/} contém uma
coleção de animações utilizando a tecnologia.

%}}}
%}}}
\section{NAVEGADORES}
%{{{
Navegadores são aplicações, onde as tecnologias da OWP são
interpretadas e geram um conteúdo útil para os usuários. São
os clientes em uma arquitetura cliente servidor. O servidor desta
arquitetura geralmente é um servidor WEB cujo objetivo principal é
fornecer páginas HTML para o navegador processar. A comunicação entre
o navegador e o servidor WEB se dá através da troca de mensagens no
protocolo HTTP.

Nos navegadores os usuários necessitam saber o endereço de determinado
servidor, ou utilizar buscadores para auxiliá-los. Este é um processo
árduo para as plataformas móveis pois necessitam maior interação dos
usuários, e não são “naturais” se comparado ao modo de consumir
aplicativos nestas mesmas plataformas. Simplesmente adquirindo
o aplicativo na loja e abrindo-o no sistema operacional. Algumas
formas de contornar este problema serão descritos nas seção de
Disponibilização da Aplicação.

Uma vez localizado o endereço o navegador manda uma mensagem em HTTP
requisitando o conteúdo de determinado endereço. O servidor responde a
mensagem HTTP com um documento HTML e o cliente, ao receber, começa o
processo de renderização.

O processo de renderização é complexo e a grande maioria dos
navegadores confia em bibliotecas especializadas para efetuar este
trabalho os motores de renderização.

Alguns motores de renderização incluem:

\begin{itemize}
    \item Blink: Utilizado no Opera, Google Chrome e projetos relacionados;
    \item Gecko: Utilizado nos produtos da Mozilla;
    \item KHTML: Utilizado no navegador Konkeror, esta serviu de base para o Blink;
    \item WebKit: Utilizado no Safari e versões antigas do Google Chrome;
\end{itemize}

A renderização consiste na decodificação de um documento em HTML
para sua representação memória e posterior pintura no espaço de tela
do navegador. Interpretar os documentos é processo árduo e alguns
motores dependem de bibliotecas externas para fazê-lo. 

Para interpretar HTML o motor WebKit utiliza a biblioteca Bison, já
o Gecko utiliza uma biblioteca própria \autocite{howBrowsersWork}.
Durante o processo de renderização o navegador pode requisitar outros
arquivos do servidor a fim de completar a experiência desejada para o
documento em questão.

Geralmente após a renderização do documento vem a execução de scripts.
As bibliotecas que executam JavaScript são chamadas de motores de JavaScript.
Abaixo segue uma lista dos motores de JavaScript mais comuns.

\begin{itemize}
    \item SpiderMonkey: Primeiro motor, desenvolvido por Brendan Eich, escrito em C++
    \item Rhino: Criada pela Netscape, escrito em Java
    \item Nitro: Criada pela Apple
    \item V8: Criada pelo Google
    \item TraceMonkey: Criada pela Mozilla
\end{itemize}

Cada um dos motores que compõem um navegador implementam partes da
especificação do HTML. E, operando em conjunto, tentam comportar
todas as tecnologias da WEB. Infelizmente a forma que as tecnologias
são suportadas varia e algumas não estão presentes de qualquer
forma nos navegadores. Não obstante o suporte vem crescendo. A figura
\ref{fig:audioCodecs} apresenta o gráfico de suporte por versões de
navegadores em dezembro de 2015.

\begin{figure}[H]
    \centering
    \includegraphics[width=0.8\textwidth,natwidth=610,natheight=642]{htmlSupport.png}
	\caption{Suporte das especificações do HTML nos navegadores}
    \label{fig:htmlSupport}
    \source{https://html5test.com/}
\end{figure}

%}}}
\section{ANDROID}
%{{{

É um sistema operacional open-source criado em 2003 pela Android
Inc e mantido pelo Google desde 2005. Android funciona em uma
variedade de dispositivos desde celulares a tablets, netbooks a
computadores desktop, mas seu foco é dispositivos com tela sensível
\autocite{chromeVsAndroid}. O sistema operacional é composto por
diversos projetos open-source, sendo o mais proeminente deles o kernel
Linux, utilizado como fundamento do sistema operacional. Além
da versão open-source (AOSP), existe a versão do Google que utiliza
ferramentas proprietárias  para adicionar funcionalidades aos dispositivos.

Softwares para Android são geralmente escritos em Java e executados
através da máquina virtual Dalvik. Dalvik é similar a máquina
virtual Java, mas roda um formato de arquivo diferenciado (.dex),
otimizados para consumir pouca memória, que são agrupados em um único
pacote (.apk). 

Aplicativos da WEB podem ser integrados no Android através de uma
arquitetura híbrida. Os arquivos da aplicação são empacotados dentro
de um apk utilizando um componente nativo do Android, a API WebView, para
executar as tecnologias da OWP.

Além da arquitetura híbrida, é possível executar jogos WEB em
dispositivos Android através dos navegadores presentes nestes
aparelhos. As novas versões do Android contam com o Google Chrome como navegador
padrão. Já a versões antigas contém um navegador próprio
que utiliza o motor de renderização LibWebCore, baseado no WebKit
\autocite{comparisonPlatforms}. Além do padrão outros navegadores como
o Firefox ou Opera também podem ser instalados.

Além do Android o sistema IOS é de grande relevância mercadológica
para o desenvolvimento de jogos. Não obstante, não será tratado neste
trabalho pelos motivos supracitados. A próxima seção deste trabalho
descreve como detectar recursos nas variadas plataformas que a WEB se
apresenta.

%}}}
\section{DETECÇÃO DE RECURSOS}
%{{{
Visto que nenhum navegador implementa as especificações HTML
completamente, cabe ao desenvolvedor detectar os navegadores que não
comportam as necessidades tecnológicas dos aplicativos que cria. Ao
deparar-se com uma funcionalidade faltante o desenvolvedor tem duas
possibilidades: notificar o usuário sobre o problema ou utilizar
polyfills.

Polyfills são recursos que simulam uma funcionalidade não
disponível nativamente nos navegadores. A biblioteca Gears
\url{https://developers.google.com/gears} é um exemplo. Gears
serve para prover recursos de Geolocalização para navegadores que
não implementam a especificação do HTML5.

Essa capacidade de suportar tecnologias que não estão ainda
disponíveis (ou nunca estarão no caso de dispositivos legados)
através de polyfills é uma das características que faz a WEB uma
plataforma de tão grande abrangência. Novas tecnologias são criadas a
todo o momento; entretanto, o suporte a essas funcionalidades geralmente
não acompanham o passo das inovações. E ainda assim os usuários
podem se beneficiar de uma taxa substancial delas através de polyfills.

Algumas funcionalidades do HTML, como geolocalização e vídeo
foram primeiramente disponibilizadas através de plugins. Outras
funcionalidades, como o canvas, podem ser totalmente emuladas via
polyfills em JavaScript \autocite{diveIntohtml}.

Detectar suporte aos variados recursos do HTML5 no navegador
pode ser uma tarefa entediante. É possível implementar testes para
cada funcionalidade utilizada abordando os detalhes de implementação
de cada uma ou então fazer uso de alguma biblioteca especializada
neste processo. O Modernizr é uma opção open-source deste tipo de
biblioteca, este gera uma lista de booleanos sobre grande variedade dos
recursos HTML5, dentre estes, geolocalização, canvas, áudio, vídeo e
armazenamento local.

A quantidade de especificações que um aplicativo complexo como um jogo
utiliza pode ser bem grande, e muitas vezes é difícil dizer qual quais
navegadores implementam o quê. Uma boa referência do suporte a recursos
nos navegadores é o site \url{http://caniuse.com/}.

%}}}
\section{RENDERIZAÇÃO}
Renderização é parte fundamental de muitos jogos. As tecnologias que
permitem renderização na WEB serão descritas abaixo.
\subsection{SVG}
%{{{
SVG (\textit{Gráficos de vetores escaláveis}), é uma linguagem
baseada em XML especializada na criação de vetores bidimensionais
\autocite{html5mostwanted}. Segundo \cite[p. 4]{svgTime} svg foi criada
em conjunto por empresas como: Adobe, Apple, AutoDesk, entre outas,
sendo que seus produtos contam com rápida integração a tecnologia.

Por descrever imagens utilizando vetores ao invés de mapas de bits
os tamanhos dos arquivos em SVG são geralmente pequenos e podem
ser comprimidos com grande eficiência. Talvez a característica mais
marcante do SVG é que não há diferença de qualidade em resoluções
visto que os vetores são escaláveis. Sendo que pequenos arquivos
servem igualmente bem um monitor com baixa resolução como um monitor
retina.

Por ser baseado em XML, uma das tecnologias da WEB, SVG permite a
utilização da API do DOM para manipular seus elementos. Tornando
simples a integração com outras tecnologias da WEB. Pode-se utilizar
arquivos CSS para customizar a apresentação, JavaScript para adicionar
interatividade, etc.

Além de grande integração com as demais tecnologias, SVG conta com
uma API nativa poderosa. Os elementos geométricos do SVG incluem
retângulos, círculos, elipses, linhas e polígonos \autocite[p.
5]{svgTime}. Também existe a possibilidade de declarar caminhos
customizados através do elemento \textit{path}, algo similar com o
\textit{Path2D} do Canvas.

E cada um dos elementos, ou agrupamento de elementos podem ser
transformados; traduzidos, redimensionados, rotacionados e distorcidos
\autocite[p. 5]{svgTime}.

Para ilustrar a utilização, a figura \ref{fig:svgCircle} demonstra um
círculo sendo definido em SVG.

\begin{figure}[H]
\centering
\begin{verbatim}

<svg width="100" height="100">
  <circle 
    cx="50" 
    cy="50" 
    r="40" 
    stroke="green" 
    stroke-width="4" 
    fill="yellow" 
  />
</svg>

\end{verbatim}
\caption{Círculo em SVG.}
\source{http://www.w3schools.com/svg/}
\label{img:svgCircle}
\end{figure}

Outra tecnologia popular da WEB para renderização que adota uma filosofia totalmente 
diferente do SVG é o canvas.
%}}}
\subsection{CANVAS}
%%{{{
O elemento \textit{canvas} define uma camada de mapa de bits em
documentos HTML que pode ser usada para criar diagramas, gráficos e
animações 2D. Foi criado pela Apple em 2004 para renderizar elementos
de interface no Webkit, logo foi adotado por outros navegadores e se
tornou um padrão da OWP.

Em um documento HTML, canvas é um retângulo onde pode-se usar
JavaScript para desenhar \autocite[p. 113]{diveIntohtml}. Mais
especificamente, o retângulo do canvas é um espaço vetorial cuja
origem se da na esquerda superior. Normalmente cada unidade do plano
cartesiano corresponde a um pixel no canvas \autocite{mdnCanvas}.

Manipular o retângulo é uma analogia de como desenhar manualmente,
move-se o "lápis" para o local desejado e traça-se os pontos onde
a linha (caminho) deve percorrer. Além da possibilidade de desenhar
linhas livremente também é possível criar retângulos nativamente.
Todas as demais figuras geométricas precisam ser feitas através da
junção de caminhos \autocite{mdnCanvas}. Para desenhar caminhos
curvos, de modo a criar círculos e elipses, existem funções especiais
de arco.

Escrever no canvas envolve a manipulação de diversos caminhos e
retângulos, em jogos que fazem extensivo uso do canvas a complexidade
de manipulação de linhas pode crescer muito. As últimas versões do
Canvas introduziram o objeto Path2D que possibilita o armazenamento e
composição de instruções de caminhos a fim de possibilitar o reuso
de formas. Ao invés de utilizar os métodos de caminhos diretamente no
contexto do canvas utiliza-se uma instancia do objeto \textit{Path2D}.
Todos os métodos relacionados aos caminhos como o \textit{moveTo},
\textit{arc} ou \textit{quadraticCurveTo} estão disponíveis no
objeto Path2D \autocite{mdnCanvas}. Também é possível utilizar a
notação do SVG na criação de uma instancia de Path2D possibilitando
a reutilização de conteúdo para ambas as tecnologias.

Além da possibilidade de desenhar programaticamente é possível
carregar gráficos. Muitos dos jogos HTML5 utilizam sprites ou padrões
recortáveis \textit{tiled}, bastante similar aos títulos antigos da
SNES e Game Boy \autocite{buildingHtml5Game}.

Apesar da API do canvas ser poderosa não é possível manipular
diretamente as camadas já desenhadas. Alternativamente pode-se limpar
o canvas inteiramente em pontos determinados ou então sobrescrever as
partes que se deseja alterar.

\begin{figure}[H]
\centering
\begin{verbatim}

var c = document.getElementById("myCanvas");
var ctx = c.getContext("2d");
ctx.fillStyle = "#FF0000";
ctx.fillRect(0,0,150,75);

\end{verbatim}
\caption{Canvas}
\source{http://www.w3schools.com/html/html5\_canvas.asp}
\label{img:retangleOnCanvas}
\end{figure}

A figura \ref{img:retangleOnCanvas} demonstra a utilização do canvas 2d
para a criação de um retângulo. Note que o objeto canvas tem que carregar um
contexto que então é utilizado como API para manipular a mapa de bits em 2D.

O canvas até aqui descrito trata-se de sua forma, ou contexto 2D. A
especificação 3D do canvas é o WebGl.

%}}}
\subsection{WEBGL}
%{{{

WebGL é uma API JavaScript otimizada desenhar gráficos em três
dimensões. Ideal para a criação de ambientes virtuais, jogos e
simulações. Por ser uma tecnologia da OWP WebGL foi especificado
para funcionar nativamente nos navegadores sem a ajuda de plugins ou
ferramentas de terceiros.

WebGL foi desenvolvido baseando-se na especificação OpenGL a
qual trata de definir como renderizar gráficos multiplataforma.
Especificamente OpenGL ES, uma versão do OpenGL otimizada para
dispositivos móveis. O órgão que especifica o WebGL é o mesmo que
especifica o OpenGL, o grupo sem fins lucrativos Kronos. Os primeiros
rascunhos do WebGL iniciaram em 2006, não obstante o grupo de trabalho
não foi formado até 2009 e a primeira versão do foi lançada em 2011.

Apesar de ter sido desenvolvido com foco em 3D, WebGL pode ser
igualmente utilizado para criação de gráficos em duas
dimensões\autocite[p. 6]{3daps}. O elemento do DOM que provê a
interface do WebGL é o canvas, no contexto 3D. Essa integração com
o DOM via tag canvas permite que o WebGL seja manipulado assim como os
demais elementos HTML.

Especificação é composta por uma API de controle em JavaScript
e o processamento shaders do lado da GPU (Central de processamento
gráfico). 

Shaders são scripts que definem níveis de cor ou efeitos especiais
sobre um modelo 2D ou 3D. Contam com grande performance, possibilitando
conteúdo em tempo real como no caso de jogos. São utilizados no
cinema, em imagens geradas por computadores e vídeo games.

Existem dois shaders principais, de vértices e fragmentos. Shaders de
vértices são chamados para cada vértice sendo desenhado definindo
suas posições definitivas. Já shaders de fragmentos atuam na cor de
cada pixel a ser desenhado \autocite[p.15]{3daps}.

\citet{html5mostwanted} cita que conforme a habilidade do desenvolvedor
aumenta, mover funções antes delegadas ao JavaScript para os shaders
pode aumentar a performance e oferecer uma ampla coleção de efeitos e
realismo.

Um site interessante para explorar exemplos WebGL avançados é o blog
 \url{http://learningwebgl.com} que conta com tutoriais cobrindo áreas
como diferentes tipos de iluminação, carregamento de modelos em JSON,
gerenciando eventos do mouse e teclado; e como renderizar uma cena WebGL
em uma textura \autocite[p.42]{3daps}\footnote{Os apêndices contam.
com uma coleção de bibliotecas que facilitam a utilização de WebGL}.

Apesar da relevância, WebGL não foi utilizado no protótipo pois
ainda não está completamente suportado em navegadores populares como
o Firefox e a grande curva de aprendizado do WebGL puro é muito grande
para se encaixar no escopo deste projeto .

Uma tecnologia que se integra profundamente como ambientes virtuais
em três dimensões criados via OpenGL é o WebVR.
%}}}
\subsection{WEBVR}
%{{{ 
Segundo \citet{virtualReality} realidade virtual é uma experiência em
que o usuário é efetivamente imerso em um mundo virtual responsivo.
Realidade virtual é uma área nem tão nova mas que recebeu interesse
renovado recentemente. Isso se dá, pelo menos me parte, pela
massificação dos dispositivos móveis inteligentes. O hardware
necessário para fornecer uma experiência minimamente viável como
acelerômetros, câmeras e telas de alta resolução está disponível
em praticamente todos os dispositivos comercializados.

Realidade virtual é uma área de grande interessa para os produtores
de jogos, pois pode oferecer alto nível de imersão nos já
interativos e desafiadores ambientes dos jogos.

A WebVR é uma especificação que pretende trazer os benefícios
da realidade virtual para dentro do mundo da WEB. Em termos simples
a especificação define uma forma de traduzir movimentos de
acelerômetros e outros sensores de posição e movimentos para dentro
do contexto de uma um contexto 3D através de JavaScript.

Atualmente a especificação do WebVR se encontra em fase de rascunho e
as últimas versões do Firefox, e versões compiladas manualmente do
Google Chrome já permitem a utilização.
%}}}
\section{WebCL}
%{{{
É uma API em JavaScript para o recursos de OpenCL que permitem
computação paralela com grandes ganhos de performance. Aplicativos
como motores de física e renderizadores de imagens, ambos relevantes
para os jogos, podem se beneficiar grandemente de processamento feito
em paralelo, possivelmente na GPU. OpenCL é um framework para escrever
programas que funcionem em plataformas com diversas unidades de
processamento, assim como WebGL e WebCL é especificada e desenvolvida
pelo grupo Kronos.

A primeira versão da especificação foi no início de 2014 mas até
então nenhum navegador implementa o definido.
%}}}

Após a revisão de tecnologias de renderização serão abordadas
tecnologias de multimídia: áudio e vídeo. Mas para falar de ambos,
antes é necessário falar de codecs.

\section{CODECS}
%{{{

Codec é o algoritmo usado para codificar e decodificar vídeo ou
áudio em um conjunto de bits \autocite{diveIntohtml}. O termo Codec é
um acrônimo, significando o processo de codificar (\textit{coder}) um fluxo de dados
para armazenamento e decodificá-lo (\textit{decoder}) para ser consumido.

Dados multimídia são geralmente enormes, sem serem codificados,
um vídeo ou áudio consistiriam em uma vasta quantidade de dados
que seriam muito grandes para serem transmitidos pela Internet em um
período de tempo razoável \autocite[p. 66]{proHtml5}. O objetivo dos
codecs é diminuir o tamanho dos arquivos com a menor perda de qualidade
possível. Para isso os codecs utilizam de várias estratégias de
compressão ou descarte de dados; podendo rodar tanto em hardware quanto
em software.

Existem codecs desenvolvidos especificamente para a Web. Buscam
uma razão de tamanho e qualidade aceitável, mas prezando por
tamanho. Uma das otimizações realizadas por codecs de vídeo na
Web, é a não troca de todo conteúdo de um quadro para o próximo,
possibilitando maiores taxas de compressão, que resulta em arquivos
menores \autocite{diveIntohtml}.

O funcionamento de codecs pode variar muito, conforme as estratégias
de compressão utilizadas, a quantidade de bits por segundo suportada,
entre outros fatores. Visto que os algoritmos de compressão podem
adquirir grande complexidade muitos codecs são encobertos por licenças
que limitam sua utilização. Não obstante, também existem opções
livres de patentes e licenças.

Abaixo segue uma lista de alguns codecs populares para áudio segundo
\cite[p. 67]{proHtml5}.

\begin{itemize}
    \item ACC
    \item MPEG-3
    \item Vorbis
\end{itemize}

Ainda segundo \cite[p. 67]{proHtml5} abaixo segue uma lista de codecs populares para vídeo.
\begin{itemize}
    \item H.264
    \item VP8
    \item OggTheora
\end{itemize}

Após um fluxo de dados multimídia ter sido codificado através do
algoritmo de codec ele é armazenado em um contêiner. Contêiners
são um padrão de metadados sobre as informações codificadas
de modo a possibilitar que outros programas consigam interpretar
estas informações de forma padronizada. Como um arquivo
\textit{zip}, contêiners podem conter qualquer coisa dentro de si
\autocite{diveIntohtml}. Assim como codecs existem contêiners livres e
com restrições de licença.

Abaixo segue uma lista de alguns contêiners de áudio.
\begin{itemize}
    \item Audio Video Interleave (.avi)
    \item MPEG-2 Audio Layer III (.mp3)
    \item Matroska (.mkv)
    \item Vorbis (.ogg)
    \item Opus (.opus)
\end{itemize}

Abaixo segue uma lista de alguns contêiners de vídeo.
\begin{itemize}
    \item Audio Video Interleave (.avi)
    \item Flash Video (.flv)
    \item MPEG4 (.mp4)
    \item Matroska (.mkv)
    \item Ogg (.ogv)
    \item WebM (.webm)
\end{itemize}

O suporte a codecs e contêiners na WEB varia de navegador para
navegador, de acordo com as preferências mercadológicas, técnicas
ou filosofias das empresas por trás dos navegadores. Segundo
\citet{diveIntohtml} não existe uma única combinação de contêiner
e codecs que funcionem em todos os navegadores \footnote{A figura
\ref{fig:audioCodecs} apresenta um comparativo interessante sobre
os formatos populares de áudio considerando o fator taxa de bits
(quantidade de informação armazenável por segundo) versus qualidade
(perceptível por humanos).}.

%}}}
\section{ÁUDIO}
%{{{

Áudio é um componente vital para oferecer imersão e feedback aos
usuários de jogos. O componente de áudio é especialmente útil para
jogos de ação \autocite{browserGamesTechnologyAndFuture}. Efeitos
de som e música podem servir como parte da mecânica dos jogos.

Antes do HTML5 não havia como consumir áudio na WEB sem a utilização de
plugins de terceiros. A especificação do HTML define duas formas de
utilizar utilizar áudio na WEB: através do elemento HTML áudio ou
através da API JavaScript de áudio.

\begin{figure}[H]
    \centering
    \includegraphics[width=0.8\textwidth,natwidth=610,natheight=642]{codec.png}
	\caption{Comparação de codecs de áudio}
    \label{fig:audioCodecs}
    \source{https://www.opus-codec.org/comparison/}
\end{figure}

\subsection{ELEMENTO ÁUDIO}

O elemento \textit{audio} foi a primeira tecnologia de áudio nativa
para WEB, ele define um som dentro de um documento HTML. Quando o
elemento é renderizado pelos navegadores, ele carrega o conteúdo que
pode ser reproduzido pelo programa dentro do navegador.

\begin{figure}[H]
\centering
\begin{verbatim}
<audio controls>
<source src="horse.ogg" type="audio/ogg">
<source src="horse.mp3" type="audio/mpeg">
Your browser does not support the audio element.
</audio>
\end{verbatim}
\caption{Exemplo de utilização da tag áudio}
\label{fig:htmlAudio}
\source{http://www.w3schools.com/HTML/HTML5\_audio.asp}
\end{figure}

A imagem \ref{fig:htmlAudio} demonstra a utilização da tag
\textit{audio}. Os elementos \textit{source} demonstrados na figura
referenciam arquivos de áudio contendo um par de contêiner
e codec. Mais de um \textit{source } é necessário pois os
criadores de navegadores não chegaram a um consenso sobre qual
formato deve ser usado, sendo necessário utilizar vários para
suportar todos os navegadores populares \footnote{Com o site
\url{http://hpr.dogphilosophy.net/test/} é possível detectar os
formatos de codecs de áudio suportados pelo navegador sendo utilizado}.

A especificação declara que todo o conteúdo dentro de uma tag
\textit{audio}, que não sejam elementos \textit{source}, sejam ignoradas
pelo navegador. O que permite que seja adicionada marcação de reserva
para tratar os casos de quando não existe suporte a tag \textit{audio}
no navegador. Visto que os navegadores que não suportam áudio vão buscar
renderizar o conteúdo dentro da tag.

A figura \ref{fig:htmlAudio} ilustra este comportamento através da
mensagem \textit{Your browser does not support the audio element } que
só será apresentada se o navegador do usuário não tiver suporte a
tag \textit{audio}.

O objetivo inicial da tag \textit{audio} é reproduzir um som e parar.
Ideal para ouvir música, como um som de fundo. Por conseguinte, a tag
\textit{audio} não é o suficiente para comportar aplicações de
áudio complexas \autocite{audioApiSpec}. A grande maioria de jogos
muitas vezes precisam lançar múltiplos sons derivados de ações
de usuário e outros eventos, nestes casos a API de áudio é mais
adequada.

\subsection{API DE ÁUDIO}

É uma interface experimental (ainda em rascunho) em JavaScript para
criar e processar áudio. O objetivo da especificação é incluir
capacidades encontradas em motores de jogos modernos e também permitir
o processamento, mistura e filtragem, funcionalidades que estão
presentes nas aplicações de processamento de áudio modernas para
desktop \autocite{audioApiSpec}.

A API especificada provê uma interface para manipular nodos de
áudio que podem ser conectados permitindo refinado controle sobre os
efeitos sonoros. O processamento se dará primeiramente em uma cada
inferior (tipicamente código Assembly / C / C++), mas síntese e
processamento em JavaScript também será suportado \autocite{audioApiSpec}.

Essa tecnologia é muito mais nova do que o elemento \textit{audio}.
Diferentemente dos demais navegadores o Internet Explorer não
dá nenhum nível de suporte a API. O polyfill AudioContext
suporta as partes básicas da API e pode ser utilizada nos
casos onde não existe suporte para a API do HTML \footnote{O
polyfill AudioContext pode ser encontrado no seguinte endereço
\url{https://github.com/shinnn/AudioContext-Polyfill}}.

As últimas versões da especificação da Audio API contam com a
possibilidade de manipular a API de áudio através de WEB Workers, o
que traz oportunidades interessantes para aplicações que dependam de
muito processamento de áudio, visto que o processamento pode ser
feito em uma thread separada.

Além de grande flexibilidade com áudio alguns jogos requerem a
disponibilidade de vídeo para utilizar como introdução, cinemáticas,
entre outros recursos que habilitam uma experiência mais rica ao
usuário. Abaixo será feita uma revisão sobre a tecnologia de vídeo
em HTML.

%}}}
\section{VÍDEO}
%{{{

O elemento \textit{video} define uma forma de adicionar vídeos na
WEB nativamente, sem a necessidade de utilizar plugins de terceiros
como o Flash Player. Assim como com o elemento \textit{audio} pode-se
adicionar um arquivo através do atributo \textit{src} do elemento
ou adicionar vários formatos de contêiner e codec dentro da tag
através de elementos \textit{source}. O navegador decidirá em tempo de
execução qual formato executar dependendo de suas capacidades.

\begin{figure}[H]
\centering
\begin{verbatim}
<video controls style="width:640px;height:360px;" poster="poster.png">
  <source src="devstories.webm" 
          type='video/webm;codecs="vp8, vorbis"' />
  <source src="devstories.mp4" 
          type='video/mp4;codecs="avc1.42E01E, mp4a.40.2"' />
  <track src="devstories-en.vtt" label="English subtitles" 
         kind="subtitles" srclang="en" default></track>
</video>
\end{verbatim}
\caption{Exemplo de utilização de vídeo}
\source{http://www.html5rocks.com/en/tutorials/video/basics/}
\label{fig:video}
\end{figure}

A figura \ref{fig:video} demonstra a utilização de algumas
funcionalidades do elemento \textit{video}. Como demonstrado na figura,
além do elemento \textit{source}, a tag \textit{video} suporta o
elemento \textit{track}. O qual permite informar subtítulos para os
vídeos sendo apresentados. Também é possível habilitar controles de
vídeo nativos dos navegadores, como demonstrado através do atributo
\textit{controls}.

Como o elemento vídeo se encontra no HTML é possível manipulá-lo com
uma gama de tecnologias. Com CSS é possível aplicar escala de cinza do
sobre o elemento vídeo gerando um efeito preto e branco interessante.
A especificação do elemento \textit{video} também permite controlar
quais partes do vídeo serão mostradas através de parâmetros de tempo
passados como argumentos junto ao nome do arquivo. Ou capturar 
frames de vídeo dentro do elemento canvas.

Além de ser flexível nas tecnologias de multimídia jogos um
requerimento comum em jogos é haver uma forma de armazenar dados
eficientemente e buscá-los com agilidade. Abaixo serão discutidas as
tecnologias de armazenamento disponíveis para a WEB.
%}}}
\section{ARMAZENAMENTO}
%{{{
Uma das grades limitações do HTML era a ausência de capacidade de
armazenamento de dados no lado do cliente. Antes do HTML5 a única
alternativa era usar cookies, os quais tem um armazenamento de no
máximo 4k e trafegam em toda a requisição, tornando o processo lento.
Essa área era ode as aplicações nativas detinham grande vantagem
sobre as aplicações web. O HTML5 solucionou este problema introduzindo
várias formas de armazenamento de dados \autocite{html5Tradeoffs}.

Armazenamento local é um recurso importante para jogos, tanto por
diminuir a latência da persistência na rede, quanto para possibilitar
um experiência offline.

Existem algumas especificações sobre armazenamento, mas a grande
parte delas não conta como suporte completo em todos os navegadores
comuns, um polyfill interessante para Web Storage  e IndexedDB é o
projeto localForge \textit{https://github.com/mozilla/localForage} da
Mozilla.

\subsection{WEB SQL}

A especificação Web SQL introduz uma API para manipular banco de dados
relacionais em SQL. A especificação suporta transações, operações
assíncronas e um tamanho de armazenamento substancial: 5 megabytes, o
qual pode ser estendido pelo usuário.

O grupo de trabalho do Web SQL iniciou-se em 2010 e foi suspendido ainda
como rascunho. Apesar de ser um recurso desejável para muitos
desenvolvedores, foi descontinuada pelos motivos descritos abaixo.

Segundo \citet{diveIntohtml}
\begin{quote}
Todos os implementadores interessados em Web SQL utilizaram a mesma
tecnologia (Sqlite), mas para a padronização ficar completa é
necessário múltiplas implementações. Até outro implementador se
interessar em desenvolver a especificação a descrição do dialeto SQL
apenas referencia o SQLITE, o que não é aceitável para um padrão.
\end{quote}

Não obstante, a especificação ainda é suportada pelo Google
Chrome, Safari, Opera e Android, entre outros. Mas até que outros
implementadores se prontifiquem a especificação continuará suspensa.
No lugar do Web SQL a W3C recomenda a utilização do Web Storage e do
IndexedDB.

\subsection{WEB STORAGE}

Web Storage, também conhecido como Local Storage, provê uma forma de
armazenar dados no formato chave valor dentro do navegador. Os dados são
persistidos mesmo que o usuário feche a página ou o navegador.

Web Storage é um recurso similar a cookies, contudo algumas diferenças
substanciais são perceptíveis. Web Storage não requer que os dados
sejam trafegados como cabeçalhos nas requisições. Também provê
maiores espaços de armazenamento quando comparado a cookies.

A tecnologia começou como parte da especificação do HTML5 mas agora
conta com um documento próprio mantido pela W3C. A especificação é
suportada pela grande maioria dos navegadores populares.

A especificação oferece duas áreas de armazenamento, o armazenamento
local e de sessão. O armazenamento local é persistido por domínio
e outros scripts provindos deste mesmo domínio poderão fazer uso da
informação. O armazenamento de sessão é para informações que podem
variar de aba para aba e que não é interessante que sejam persistidos
para demais acessos além do atual.

A API do Web Storage é simples, consistindo em uma interface para
buscar dados e outra para armazenar, no formato chave/valor.

\begin{figure}[H]
\centering
\begin{verbatim}
// Store value on browser for duration of the session
sessionStorage.setItem('key', 'value');

// Retrieve value (gets deleted when browser is closed and re-opened)
alert(sessionStorage.getItem('key'));

// Store value on the browser beyond the duration of the session
localStorage.setItem('key', 'value');

// Retrieve value (persists even after closing and re-opening the browser)
alert(localStorage.getItem('key'));

\end{verbatim}
\caption{Web Storage na prática}
\label{fig:WebStorage}
\source{https://en.wikipedia.org/wiki/Web\_storage\#usage}
\end{figure}

A figura \ref{fig:WebStorage} exemplifica a utilização do Web
Storage, para utilizar o armazenamento de sessão utiliza-se o objeto
\textit{sessionStorage}. Já para utilizar o armazenamento local utiliza-se o
objeto \textit{localStorage}.

Web Storage é uma solução simples que comporta muitos casos de uso.
Não obstante muitas vezes é necessário um controle mais refinado
sobre os dados, ou mais performance em uma base de dados massiva. Para
responder a estes desafios existe a especificação do IndexedDB.

\subsection{IndexedDB}
%{{{
IndexedDB é um banco de dados que suporta o armazenamento de grandes
quantidades de dados no formato de chave/valor o qual  permite alta
performance em buscas baseadas em índices. A tecnologia é uma recomendação
da W3C desde janeiro de 2015 e suportada, pelo menos parcialmente, por
praticamente todos os navegadores populares.

Inicialmente IndexedDB permitia operações síncronas e assíncronas.
Não obstante, a versão síncrona foi removida devido a falta de
interesse da comunidade. Operações assíncronas permitem que
aplicativos JavaScript nunca esperam pelo resultado para continuar a
execução. Outrossim, cada interação com o banco de dados é uma
transação que pode retornar um resultado ou um erro. Os eventos da
transação são internamente eventos DOM cuja propriedade \textit{type}
do elemento foi setada para \textit{success} ou \textit{error}.

Ao invés de tabelas, IndexedDB trabalha com repositórios de objetos.
Cada entrada, tupla em SQL, de um determinado repositório pode ser de
um formato diferenciado, com exceção da chave única que deve estar
presente em cada uma das entradas.

\begin{figure}[H]
\centering
\begin{verbatim}
	var db;
	var request = window.indexedDB.open("Mydb", 9);
	request.onsuccess = function(event) {
		db = event.target.result;
		var transaction = db.transaction(["customers"], "readwrite");
		var objectStore = transaction.objectStore("customers");
		var request = objectStore.add({email: "mymail@domain.com", name: "foo"});
		request.onsuccess = function(event) {
			console.log('customer added')
		};
	}
\end{verbatim}
\caption{Adicionando um cliente em IndexedDB.}
\label{fig:IndexedDB}
\end{figure}

A figura \ref{fig:IndexedDB} demonstra um exemplo simplificado da
utilização do IndexedDB, como cada iteração com o banco de dados é
construído através de uma nova requisição e o tratamento do resultado
é dado dentro de eventos.

Apesar de ser desenvolvido com objetivo de ser uma solução para todas
as necessidades de armazenamento no Frontend IndexedDB ainda sofre
algumas limitações.

Abaixo segue uma lista com algumas das limitações do IndexedDB.

\begin{itemize}
\item Tem limites de armazenamento e as regras variam de navegador para navegador.
\item O comportamento em abas anônimas não está especificado e os resultados também variam.
\item Existe uma pequena probabilidade de os dados se perderem, no caso do Firefox a API não espera confirmação do sistema operacional para considerar um dado válido, essa foi uma escolha em detrimento de performance.
\item Não existe a possibilidade de fazer buscas em textos como o \textit{LIKE} do SQL.
\item o usuário pode configurar o navegador para não aceitar armazenamento local para determinado domínio.
\end{itemize}

%}}}

A característica assíncrona do IndexedDB, é fundamentada na
premissa de não perturbar o fluxo principal da aplicação enquanto
processamento não vital, e possivelmente demorado, ocorre. Outra
tecnologia da web que utiliza os mesmos princípios é o Web Workers.

%}}}
\section{WEB WORKERS}
%{{{

É uma API que possibilita executar vários scripts
(\textit{threads}) JavaScript ao mesmo tempo. O script que cria uma
thread é chamado de pai da thread, e a comunicação entre pai e filhos
pode acontecer de ambos os lados através de mensagem encapsuladas
em eventos. Um script que não seja pai de uma thread não pode se
comunicar com ela, a não ser que a thread seja em modo compartilhado.

O contexto global (objeto \textit{window}) não existe em uma
thread, no seu lugar o objeto \textit{DedicatedWorkerGlobalScope}
pode ser utilizado. Workers compartilhados podem utilizar o
\textit{SharedWorkerGlobalScope}. Estes objetos contém grande parte das
funcionalidades proporcionadas pelo window com algumas exceções, por
exemplo threads não podem fazer alterações no DOM.

%}}}
\section{OFFLINE}
%{{{
Disponibilizar aplicações WEB offline é uma característica
introduzida no HTML5. Uma nova gama de aplicativos WEB foram
possibilitados devido as tecnologias offline. A importância de gestão
offline é tanta em alguns nichos que os avaliadores do mercado
de software do IOs consideram quase uma obrigação da gestão da
aplicação offline \autocite{publishHtml5}.

Jogos de usuário único podem se beneficiar enormemente de aplicativos
offline, tornando possível utilizar a aplicação com ou sem a
presença de rede. Para tanto é necessário poder armazenar dados
locais, tecnologias como IndexedDB e Web Storage permitem isso. O outro
requerimento para estar offline é uma forma de armazenar os arquivos
da WEB localmente de forma que sejam utilizados quando não houver uma
conexão a rede.

HTML5 conta com uma especificação estável de APIs de cache offline
mantida pela W3C. Esta especificação determina que uma arquivo de
manifesto contenha quais arquivos serão guardados para utilização
offline e possivelmente quais serão usados pela rede.

A figura \ref{fig:offline} exemplifica um arquivo de manifesto. Os
arquivos abaixo da palavra \textit{CACHE MANIFEST} serão armazenados
em cache e não serão buscados na rede a não ser que o arquivo
de manifesto seja modificado. Já os arquivos abaixo da palavra
\textit{NETWORK:} serão utilizados exclusivamente com rede e serão
buscados todas as vezes. Ainda exite a palavra chave \textit{FALLBACK}
onde todos os itens abaixo dela serão utilizados para substituir
arquivos de rede.

É uma boa prática colocar um comentário com a versão do arquivo de
manifestos. Desse modo quando um arquivo for modificado incrementa-se a
versão do arquivo de manifestos e as modificações serão baixadas nos
navegadores clientes.

\begin{figure}[H]
\centering
\begin{verbatim}
CACHE MANIFEST
index.html
help.html
style/default.css
images/logo.png
images/backgound.png

NETWORK:
server.cgi
\end{verbatim}
\caption{Exemplo de arquivo de manifesto offline}
\source{http://www.w3.org/TR/offline-webapps/}
\label{fig:offline}
\end{figure}

%}}}
\section{ENTRADA DE COMANDOS}
%{{{
Na construção da grande maioria dos jogos é muitas vezes
imprescindível grande flexibilidade na gestão de entrada comandos.
Esta necessidade amplia na criação de jogos multiplataforma, em
determinadas plataformas a entrada de comandos pode-se dar través de
teclado, em dispositivos móveis através tela sensível ou sensor de
movimentos.

O HTML5 trata todos estes casos abstratamente na forma de eventos, os
quais podem ser escutados através de \textit{listeners}. JavaScript
pode ser configurado para escutar cada vez que um evento ocorre seja um
clique do mouse o pressionar de uma tecla ou o mover de um eixo em um
joystick.

Quando um evento de interação é disparado, um \textit{listener}
que esteja ouvindo a este evento pode invocar uma função e realizar
qualquer controle desejado \autocite{buildingHtml5Game}.

O teclado é um periférico comum no caso de jogos da WEB, para existem
os eventos \textit{keyup} e \textit{keydown} que representam uma tecla
sendo solta e pressionada respectivamente. A \ref{fig:keyboardEvents}
demostra a captura do pressionar das setas em JavaScript. Cada
número corresponde a um botão do teclado especificado através da tabela ASCII.

\begin{figure}[H]
\centering
\begin{verbatim}
window.addEventListener('keydown', function(event) {
  switch (event.keyCode) {
    case 37: // Left
      Game.player.moveLeft();
    break;

    case 38: // Up
      Game.player.moveUp();
    break;

    case 39: // Right
      Game.player.moveRight();
    break;

    case 40: // Down
      Game.player.moveDown();
    break;
  }
}, false);
\end{verbatim}
\caption{Utilização dos eventos do teclado}
\label{fig:keyboardEvents}
\source{\url{http://nokarma.org/2011/02/27/javascript-game-development-keyboard-input/}}
\end{figure}

\subsection{Gamepad}

Atualmente a única forma de utilizar gamepads na WEB é
através software de terceiros. E sua utilização é limitada
restringida a emulação de mouse e teclado subutilizando seus
recursos \autocite{gamepad}.

Em 2015 a W3C introduziu uma API de Gamepads, atualmente em rascunho,
que pretende solucionar estes problemas. A especificação define um
conjunto de eventos e um objeto \textit{Gamepad} que, em conjunto,
permitem manipular os estados de um Gamepad eficientemente.

Exitem eventos para atividades esporádicas como a conexão de
desconexão de dispositivos. Já acontecimentos mais frequentes, como o
pressionar de botões, é detectado através da inspeção dos objetos
aninhados ao principal \textit{Gamepad}. Cada objeto botão contém um 
atributo \textit{pressed} que pode ser utilizado para saber se foi pressionado.

\begin{figure}[H]
    \centering
    \includegraphics[width=0.8\textwidth,natwidth=610,natheight=642]{gamepad.png}
    \caption{Objetos de um Gamepad}
    \label{fig:gamepad}
    \source{https://w3c.github.io/gamepad}
\end{figure}

A figura \ref{fig:gamepad} contém os objetos tradicionais de um Gamepad.

%}}}
\section{ORIENTAÇÃO}
%{{{
Muitos dispositivos móveis contam com tecnologias que permitem detectar
a orientação física e movimento como acelerômetros e giroscópios.
Visto que são comuns em dispositivos móveis, jogos podem se beneficiar
deste tipo de ferramenta para criar experiências peculiares para seus
usuários.

A W3C tem uma especificação em rascunho que abstrai as diferenças dos
dispositivos e prove uma API padronizada para consumir informações
de orientação. A especificação define dois eventos de DOM
principais: \textit{deviceorientation} e \textit{devicemotion}. 

O evento \textit{deviceorientation} prove a orientação do
dispositivo expressa como uma série de rotações a partir de um
ponto de coordenadas locais \autocite{orientationSpec}. Para colocar
claramente, a o evento lançado em uma mudança de orientação provê
variáveis correspondentes a eixos (alfa, beta, gama) que podem ser
consultadas para determinar a orientação do dispositivo.

Já o evento \textit{devicemotion} dispõe de informações de
orientação, como o evento \textit{deviceorientation} com o adicional
de informar a aceleração do dispositivo. A aceleração também é descrita
em eixos e sua unidade de medida é metros por segundo.

%}}}
\section{HTTP/2}
%{{
HTTP/2 é a última verão do protocolo de trocas de documentos entre
cliente e servidor na WEB. Quando um navegador requisita algum documento
de esta requisição é geralmente feita através do protocolo HTTP. O
foco da nova versão do HTTP é performance; especialmente a latência
percebida pelos usuários e o uso de rede e servidores \autocite{http2}.

A forma que os documentos trafegam do servidor para o
cliente afeta diretamente a performance de uma aplicação, nos jogos
esse fator se amplia devido a grande quantidade de arquivos geralmente
necessários para montar uma cena de jogo.

Diferentemente do HTTP/1, HTTP/2 abre apenas uma conexão por servidor.
Usando a mesma para trafegar todos os dados necessários para montar
a página HTML. Dessa forma o HTTP/2 não necessita repetir as
negociações de protocolo nem aguardar parado quando o limite de
requisições que os navegadores suportam concorrentemente é atingido.
Segundo \citet{gameAssetManagement} o limite concorrente de conexões por
servidor é geralmente 5. Jogos que muitas vezes trafegam muitos objetos
via rede, podem se beneficiar substancialmente.

Outro benefício do HTTP/2 em relação a seu predecessor é que
os cabeçalhos das requisições são comprimidos, diminuindo
substancialmente o tamanho das requisições. HTTP/2 também permite
mensagens do servidor para o cliente (\textit{full-duplex}), o
que abre um leque de novas oportunidades que antes só podiam ser
obtidas através de requisições de tempos em tempos ao servidor
(\textit{pooling}) ou através de WebSockets.

HTTP/2 não recomenda a utilização de minificação nos arquivos,
visto que não existe abertura de novas conexões, trafegar múltiplos
arquivos se tornou barato. Dessa forma, algumas práticas, antes
recomendadas no desenvolvimento WEB, tem de ser revistas depois do
HTTP/2.

Dentro do navegador as requisições HTTP/2 não convertidas em
equivalentes do HTTP/1, mantendo a retrocompatibilidade em aplicações
legadas. Sendo assim, os fatores que justifiquem a utilização do HTTP/1
são escassos e a tendência é observarmos cada vez mais aplicações
utilizando com o HTTP/2.

%}}}
\section{DEBUG}
%{{{

Depuração (\textit{debug}) é o processo de encontrar e reduzir defeitos
em um aplicativo de software ou mesmo hardware \autocite{depuracao}.
As ferramentas de desenvolvimento do Google Chrome (\textit{DevTools})
são uma boa opção para depurar aplicações feitas utilizando as
tecnologias da WEB.

Dos depuradores para navegadores o do Google Chrome é o mais fácil
de utilizar e já vem integrado nativamente junto com o software
\autocite{gamesDebug}.

\citet{chromeDevTools} cita algumas funcionalidades do DevTools:
\begin{quote}
Provê aos desenvolvedores profundo acesso as camadas internas do
navegador e aplicações WEB. É possível utilizar o DevTools
para eficientemente detectar problemas de layout, adicionar
breakpoints em JavaScript, e pegar dicas de otimização de código.
\end{quote}

Estas características são comuns na maioria dos depuradores dos
navegadores como o do Internet Explorer (Visual Studio For Web) e do
Firefox (FireBug) \autocite{gamesDebug}.

Com o DevTools também é possível emular dispositivos alvo da
aplicação ou conectar-se a um dispositivo Android real, ideal para
testar ambientes multiplataforma. Para conectar-se a um dispositivo real
é necessário habilitar a depuração via USB no dispositivo, conectar
o dispositivo via cabo, e rodar o aplicativo no dispositivo através
do Google Chrome ou habilitando o modo depuração nos aplicativos
híbridos. Dessa forma é possível visualizar a aplicação dentro
Google Chrome do computador e utilizar as já mencionadas tecnologias do
DevTools.

O DevTools também permite depuração do elemento canvas. Segundo
\citet{html5mostwanted} com o inspetor WebGl é possível conferir os
estados dos buffers, informações de texturas, frames individuais e
outras informações úteis. Todos os recursos mencionados acima estão
disponíveis tanto para o contexto 2d quanto o 3d do canvas (WebGL).

Especificamente para WebGl existe o plugin independente WebGL Inspector.
A ferramenta permite fazer inspeção da execução de métodos
WebGL, observar o estado de texturas, buffers, controlar o tempo de
execução entre outras funcionalidades \footnote{Mais informações
sobre o WebGL Inspector podem ser encontradas no seguinte endereço
\url{http://benvanik.github.io/WebGL-Inspector/}}.

\subsection{Source Maps}

Source Maps é uma tecnologia que permite mapear códigos fontes
minificados para seus respectivos originais. Este recurso é
interessante pois permite que os desenvolvedores visualizem o código
fonte em sua versão original, legível e fácil de depurar, enquanto
entregam ao usuário final a versão minificada, optimizada para
performance. Para o usuário final não há diferença pois Source Maps
são carregados apenas se as ferramentas de desenvolvimento estão
abertas e com a funcionalidade de Source Maps habilitada.

Source Maps foi desenvolvido como um trabalho em conjunto entre a
Mozilla e Google em 2010, atualmente na terceira revisão o projeto é
considerado estável e não recebe modificações na especificação
desde 2013. Sendo suportado por diversas ferramentas de desenvolvimento
como Google Chrome (DevTools) e Firefox.

A especificação prevê a existência de um arquivo \textit{.map} o
qual contém o mapeamento dos arquivos fonte e outros metadados. Este
arquivo é referenciado pelos arquivos minificados de modo a permitirem
o navegador a realizar o mapeamento.

É possível informar o navegador a localização do arquivo de metadados
seguindo a seguinte sintaxe.

\begin{verbatim}
//# sourceMappingURL=/path/to/script.JavaScript.map
\end{verbatim}

Ou através de cabeçalhos HTTP como demostrado abaixo.

\begin{verbatim}
X-SourceMap: /path/to/script.JavaScript.map
\end{verbatim}

Para os arquivos de Source Maps é possível utilizar ferramentas
especializadas ou integrá-los ao processo de \textit{build} como Grunt ou Gulp
\footnote{A biblioteca https://github.com/mishoo/UglifyJS2 é uma
ferramenta de minificação capaz de gerar Source Maps}.

Após ter o jogo ter sido depurado e seus erros minimizados, pode-se
disponibilizar para ser consumido por seus usuários finais.
Abaixo será abordada algumas formas de disponibilizar jogos web em
ambientes multiplataforma.
%}}}
%}}}
\section{DISPONIBILIZAÇÃO DA APLICAÇÃO} %{{{

Os aplicativos puramente em HTML não requerem instalação e
funcionam apenas acessando o endereço através de um navegador.
\autocite{browserGamesTechnologyAndFuture} cita à respeito de
aplicações WEB: por não requererem instalação, sua distribuição é
superior ao estilo convencional de aplicações desktop.

Não obstante, se o objetivo é fornecer uma experiência similar aos
demais aplicativos mobile ou integrar um sistema de compras, geralmente
feitos através de um mercado como o GooglePlay, pode-se adotar a
alternativa híbrida criando um pacote para o software.

O PhoneGap é uma tecnologia que permite encapsular um código em
HTML e disponibilizá-lo nativamente. Não obstante, para empacotar
os aplicativos localmente é necessária configuração substancial
e só é possível empacotar para IOS em um computador da Apple.
Alternativamente o PhoneGap disponibiliza um serviço de empacotamento
na nuvem que soluciona estes problemas o PhoneGap Build.

Através do PhoneGap Build pode-se carregar um arquivo zip, seguindo
determinado formato, o qual contenha os arquivos escritos com as
ferramentas da web, que o PhoneGap Build se responsabiliza por
empacotá-los nos formatos requeridos para serem instalados em Android,
IOS e Windows Phone.

\citet{publishHtml5} descreve os passos que são feitos pelo PhoneGap para 
disponibilizar a aplicação nativamente.

\begin{itemize}
\item É criada uma aplicação nativa utilizando a WebView da plataforma;
\item Todos os recursos da aplicação são armazenados dentro da aplicação nativa;
\item O PhoneGap carrega o HTML dentro da WebView;
\item A WebView mostra a aplicação para o usuário;
\end{itemize}

No endereço https://github.com/phonegap/phonegap-start encontra-se um
template no formato requerido pelo PhoneGap Build que pode ser utilizado
para começar aplicações que serão servidas através da solução.
%}}}
\section{TRABALHOS SIMILARES}
%{{{
\citet{crossPlatformMobileGame} elaborou uma revisão de aspectos do
HTML5 através da construção de um jogo. O autor foca muito nos
aspectos de criação de jogos e feedback do desenvolvimento. Troca
de tecnologias e não especificamente nas limitações conforme o meu
trabalho. Em outras palavras seu escopo é mais genérico e não tão
preciso quanto este

\citet{aSeriousContender} realizou uma pesquisa através de questionário
e protótipo sobre a viabilidade de aplicativos em HTML5, concluindo que
no geral desenvolvimento de aplicativos em HTML5 são opções viáveis
e lucrativas. Seu trabalho difere a este por não focar no contexto dos
jogos, não observando muitas das nuances e necessidades específicas
para o desenvolvimento de jogos. Outra diferença substancial é que o
autor foca apenas na viabilidade não ressaltando as limitações da
plataforma.

\citet{crossPlatformMobileGameDevelopment} revisa algumas tecnologias
da web e constrói um jogo protótipo para aprender um framework que
possibilita a construção de jogos multiplataforma. Não obstante
o autor foca em um framework de compilação múltipla, não usando
diretamente as tecnologias da WEB. O autor também não foca na experiência do
desenvolvimento ou em coletar limitações como este projeto se propõe.

\citet{viabilityBusinessApplications} estuda a viabilidade de
aplicações comerciais multiplataforma em HTML5 através da
construção de um aplicativo comercial em Sencha Touch. É construída
uma lista de recursos interessantes no desenvolvimento comercial
de aplicações e cada um destes recursos é revisado depois do
desenvolvimento, assemelhando-se muito a metodologia deste trabalho. Em
seu estudo o autor utiliza uma ferramenta de desenvolvimento em C\# que
compila nativamente o que se distancia da proposta deste trabalho de
avaliar as limitações com a construção de um protótipo diretamente
em HTML. O autor também não se foca em tecnologias dos jogos,
outrossim aplicações genéricas.

%}}}

%}}}
\chapter{PROJETO}
%{{{
\thispagestyle{myheadings}
Este capítulo tem por objetivo detalhar os aspectos do desenvolvimento
do protótipo. Sua mecânica, funcionamento, decisões tecnológicas,
diagramas entre outros aspectos serão tratados aqui.

\section{MECÂNICA}

Para a análise prática das limitações foi escolhido um jogo de
matemática simples. Consistindo na geração de equações com uma
resposta candidata. Cabe ao usuário informar se o resultado apontado
pelo jogo está correto ou não. A cada resposta dada o nível de
complexidade da equação cresce. O tempo é um fator determinante
no resultado do jogo pois quão mais rápido o jogador acertar se a
afirmação está correta ou não mais pontos ele receberá.

Esta categoria de jogo foi selecionada por ter profundidade, oferecendo
a possibilidade de explorar diversos recursos do HTML, e criar melhorias
incrementais. E também por oferecer uma dificuldade técnica não
tão desafiadora visto que não disponho de experiência profunda no
desenvolvimento de jogos em HTML.

Jogos como o Math Workout e o Countdown para Android tem uma temática
similar. Não obstante, o Math Workout não apresenta a resposta, sendo
o papel do usuário computar a equação e digitar o resultado. Já o jogo
Countdown apresenta um número final e requer que o usuário determine
a equação que resultou no valor à partir de um dado conjunto de
números e operadores.

O jogo desenvolvido para o protótipo parece ter uma melhor jogabilidade
em dispositivos móveis que ambos os jogos acima citados pois não
requer a presença de um teclado numérico. Os botões de verdadeiro
ou falso contém todas as possibilidades de interação com o jogo. A
figura \ref{fig:gameScreen} demonstra a interface contendo os botões descritos.

Abaixo estão detalhados os requisitos que uma mecânica como a descrita
acima deve prover.
%Também para não interferir na pesquisa busquei não me distanciar do
%que é considerado padrão em ferramentas e métodos.

\section{Requisitos}

\subsection{Requisitos funcionais}

As funcionalidades que o sistema deve apresentar estão descritas abaixo.

\begin{itemize}
    \item O sistema deve prover equações matemáticas de dificuldade crescente para o usuário informar se estão corretas ou não;
    \item O sistema deve pontuar as respostas dadas com maior agilidade com uma pontuação maior, que as respondidas com menor agilidade;
    \item O sistema deve apresentar a colocação da partida do usuário em comparativo com seu histórico.
\end{itemize}

\subsection{Requisitos não funcionais}

Outros aspectos requisitados mas que todavia não fazem parte da regra de negócio.

\begin{itemize}
    \item O sistema deve ser desenvolvido utilizando as ferramentas da web.
    \item O sistema deve funcionar para a plataforma desktop e Android.
    \item O sistema deve ser desenvolvido sem a utilização de nenhuma biblioteca ou framework.
\end{itemize}

\section{Modelagem}

A figura \ref{fig:simpleDiagram} apresenta o diagrama de classes
simplificado \footnote{Nos anexos pode-se encontrar a versão completa
do diagrama de classes}.

\begin{figure}[H]
    \centering
    \includegraphics[width=0.8\textwidth,natwidth=610,natheight=642]{ClassesSimpleView.png}
	\caption{Diagrama de classes simplificado}
    \label{fig:simpleDiagram}
\end{figure}

\section{Desenvolvimento}
%Comecei escrevendo o aplicativo para o Navegador do desktop pois era o
%que estava mais acessível no momento.

O desenvolvimento se deu com uma postura de melhoria progressiva
criando primeiramente  a versão mais simples possível para atingir as
requisitos funcionais. A partir dessa versão, novos recursos foram
sendo adicionadas para melhorar a experiência do usuário. Iniciei o
desenvolvimento criando para a plataforma Desktop pois esta contém
um grande número de ferramentas de desenvolvimento nativas que não
requerem integração especial.

O primeiro passo foi a criação do documento HTML, foram utilizados elementos div
para simbolizar telas do jogo. Conter todas as telas em único
HTML caracteriza o jogo como uma aplicação de uma única página
(\textit{Single Page Application}). Alternativamente poderia se depender
de um servidor para mandar as páginas prontas, mas isso distribui
a complexidade do sistema para tecnologias do lado do servidor,
distanciando-se da proposta de um jogo construído exclusivamente com as
tecnologias da Web.

Seguindo a construção do documento HTML veio o desenvolvimento de um
CSS simples que comporta a visualização em múltiplos dispositivos.
Para tal, foram utilizadas posições e tamanhos relativos. Por exemplo,
a largura de cada tela da SPA é 98\% do tamanho total disponível no
dispositivo. Já o tamanho da fonte do quadro principal, representado pelo id
\textit{billboard}, é duas vezes o tamanho da fonte normal 2em.

Sem a ajuda de bibliotecas especializadas o processo de criação da
interface não é trivial. Apesar dela ser simples, fazer os
elementos se alinharem em diversos tamanhos de telas não é fácil e
pode se tornar um problema substancial para interfaces realmente complexas.

Após a concepção do CSS deu-se início ao desenvolvimento da lógica
das páginas em JavaScript. O primeiro passo consistiu em habilitar o
funcionamento de múltiplas telas em um único arquivo HTML através
de JavaScript. Os botões que levam a outras telas, quando clicados,
simplesmente escondem todas as seções e por fim carregam a que querem
mostrar.

Aplicativos SPA introduzem outros desafios, por exemplo,
o botão de voltar perde sua utilização visto que não se está
trafegando de uma página a outra de fato. HTML5 provê uma API em
JavaScript para manipular o histórico que pode resolver este problema;
não obstante, no protótipo não adicionamos esta funcionalidade pois
a quantidade de telas é realmente baixa e os benefícios introduzidos
seriam pequenos.

Ao iniciar a execução do JavaScript todas as divs que representam
telas são escondidas e a div de carregamento de recursos é mostrada em
seguida. O objetivo desta tela é não deixar o usuário sem feedback enquanto
todos os recursos necessários para a utilização do jogo estejam disponíveis.
Não obstante o carregamento é realmente rápido e o usuário, a não ser que 
dependa de uma rede muito lenta, geralmente não vê a tela.
No final do carregamento de todos os recursos é disparado o
evento \textit{window.onload} neste momento carrega-se a div da partida
e dá-se início a mesma.

Durante esta fase do desenvolvimento também foram adicionados
\textit{listeners} aos demais elementos interativos do jogo, como
clicar nas configurações e nos botões de certo e errado da página
principal. Deixando a página  pronta para receber a funcionalidade 
propriamente dita.

Com esqueleto da aplicação definido foi introduzido a lógica
de negócio. A figura \ref{fig:simpleDiagram} apresenta do a
relação entre as classes do jogo. De toda a regra do jogo, a classe
\textit{Match} é a mais importante. Ela simboliza uma partida dentro
do jogo, é na classe \textit{Match} que as informações de quantas
equações existem, quantas foram acertadas e o tempo total da partida
bem como a pontuação atual do usuário. O propósito da classe
\textit{Match} é iterar a cada pergunta e esperar por uma resposta.
Quando a resposta é dada a classe \textit{Match} computa se a resposta
está correta ou não e o tempo que levou para chegar ao resultado.
Quando não existem mais perguntas para serem processadas é lançado
um evento de final de partida onde o resultado pode ser processado
pelas demais classes do jogo. A figura \ref{fig:gameScreen} demostra o
mecanismo da classe \textit{Match} integrado ao jogo.

O cálculo da pontuação se dá por uma operação matemática simples.
Ao final de uma partida é computado o total de questões acertadas
vezes 10 divido pelo total de respostas menos o total de respostas
certas adicionando-se 20\% do tempo da duração em segundos. A figura
\ref{fig:punctuationCalculation} apresenta o código utilizado para o
cálculo acima descrito.

\begin{figure}[H]
\centering
\begin{verbatim}
parseInt(
(this.rightAnswered * 10 ) / (
    (this.answersTotal - this.rightAnswered) 
    + (this.duration*0.2))
)
\end{verbatim}
\caption{Exemplo de utilização de funções imediatamente invocadas}
\label{fig:punctuationCalculation}
\end{figure}


\begin{figure}[H]
    \centering
    \includegraphics[width=0.8\textwidth,natwidth=610,natheight=642]{board.png}
	\caption{Interface do jogo com equação sendo apresentada}
    \label{fig:gameScreen}
\end{figure}

A construção da classe \textit{Match} disponibilizou o comportamento principal do
jogo; entretanto, nesta etapa não havia um gerador de equações. O
jogo contava apenas com uma coleção de equações preestabelecidas
que eram selecionadas aleatoriamente a cada turno. Adiar a construção
do gerador de equações se provou uma boa escolha pois possibilitou
que o desenvolvimento se focasse em outros aspectos importantes como a
elaboração do laço do jogo, ranking, configurações entre outros.

O ranking serve para armazenar o resultado de cada partida do jogador
possibilitando uma percepção de histórico da performance do jogador.
Os dados são armazenados em Local Storage, escolhido por ter uma API
simples. Arquiteturalmente falando, IndexedDb se encaixaria bem na
aplicação por ter uma interface chave valor, ideal para um ranking,
onde as chaves poderiam ser as posições do usuário. Não obstante,
a interface totalmente orientada a eventos do IndexedDb introduz uma
camada de complexidade desnecessária para os casos simples. Visto que
os requerimentos do protótipo não demandam grande performance ou
armazenamento massivo de dados a opção modesta, Local Storage,
foi preferida. A classe ranking é simplesmente uma interface para
converter partidas e armazenar e recuperar estas informações em Local
Storage. A figura \ref{fig:placar} demostra as informações do resultado
de uma partida bem como a posição do ranking.

%get a better image with better alignment
\begin{figure}[H]
    \centering
    \includegraphics[width=0.8\textwidth,natwidth=610,natheight=642]{score.png}
	\caption{Resultado de uma partida}
    \label{fig:placar}
\end{figure}

O objeto \textit{Settings}, assim como o Ranking, utiliza Local Storage
e provê uma interface para armazenar e recuperar preferências sobre
o jogo. Cada campo editável na tela de configurações contém
\textit{listeners} prontos para registrar no objeto \textit{Settings}
cada mudança que ocorrer em seus estados. Os objetos que utilizam estas
configurações também o fazem através do objeto Settings, nunca
acessando configurações diretamente em Web Storage, dessa forma a
validade das configurações é garantida e a migração para uma forma
de armazenamento diferente é possível com relativa facilidade.
A figura \ref{fig:configurations} demonstra a tela de configurações do 
jogo.

\begin{figure}[H]
    \centering
    \includegraphics[width=0.8\textwidth,natwidth=610,natheight=642]{settings.png}
	\caption{Configurações do jogo}
    \label{fig:configurations}
\end{figure}

Implementados estes mecanismos essenciais para o jogo, pude me focar no
ponto central do negócio e possivelmente o mais complexo: a geração
de equações. A geração das equações é feita randomicamente e
envolve duas classes: \textit{Statement} e \textit{StatementGenerator}.
O objeto \textit{Statement}, é responsável por armazenar as
informações de uma equação, à dizer: a afirmação sendo feita
e se seu resultado está correto ou não (a reposta da afirmação).
O objeto \textit{StatementGenerator} é responsável por gerar
instâncias da classe \textit{Statements}.

A classe \textit{StatementGenerator} conta com um método
\textit{getStatement} que realiza o processamento para gerar um novo
\textit{Statement}. Esta função recebe como argumento um inteiro que
simboliza a dificuldade da equação. A cada iteração do usuário
armazenada no objeto \textit{Match} a dificuldade é incrementada
e repassada para o gerador. O valor da dificuldade é utilizado
internamente no \textit{StatementGenerator} para selecionar qual
operador será utilizado e o tamanho do multiplicador dos números
que compõem as equações. Para colocar claramente: as equações
são geradas através da randomização de valores e operadores (com
suas respectivas dificuldades processadas), seguido da execução da
equação para determinar seu resultado e a geração, em 50\% dos
casos, de um valor errado, de modo que a resposta não seja sempre
correta.

O objeto \textit{StatementGenerator} reside como membro de classe de
\textit{Match} e a cada interação com o usuário uma nova equação
é gerada por ele e armazenada no atributo \textit{currentStatement}
do objeto \textit{Match}. Ao final das iterações com o usuário o
evento \textit{endOfMatch} é lançado onde os pontos são computados,
armazenados e mostrados para o usuário. Neste ponto é possível
começar outra partida reiniciando o processo.

Estas classes comportam os requisitos funcionais do jogo. Entretanto,
após um período de uso, foi identificado que muitas vezes o usuário
começa uma partida e não está prestando atenção para a tela o
que acarreta na perda de pontos no período inicial da partida. Para
reduzir este problema foi adicionada a classe \textit{Countdown},
que é basicamente um temporizador regressivo que demarca o início
de cada partida, notificando o usuário quanto falta para a partida
iniciar. O temporizador foi desenvolvido em canvas e apresenta 4
demarcações desenhadas a cada 90 graus, formando um circulo com uma
mensagem no centro, neste caso os números do temporizador. A figura
\ref{fig:counter} demonstra o contador prestes a iniciar uma nova
partida.

\begin{figure}[H]
    \centering
    \includegraphics[width=0.8\textwidth,natwidth=610,natheight=642]{countdown.png}
	\caption{Contador em Canvas}
    \label{fig:counter}
\end{figure}

Para melhorar a experiência em desktops foram adicionados controles de
teclado além dos já presentes botões na interface, possibilitando que
o usuário utilize o teclado além do mouse. Quando o usuário utilizar
a seta para esquerda ou direita os botões Sim ou Não respectivamente,
são clicados através de JavaScript. Simular o clique ao invés ao
invés acionar duas vezees o tratamento das escolhas de resposta foi uma boa
estratégia pois centralizou o tratamento evitando duplicação de código.

No mobile, com intuito de melhorar a experiência, foi adicionada
vibração para as respostas erradas. A API de vibração é trivial e sua 
utilização possibilita uma experiência mais profunda com a aplicação, sendo 
uma adição de bom custo/beneficio.

Durante o desenvolvimento foram utilizadas funções imediatamente
invocadas para declarar as classes. Isso se demonstrou uma boa forma
de separar os objetos, tornando o conflito de variáveis globais um
problema inexistente. Outro aspecto positivo foi a utilização de
um meta objeto para encapsular os demais, similar ao conceito de
namespaces, neste caso utilizei o nome MyMath garantindo que problemas
de conflitos de nomes não aconteçam.

Ao final do desenvolvimento foi feita a integração de Grunt com
plugins de minificação do JavaScript e CSS \footnote{Para mais
informações sobre o Grunt veja os apêndices}. O grunt foi configurado 
para disponibilizar a aplicação para distribuição dentro do diretório
\textit{dist/web}. 

Após de a disponibilização Web estar funcionando, foi feita a
integração para o PhoneGap build. Para tanto foi utilizado um makefile
que simplesmente copia os arquivos de distribuição da Web, gerados
pelo grunt, e o arquivo de metadados do Phonegap Build e cria a árvore
de arquivos padronizada que o phonegp build requer.

%Para melhor a experiencia do usuário foram adicionados sons nos botẽos utilizando a Api de audio em JavaScript.

\subsection{Performance}

A performance, mesmo sem otimizações ficou razoável. O tempo de
carregamento no Google Chrome desktop em uma rede 4G, comum no Brasil,
ficou em 1.4 segundos em média. Esta métrica foi extraída através da
ferramenta de depuração do Google Chrome que fornece a possibilidade
de simular a velocidades de redes.

Utilizado os arquivos minificados, gerados pelo Grunt, o tempo médio
de download ficou em 1.1 segundos. Uma diferença substancial dada a
quantidade pequena de arquivos minificados 8 arquivos JavaScript e um
arquivo CSS.

A seguir serão apresentadas as limitações do HTML encontradas durante
o desenvolvimento do protótipo e pesquisa relacionada.

\section{Otimizações para jogos} \label{optimizations}
%{{{
Navegadores tentam otimizar a experiência de navegação definindo
um conjunto de regras e configurações razoáveis para a maioria dos
casos. Não obstante, nem sempre estes valores padrões são as melhores
opções no contexto de jogos. Abaixo seguem algumas otimizações
nas tecnologias da Web para jogos que foram identificadas através da
revisão e desenvolvimento do protótipo.

\subsection{CSS}

Scroll é um recurso interessante para longas páginas de texto,
o mesmo não se pode dizer à respeito de jogos.
Principalmente aqueles dependente de contato com a tela, pois
no contato a tela pode se mover e desconcentrar o usuário. Para
remover este comportamento deve-se utilizar o \textit{overflow:
hidden;} do seletor do corpo do documento (\textit{body}).

A barra de endereço é outro recurso de pouca utilidade no contexto de
jogos, e muitas vezes um empecilho para jogos em dispositivos móveis,
devido ao limitado tamanho da tela.

Para desabilitar a barra em dispositivos da Apple pode-se utilizar a
seguinte configuração:

\begin{verbatim}
<meta name="apple-mobile-web-app-capable" content="yes" />
\end{verbatim}

Segundo \citet{homescreenwebapps} o Google Chrome a partir da versão 31
adicionou suporte a esta notação, inclusive o prefixo Apple, mas as intenções
é que o prefixo seja removido nas próximas versões.

Para os demais dispositivos não existe meio oficial de esconder a barra
de endereço. Não obstante, alguns sites recomendam a solução descrita abaixo:

\begin{verbatim}
<body onload="setTimeout(function() {window.scrollTo(0, 1)}, 100)">
</body>
\end{verbatim}

Apesar de não fazer parte da especificação, a maioria dos navegadores
implementa a possibilidade de desativar a seleção de elementos na tela.
Em jogos essa possibilidade é útil, pois não é natural a seleção de texto
neste tipo de software. \citet{html5mostwanted} cita que desabilitar
a seleção de texto em jogos é uma otimização importante para a
experiência do usuário. Para desabilitar pode-se utilizar as regras
CSS demonstradas abaixo.

\begin{verbatim}
-moz-user-select: none;
-webkit-user-select: none;
-ms-user-select: none;
\end{verbatim}

\subsection{JavaScript}

A seguir serão descritas algumas otimizações de JavaScript no
contexto de jogos.

\subsubsection{Modo estrito}

Um recurso interessante do JavaScript é seu modo estrito, este faz
um conjunto de modificação na semântica do interpretador de modo
que alguns recursos suportados, mas propensos a problemas, sejam
desabilitados. Um exemplo é variáveis não prefixadas pela palavra
chave \textit{var}.

O modo estrito pode ser entendido como uma variante mais rígida
do JavaScript. O modo restrito pode ser habilitado utilizando o
termo \textit{"use strict";} nos cabeçalhos de arquivos ou funções
permitindo que código não estrito trabalhe em conjunto com código
estrito, característica conveniente para a utilização em sistemas
legados.

\subsubsection{Funções imediatamente invocadas}

Um problema comum de sistemas complexos em JavaScript é que muitos
objetos vivem em ambiente global. Isso pode causar uma coleção de
problemas, desde conflitos de nomes à sobrescrita de variáveis. Para
contornar esse problema pode-se utilizar as funções imediatamente
invocadas IFE (\textit{Immediatly invoked function expression}).

\begin{figure}[H]
\centering
\begin{verbatim}
    (function() {
        'use strict';

        function bar() {
            return 'foo';
        }

        window.bar = bar;
    })();
    window.bar();
\end{verbatim}
\caption{Exemplo de utilização de funções imediatamente invocadas}
\label{fig:iife}
\end{figure}

A figura \ref{fig:iife} demonstra a utilização deste padrão. As
funções definidas no mesmo nível que bar não estarão no contexto
global - a não ser que seja especificado diretamente - e não sofrerão
conflitos de nomes e outros problemas relativos ao contexto global.

Outro fator importante na construção de jogos em JavaScript é a
otimização do laço do jogo. Para escrever o laço é possível
utilizar as funções do JavaScript \textit{window.settimeout} ou
\textit{window.setInterval}. Não obstante, a forma mais recomendada é
utilizar o \textit{window.requestAnimationFrame} reduz ou completamente
para a execução do laço enquanto o usuário está em outra aba.
Isso reduz o consumo de bateria, uma característica importante para
dispositivos móveis.

\subsubsection{HTML}

Um problema que jogos em HTML sofrem é a demora no carregamento da
grande quantidade de recursos que geralmente precisam estar em memória
para o jogo funcionar. Muitos jogos utilizam uma tela de carregamento
enquanto os recursos são adquiridos.

Existem algumas formas de minimizar este problema, como 
utilização de cache, CDN's, minificação, entre outros.

Uma funcionalidade do HTML interessante para otimizar o tempo de rede
o pré carregamento de recursos (\textit{Link Prefetching}). Esta
tecnologia possibilita que o navegador, em seu tempo livre, adquira
recursos que provavelmente serão necessários em um futuro próximo.

Nem todos os recursos necessitam ser pré carregados, mas uma impressão
muito superiora é criada se os recursos estão imediatamente
disponíveis quando uma nova fase é carregada \autocite[p. 39]{creatingFun}.
%}}}


%}}}
\chapter{RESULTADOS}
%{{{
\thispagestyle{myheadings}

Muitos das limitações dos jogos multiplataforma não são
problemas específicos dos jogos, mas aplicam-se a todos tipos
de software \autocite[p. 3]{currentStateCrossPlatform}. Alguns
problemas são inerentes da categoria multiplataforma \citet[p.
7 ]{viabilityBusinessApplications} afirma que é geralmente muito
mais complexo obter aparência nativa, funcionalidade e performance
em aplicações multiplataforma. Outros problemas derivam-se dos
dispositivos ou da tecnologia atual.

Pode não ser trivial distinguir limitações de
problemas. Uma forma de interpretar limitações é como sendo problemas
que impedem algo seja feito. Para ajudar na distinção os
problemas encontrados foram descritos como contornáveis ou não.

A abaixo constam os problemas e limitações do HTML5, aplicáveis,
mesmo que não exclusivamente, aos jogos encontrados durante a pesquisa
e concepção do protótipo. Quando possível, buscou-se apresentar
as limitações na mesma ordem das tecnologias estudadas na revisão
bibliográfica. Não obstante, algumas limitaçãos foram tratadas em
partes separadas visto que se aplicam a várias das tecnologias da
Web. Para melhor organizá-las junto ao texto, foram adicionados
códigos as limitações que seguem o seguinte padrão: LMT + número da
limitação.

\section{HTML}

\noindent\citet{crossPlatformMobileGame} afirma que (\limitation{multipleTesting}):
\begin{quote}
Enquanto o HTML é desenvolvido muitas das funcionalidades
disponibilizadas são testadas em um pequeno conjunto de navegadores
para um pequeno conjunto de versões. Isso acarreta em suporte
inconsistente. A forma mais segura de garantir suporte é testando em
todas as versões alvo, entretanto essa solução não é prática.
\end{quote}

Este não é um problema exclusivo dos navegadores. Segundo
\citet{chromeWebView} da versão 4.4 do Android em diante a WebView
mudou de um projeto local para utilizar o Chromium. Entretanto, grande
parcela dos usuários Android ainda utilizam o sistema antigo o que
força os desenvolvedores suportarem ambas as versões. É razoável
afirmar que para o início de 2016 um terço dos usuários de Android
ainda utilizem a versão antiga da WebView \autocite{chromeWebView}.

O Crosswalk é uma tecnologia que pode resolver parcialmente o problema
de suporte de funcionalidades em diversas versões de dispositivos.
O Crosswalk funciona disponibilizando, juntamente com a aplicação,
uma versão recente do Chromium que será responsável por rodar a
aplicação. Desse modo todos os dispositivos que utilizam o pacote
gerado pelo Crosswalk vão rodar na mesma versão de navegador. Não
obstante, o Crosswalk só é suportado para plataforma Android \footnote{Na
seção \ref{crosswalk} dos Apêndices o Crosswalk é tratado com mais
detalhes}.

\section{CSS}

É muito custoso desenvolver interfaces que pareçam nativas
para cada dispositivo sem a utilização de plugins e outras
ferramentas auxiliares. No protótipo foi utilizada uma estilização
simples a qual pode ser interessante na Web. Conquanto, nos
dispositivos móveis, o layout criado para o jogo é muito diferente
da experiência normal destes aparelhos. Sendo que cada sistema
operacional conta com regras próprias para definir a experiência
do usuário \footnote{ O Android utiliza o projeto Material design
\url{https://www.google.com/design/spec/material-design} como base
para suas regras de como construir interfaces. Já o IOS utiliza um
conjunto de regras próprias que podem ser encontradas neste endereço
\url{https://developer.apple.com/library/ios/documentation/UserExperienc
e/Conceptual/MobileHIG/}} (\limitation{hardToBuildGuis}).

Uma solução para este problema é utilizar frameworks como o jQuery
Mobile e Kendo UI Mobile \footnote{Mais informações sobre o Kendo UI
e jQuery Mobile podem ser encontradas nos apêndices}. Estes frameworks
permitem criar elementos típicos de interfaces mobile como listas
com scroll, botões e transições com uma aparência nativa de forma
relativamente fácil \autocite{publishHtml5}.

Em alguns casos o tamanho das telas pode ser um fator limitante,
como por exemplo em jogos de estratégia. Estes jogos geralmente
necessitam mostrar uma vasta quantidade de informações, neste
contexto, jogadores com telas menores podem sair em desvantagem
(\limitation{differentScreenSizesMayPutSomeUsersInDisvantage}).
Em termos gerais, pode-se mitigar este problema utilizando design
responsivo. Na perspectiva do desenvolvedor, pode-se lidar com
múltiplos tamanhos de tela trabalhando com tamanhos relativos via
CSS. Todavia há casos, como o dos botões de certo e errado
do protótipo, em que a proporções ficam exageradas. Nestas ocasiões,
utilizar controles como o \textit{max-width} e \textit{min-width} é uma
solução conveniente. No protótipo, os botões de certo e errado tem uma proporção
de 40\% da tela. Em dispositivos móveis este valor é sensato, mas
em desktops com grandes resoluções a largura fica exagerada. Para
resolver este problema foi configurado via CSS que a largura dos botões
tenham no máximo 300 pixels via regra \textit{max-width}.

Outro fator problemático do CSS é a presença de prefixos
(\limitation{cssPrefixes}). Tecnologias experimentais do CSS geralmente
levam o nome do distribuidor como prefixo da propriedade. Utilizar as
mesmas regras em CSS com prefixos diferentes para suportar diversos
motores de renderização é um processo entediante e propenso à
duplicação e erros. A biblioteca \textit{-prefix-free} é uma
possível solução para este problema, detectando automaticamente
quando prefixos são necessários e adicionado-os em tempo de execução
\footnote{Mais informações sobre a biblioteca pode ser encontradas nos
apêndices}.

\section{JavaScript}

O JavaScript, por ser uma linguagem desenvolvida por consenso,
tem um ciclo de vida de atualizações demorado; pois necessita
que todos os consumidores da especificação entrem em acordo
(\limitation{jsSpecificationCycle}). Com a especificação
pronta, outra fase demorada é a adoção das tecnologias
nos navegadores (\limitation{jsImplementaionCycle}). O site
\url{https://kangax.github.io/compat-table/es6/} contém uma lista do
suporte aos recursos do ECMAScript nos motores JavaScript. Através
desta lista pode-se observar que as novas funcionalidades do ECMAScript
6 contém pobre suporte na grande maioria dos motores, especialmente
quando referente a funcionalidade de subclasses.

Por ser uma linguagem interpretada, erros tem de descobertos rodando a
aplicação (\limitation{discoverErrorsWhileRunning}). Alternativamente,
se JavaScript fosse compilado, vários problemas poderiam ser capturados
e informações úteis reveladas antes de se testar \autocite[p.
12]{viabilityBusinessApplications}. Uma forma de solucionar 
este problema é utilizando alguma linguagem compilada através de 
Web Assembly.

Arquivos JavaScript geralmente necessitam ser minificados, reunidos e
ofuscados. Muitas vezes esses processos precisam ser executadas durante
o desenvolvimento, e sem a utilização de automatizadores como o
Grunt Watch, o processo torna-se entediante bem como propenso a erros
(\limitation{complexBuild}).

Segundo \citet{htmlResearchJournal}, um problema do JavaScript é que
não é possível transferir métodos de objetos através de sistemas
via WebSockets, somente dados (\limitation{passCompleteObjectsOnSockets}).
Uma forma de contornar este problema é utilizar funções para
converter os dados em objetos em cada ponta do processamento mas isto
adiciona complexidade ao software.

\citet{howBrowsersWork} cita à respeito de scripts requerendo informações de
estilo durante o processo de parse. Se o estilo ainda não foi carregado
o script vai utilizar informações erradas, causando uma série de
problemas (\limitation{runScriptsOnlyOnTheEndOfTheProcessment}). Pode-se
contornar este problema executando scripts exclusivamente ao final da
renderização do HTML.

Por ser uma linguagem orientada a protótipos, JavaScript dá grande
poder e flexibilidade para os desenvolvedores. Não obstante, um
possível problema desta característica é quanto a manutenibilidade de
código (\limitation{harderToDoMaintainence}). \citet{html5Tradeoffs} afirma que:

\begin{quote}
A flexibilidade de manutenção de código em JavaScript é muito
dependente do expertise da pessoa que está escrevendo o código.
Escrever código "sustentável" em JavaScript é mais difícil se
comparado com Java ou C\# mas, aplicando bons padrões de design, é bem
possível escrever bom JavaScript.
\end{quote}

Além de bons padrões de design, outra forma de minimizar
os problemas de manutenibilidade de JavaScript é utilizando
guias de estilo de escrita. \citet{jsStyleGuide} afirma que
quando não existem padrões de escrita todos acabam escrevendo
código que apenas os autores conseguem entender. O endereço
\url{http://noeticforce.com/best-JavaScript-style-guide-for-maintainable
-code} contém um comparativo entre os guias de estilo mais comuns para
JavaScript.

\subsection{Sistema de tipos}

O sistema de tipos do JavaScript também é problemático. Erros
numéricos resultam no valor NaN (\textit{not a number}). Sendo que
todas as operações com NaN como operadores irão retornar outro
NaN. Isso torna a depuração de erros desnecessariamente complexa
\autocite{html5mostwanted} (\limitation{NANPropagation}). Uma saída
para este problema é checar constantemente os tipos de forma a
assegurar-se que eles continuam válidos. Entanto, este processo
pode se tornar burocrático e uma forma adicional de criar erros.

Outra limitação do JavaScript é quanto a checagem de tipos. Ao
testar o tipo de uma variável vazia com a função \textit{typeof}
o JavaScript retorna como se a variável fosse um objeto
(\limitation{typesCheck}).

\begin{figure}[H]
\centering
\begin{verbatim}
function is(type, object) {
    type = type[0].toUpperCase + type.slice(1);
    return Object.prototype.toString.call(object)
    === '[object ]' +  type + ']';
}
\end{verbatim}
\caption{Função para testar tipos que funciona como o esperado.}
\label{fig:fixJSTypes}
\source{\url{http://www.slideshare.net/fdaciuk/javascript-secrets-front-in-floripa-2015}}
\end{figure}

A figura \ref{fig:fixJSTypes} demonstra uma solução possível para
remediar o problema dos testes de tipos em JavaScript. A diferença
desta função é que ela converte a variável que se está testando
para sua representação em texto a qual contém o nome do tipo escrito
por extenso - utilizando-se deste valor pode-se deduzir o tipo da
variável corretamente.

\subsection{Performance}

Apesar da performance ter notavelmente melhorado, ainda é geralmente
menos eficiente produzir animações em JavaScript do que utilizando
transições e animações do CSS, que por sua vez são mais
otimizados e acelerados via hardware \autocite{html5mostwanted}
(\limitation{JSanimations}). O Web Assembly pode resolver este problema,
substituindo o JavaScript para os casos onde grande performance é
necessária. Mas para isso o Web Assembly precisa evoluir em sua
especificação e implementações.

\subsection{Fullscreen}

Não existe forma padronizada de detectar se uma aplicação
está em tela cheia ou não através de JavaScript
(\limitation{fullscreenManagement}). O IOS suporta a variável
\textit{navigator.standalone } para identificar se a aplicação está
em tela cheia. \citet{homescreenwebapps} recomenda a utilização do
trecho de código para determinar se está em modo tela cheia para
os demais navegadores. Não obstante, não existe solução oficial,
tanto para detectar quanto para manipular Fullscreen. O argumento da
não possibilidade de manipulação é que isso pode ser um problema de
segurança para o usuário.

\begin{figure}[H]
\centering
\begin{verbatim}
navigator.standalone = navigator.standalone 
|| (screen.height-document.documentElement.clientHeight<40)
\end{verbatim}
\caption{Teste de tela cheia}
\label{fig:fixJSTypes}
\source{\url{http://www.mobilexweb.com/blog/home-screen-web-apps-android-chrome-31}}
\end{figure}

\section{SVG}

Uma das grandes vantagens do SVG é que é definido via linguagem
de marcação e se integra bem com as demais tecnologias da Web.
Contudo, isso também implica em problemas de performance para
arquivos muito grandes pois os elementos são manipulados via DOM
(\limitation{svgDomPerformance}). Segundo \citet{html5mostwanted} a
grande desvantagem do SVG é que quão maior o documento mais lenta a
renderização.

Controle refinado sobre posicionamento também é um problema do SVG.
\citet{html5mostwanted} afirma que um aspecto negativo do SVG é que
é muito difícil atingir a perfeição na posição dos pixels,
por ser uma linguagem vetorizada (\limitation{svgRefinendControl}).
Essa característica acaba limitando a aplicabilidade do SVG como
renderizador de jogos para os casos não muito complexos, onde posicionamento
não seja uma fator crucial.

\section{Canvas}

Segundo \autocite{html5mostwanted}, os aspectos negativos do canvas é
que a performance varia de plataformas para plataformas e não existe
implementação nativa para animações.

\subsection{Performance}

O problema de performance (\limitation{canvasPerformance}) pode
ser parcialmente remediado com o FastCanvas. FastCanvas é uma
implementação nativa em C++ do canvas para Android que roda
separadamente do JavaScript. Devidas as características acima citadas
FastCanvas é substancialmente mais rápido do que a tag canvas para
navegadores Android. Todavia, o FastCanvas não suporta a
especificação do canvas completamente, e existem algumas diferenças no
comportamento entre o canvas original e o FastCanvas.

\subsection{Integrações}

A implementação de animações de fato não existe no canvas
(\limitation{noCanvasAnimation}). Similarmente carece-se de integração
com as demais tecnologias da Web (\limitation{noCanvasIntegration}).
Os desenhos do canvas não podem ser acessados via DOM nem serem
manipulados via CSS. Todas as modificações necessárias devem ser
feitas através de JavaScript. No protótipo, especificamente na classe
\textit{Countdown}, foi necessário adicionar regras de estilo via
JavaScript para a interação com o canvas ficar completa.

CSS já conta com definições quanto a animações e a manipulação de
elementos do canvas através do DOM facilitaria uma gama de situações.
Por exemplo, seria possível utilizar os eventos do DOM para capturar
interações do usuário através de elementos utilizados no canvas.
Se um retângulo fosse clicado o evento seria lançado a partir dele.
Controle similar é implementado atualmente acessando as coordenadas
do canvas onde uma interação aconteceu e fazendo o processamento com
aquela determinada área.

Felizmente a integração entre o canvas e as demais tecnologias da
Web está começando a acontecer. A interoperabilidade entre o Path2D
do canvas e a notação SVG é uma iniciativa na direção correta,
tornando ambas tecnologias cada vez mais relevantes e dinâmicas.

Outra característica peculiar, descoberta durante o desenvolvimento
do protótipo, é que o Canvas pode gerar resultados inesperados se
seus tamanhos de elemento e tela diferirem de formas específicas. Em
suma, existem dois tamanhos, o tamanho do elemento e da superfície de
desenho. Quando o tamanho do elemento é maior do que o da superfície
de desenho do documento escala a superfície de desenho para preencher o
elemento, o que pode gerar resultados inesperados.

\section{WebGL}

Como a especificação do WebGL é baseada na versão otimizada
para dispositivos móveis do OpenGL não é possível utilizar
muitos recursos especiais disponíveis para os ambientes desktop
(\limitation{noWebglDesktopFunctions}). Como a definição de caminhos
com a função \textit{glBegin} e a utilização de pontos flutuantes
para calcular coordenadas de vértices \autocite{esVsGl}. Em outras
palavras, a especificação do WebGL é cortada por baixo. Visto que
alguns dispositivos não teriam performance utilizando estes recursos,
nenhum dispositivo que use WebGL pode os desfrutar.

Segundo \citet{html5mostwanted} um dos problemas do WebGL é sua alta
curva de aprendizagem e o fato de não ter suporte para o Internet
Explorer. Entretanto, o suporte foi adicionado na última versão do
Internet Explorer (11). Infelizmente a dificuldade de utilização
ainda persiste (\limitation{hardToUseWebGL}), forçando a maioria dos
desenvolvedores a utilizarem abstrações criadas por bibliotecas de
terceiros \footnote{Os apendices contém algumas destas bibliotecas como
o tree.js}.

É importante também ressaltar que o WebGL não se comporta de
maneira simétrica nos navegadores que implementam a especificação.
Pode haver diferenças substanciais de performance em plataformas
diferentes e em navegadores diferentes. Para ver um programa em WebGL
é necessário um navegador recente, uma placa gráfica recente e um
sistema operacional que suporte a tecnologia \autocite{html5mostwanted}
(\limitation{limitedToRecentThingsWebgl}). Existe também uma lista de
placas gráficas com \textit{drivers} bloqueados por não funcionarem
corretamente no Firefox \autocite[p.42]{3daps}.

Além dos problemas citados, o suporte a especificação ainda é
incompleto em vários navegadores (\limitation{incompleteSupportWebgl}).
CocoonJS é uma aplicativo híbrido que preenche a fraca implementação
de WebGL nos dispositivos móveis possibilitando se desenvolver em
WebGL. CocoonJS conta com suporte a dispositivos legados à partir do
Android 2.3 e IPhone 5.

\subsection{Áudio}

Segundo \citet{html5mostwanted} a limitação do elemento de áudio no
HTML5 é que seu propósito é para executar apenas um som, como o som
de fundo dentro de um jogo. Não sendo adequada para efeitos sonoros
ou necessidades flexíveis de áudio.

Durante a concepção do protótipo foi utilizada a API de áudio, devido
sua maior flexibilidade. Infelizmente algumas vezes o som não é
executado ou demora até executar de modo de o que é executado ao mesmo
tempo que o próximo som (\limitation{soundAPIConflicts}).

A experimentação do protótipo, relativo a API de áudio, confirmou
as indicações de \citet{html5mostwanted} que afirma que a API de som
é boa se você deseja apenas tocar alguma música, mas se você está
lançando eventos em um jogo ela ainda é problemática.

Outro problema de áudio em HTML5 são os codecs. Alguns
navegadores favorecem formatos OGG (vorbis) e outros favorecem o
MP3 (\limitation{bestAudioCompressionNotSupportedByAllBrowsers}).
No protótipo foi utilizado o MP3, que apesar de não bem visto por
muitos navegadores, é suportado como uma forma de mínimo múltiplo
comum. Todavia, segundo \citet{opus}, MP3 é um codec antigo e existem
alternativas mais adequadas para a Web como o codec Opus, que oferece
uma melhor relação entre taxa de compressão e qualidade.

Além das restrições de codecs de áudio nos produtos da Apple,
áudio, especificamente no Safari do IOS, contém alguns problemas
específicos. Por exemplo, não é possível trocar o volume
através de JavaScript também não é possível tocar mais de um
som ou vídeo simultaneamente \autocite{unsolvedMediaHtmlIssues}
(\limitation{limitedMultimidiaControlOnSafari}).

\subsection{Vídeo}

O suporte a vídeo, apesar de estar melhorando, ainda é rudimentar.
No Silverlight existem uma coleção de possibilidades como aplicar
shaders diretamente no vídeo \autocite[p. 8]{researchOnHtml}
(\limitation{noEffectsOnVideo}). Não obstante, efeito similar
a shaders em vídeo é atingido através da importação de
vídeo dentro de um contexto WebGL \footnote{O endereço
\url{https://developer.Mozilla.org/en-US/docs/Web/API/WebGL_API/Tutoria
l/Animating_textures_in_WebGL} contém exemplos de como utilizar vídeos
em um contexto WebGL} ou canvas 2D.

Também não é possível controlar a entrada em tela cheia através de
scripts (\limitation{videoFullscreenControl}) pois é considerado uma
violação de segurança. Entretanto, os navegadores tem a opção de
deixar os usuários escolherem ver vídeos em tela cheia através de
controles adicionais \autocite[p. 68]{proHtml5}.

Assim como áudio, o elemento \textit{video} sofre com problemas de
codecs (\limitation{videoCodecs}), segundo \citet{html5Tradeoffs}:
\begin{quote}
O maior problema com as API's de áudio e de vídeo do HTML5 é
a disputa entre os codecs dos navegadores. Por exemplo, Mozilla e
Opera suportam Theora, já o Safari suporta H.264 que também é
suportado pelo IE9. Ambos, Iphone e Android suportam H.264 em seus
navegadores. A W3C recomenda OggVorbis e OggTheora para áudio e vídeo
respectivamente.
\end{quote}

O Safari do IOS também contém problemas exclusivos de vídeo. Segundo
\citet{unsolvedMediaHtmlIssues} não é possível capturar frames de
vídeo usando o método \textit{drawImage} do canvas. Também não é
possível pré carregar arquivos de vídeo sem iteração do usuário
(\limitation{safariVideoMissingControlAndCanvas}).

Para os raros casos onde o suporte a vídeo não existe,
como no Opera Mini, o projeto \textit{Vídeo for Everybody}
\url{http://camendesign.com/code/video_for_everybody} é um polyfill que
recorre à flash para apresentar o conteúdo.

\section{Armazenamento}

Não existe opção oficial que permita a utilização de tecnologias
SQL para Frontend (\limitation{noSqlSupport}). O WebSQL foi uma
tentativa nesta direção mas, visto que não continha mais de uma
implementação, foi descontinuado. Apesar de depreciado o Web SQL ainda
é usado por muitos desenvolvedores e mantido por alguns navegadores,
todavia, os desenvolvedores que procedem desta maneira ficam à mercê
dos caprichos dos navegadores não podendo contar com a força de um
órgão regulador para suportar a tecnologia. SQL seria interessante pois
muitos desenvolvedores tem experiência com este tipo de tecnologia e
por SQL permitir filtragem e correlacionamento de dados de uma forma
flexível e poderosa.

Uma outra característica da Web que pode afetar negativamente
aplicações que utilizam o persistência local de dados é que o
usuário pode configurar o navegador para não aceitar armazenamento
para determinado domínio, possivelmente comprometendo a experiência
de determinada aplicação. Esta característica é muito diferente do
que geralmente é oferecido em aplicações desktop, onde a aplicação
determina o que precisa usar sem necessitar de consenso do usuário, que
por sua vez pode escolher usar os não a aplicação.

\subsection{Web Storage}

Muitos navegadores não permitem armazenar mais de 5MB por domínio
em Web Storage, apesar da especificação permitir fazê-lo
\autocite{gameAssetManagement} (\limitation{webStorageLimit}). Também
não é possível saber quanto espaço já foi consumido pelo Web
Storage (\limitation{webStorageQueryLimit}).

Outra limitação do Web Storage é que todas as informações são
guardadas no formato texto (\limitation{webStorageStringOnly}). Isso
força a converter os valores toda a vez que algo for armazenado ou
recuperado \autocite{gameAssetManagement}. No protótipo isso não foi
problemático pois não existe grande manipulação sobre os dados. Mas
no caso de jogos mais complexos esta característica do Web Storage pode
ser uma limitação substancial.

\subsection{IndexedDB}

Apesar de ser desenvolvido com objetivo de ser uma solução para
todas as necessidades de armazenamento no Frontend IndexedDB ainda
sofre algumas limitações. O comportamento em abas anônimas não
está especificado e os resultados também variam de navegador para
navegador (\limitation{indexedDbAnonymousBehaviour}). Também não é
possível realizar buscas em textos, algo similar ao \textit{LIKE} do
SQL (\limitation{indexedDbNoLike}).

Outro problema encontrado durante a pesquisa é que no Firefox existe
uma pequena probabilidade de os dados se perderem. Isso se dá pois a
API não espera confirmação do sistema operacional para considerar
um dado válido, essa foi uma escolha em detrimento de performance.
Este comportamento pode ser modificado mas é a forma padrão de
funcionamento e os desenvolvedores podem não estar considerando esta
peculiaridade.

\section{Offline}

Nos arquivos de cache via manifestos, quando o download falha, o
navegador emite um evento mas não há indicação de qual problema
aconteceu (\limitation{noErrorMessagesOffline}). Isso pode tornar a
depuração ainda mais complicada que o usual \autocite{diveIntohtml}.
Este problema aconteceu durante o desenvolvimento do protótipo e foi
trabalhoso de depurar até descobrir que um arquivo estava faltando,
e qual especificamente. Uma coleção de códigos de erro e suas
respectivas mensagens na especificação poderia solucionar este
problema.

Os itens da palavra chave \textit{NETWORK:} apresentam um
comportamento não intuitivo. Arquivos de rede que não
são declarados ali não poderão ser consumidos uma vez
que exista cache na aplicação. O exemplo neste endereço
\url{http://appcache-demo.s3-website-us-east-1.amazonaws.com/without-net
work/} demonstra este comportamento. Segundo \citet{gameAssetManagement}
a especificação define que se um arquivo não for listado em alguma
das seções do cache então o arquivo não estará disponível de
qualquer forma para a aplicação.

\section{Orientação}

Visto que a especificação de orientação não está pronta,
o suporte está longe de completo. Da mesma forma, existem
diferenças substanciais nas implementações entre os navegadores
(\limitation{orientationIsntReady}). Por exemplo, o Firefox e Google
Chrome não manipulam ângulos da mesma forma. Outrossim, alguns
eixos se comportam de maneiras opostas \autocite{mdnOrientation}. O
polyfill \textit{gyronorm.js} é uma tentativa de normalizar o uso de
dados de orientação nos navegadores que pode ser utilizada até a
especificação e os navegadores evoluírem.

\section{Detecção de recursos}

\citet{diveIntohtml} cita que (\limitation{checkResourcesOnlyOnJavascrit}):
\begin{quote}
Grande parte da detecção de funcionalidades é feita através de
JavaScript, isso força os desenvolvedores a criarem pelo menos parte da
marcação em JavaScript, isso pode ser um fator limitante para o uso
generalizado de HTML5.
\end{quote}

Para resolver este problema, o ideal seria que cada tecnologia contasse
com formas de detectar as funcionalidades que comporta. Não obstante,
este tipo de mecanismo conflita com outros princípios da Web. Pelo
fato de o HTML uma linguagem fundamentada exclusivamente em marcação
e estrutura, adicionar estruturas de controle não é uma alternativa
viável.

\section{Debug}

Com o depurador remoto do Google Chrome foi observado uma falta de
sincronia nas taxas de atualização da imagem no computador. Os
efeitos do CSS não são bem apresentados sendo provavelmente difícil
depurar animações. Em alguns casos quando a tela foi modificada
substancialmente (com nas mudanças de tela) a mudança de estado no
computador levou vários segundos para ser atualizada.

Outro problema do depurar do Chrome é que não existe inspetor
nativo de canvas. O inspetor de embutido do Google Chrome foi
removido pois continha comportamentos indesejáveis para os
desenvolvedores do navegador \autocite{canvasinspector}. Os
desenvolvedores mencionaram criar uma extensão para a funcionalidade
mas até o tempo de desenvolvimento deste trabalho não havia
nenhuma disponível. A única alternativa viável até o momento é
utilizar uma versão antiga do Chromium. O Firefox introduziu uma
ferramenta de inspeção para o canvas 2D em uma Conferência de
desenvolvimento de jogos em São Francisco em 2014, mas até então a
tecnologia não apareceu no navegador \autocite{firefoxCanvasDebug}.
Atualmente a área de ferramentas para depuração do canvas
2D está parcialmente comprometida por não existir um plugin
de terceiro onipresente nos navegadores como o WebGl Inspector
(\limitation{noCanvas2DIsnpectorOnipresent}).

O projeto Canvas Interceptor é uma iniciativa para preencher
esta demanda permitindo capturar alguns eventos do contexto 2D e
tirar \textit{snapshoots} dos passos da renderização do canvas.
Entretanto, o projeto requer que o desenvolvedor inclua um arquivo
JavaScript em seu projeto e inicialize o depurador dentro do código,
se tornando uma saída mais custosa do que um inspetor integrado
no navegador ou via extensão \footnote{Mais informações sobre o
projeto Canvas Interceptor podem ser encontradas no seguinte endereço
\url{https://github.com/Rob--W/canvas-interceptor}}.

\section{Entrada de comandos}

\citet[p. 9]{aSeriousContender} cita que: fazer scroll juntamente
com gestos de ações similares é uma área onde o HTML ainda é
fraco (\limitation{multiTouch}). Para contornar este problema pode-se utilizar bibliotecas como o
iScroll, TouchScroll, GloveBox, Sencha, jQuery Mobile, entre outros.

\subsection{Gamepad}

A forma de gerenciar a conexão de Gamepads é diferente entre o Firefox
e Google Chrome. No Firefox é lançado um evento toda a vez que o
Gamepad é desconectado, já no Google Chrome é necessário verificar
um vetor de gamepads pela existência do objeto de tempos em tempos
\autocite{gamepad} (\limitation{gamepadObject}). Este problema é
reflexo da especificação ainda incompleta e força os desenvolvedores
a duplicarem a mesma atividade no código.

\section{Disponibilização}

Durante o desenvolvimento notou-se que o serviço PhoneGap Build não
contém a última versão do PhoneGap e não é possível utilizar
plugins através dele. Embora no protótipo isso não tenha sido um
problema, em projetos maiores estes tipos de requerimentos podem ser
essenciais.

Outra limitação de disponibilização, neste caso para aplicativos
Web puros, é que a possibilidade de adicionar aplicativos a área
de trabalho está disponível apenas para o Safari e Google Chrome
(\limitation{desktopIcon}). Este tipo de possibilidade é importante por
gerar uma experiência similar a nativa em dispositivos móveis.

A disponibilização via mercado do IOS também é problemática. Toda
a aplicação precisa ter seu conteúdo revisado a cada nova versão,
e essa revisão é demorada e contém uma fila de espera substancial.
Se for comparado a muitos processos de disponibilização automatizados na
Web, disponibilizar aplicações para o IOS é rudimentar.

\subsection{Monetização}

Outro aspecto que pode ser limitante para aplicações Web é
sua forma de monetização. É muitas vezes difícil encontrar
oportunidades financeiras em jogos para navegador, visto que não
se pode vender pacotes para o usuário baixar \autocite[p. 44]{gameCommunities}
(\limitation{monetizationDifferent}).

Dado este este problema, desenvolvedores são muitas vezes forçados
a criar formas alternativas de lucro. A monetização de aplicativos
Web é geralmente feita através de propagandas \autocite[p.
44]{gameCommunities}. Outra saída é a venda de recursos extra,
\citet[p. 44]{gameCommunities} afirma que a venda de moeda virtual é
uma forma comum de monetização em jogos.

De outra forma, jogos Web podem ser integrados pelos desenvolvedores
a sistemas de pagamento, mas possivelmente aumentado a complexidade
da aplicação. Ou se optarem pela arquitetura híbrida, podem ser
integrados a algum mercado como o do Android ou da Apple.

\section{Outras Limitações}

\citet{html5Tradeoffs} menciona algumas limitações em uma âmbito HTML em geral:

\begin{quote}
Não podemos mudar a imagem de fundo do dispositivo, ou adicionar toques
etc. Similarmente, existem muitas API's de nuvem como os serviços
de impressão do ICloud ou Google Cloud que estão disponíveis para
aplicações nativas mas não para HTML5. Outros serviços utilitários
como o C2DM do Google que está disponível para desenvolvedores Android
para utilizar serviços de \textit{push} também não estão disponíveis
para o HTML5.
\end{quote}


\section{Revisão das Limitações}

O quadro \ref{table:technologies} é um comparativo das
tecnologias pesquisadas e seu suporte. As
informações de suporte foram extraídas do site \textit{Can I Use} e
os navegadores populares em questão são: Internet Explorer (11), Edge
(13), Firefox (43), Google Chorme (47), Safari (9) e Opera (34).

\begin{table}
\begin{tabular}{ |p{3cm}|p{3cm}|p{3cm}|  }
\hline
Tecnologia & Suporte nas últimas versões estáveis dos navegadores populares & Polyfills disponíveis  para versões antigas \\ \hline
Canvas & Sim & Sim \\ \hline
SVG & Sim & Sim \\ \hline
Gamepad & Não & Sim \\ \hline
WebGL & Parcial & Sim \\ \hline
WebSockets & Sim & Sim \\ \hline
IndexedDB & Parcial & Sim \\ \hline
WebCL & Não & Não \\ \hline
WebVR & Não & Não \\ \hline
WebAssembly & Não & Sim \\ \hline
Tag áudio & Sim & Sim \\ \hline
Áudio API & Não & Sim \\ \hline
Vídeo & Não & Sim \\ \hline
Prefetch & Não & Sim \\ \hline
Web Animations & Não & Sim \\ \hline
WebWorkers & Sim & Sim\\ \hline
\end{tabular}
\label{table:technologies}
\caption{Tecnologias dos jogos e seu suporte}
\end{table}

Com este comparativo fica visível que, mesmo em alguns casos não
existindo suporte nativo ou ele sendo incompleto e com limitações,
na vasta maioria dos casos, é possível utilizar tecnologias da Web
direcionadas para o desenvolvimento de jogos.

\newpage

O quadro \ref{table:technologies} é um resumo das limitações mais relevantes 
encontradas e se as mesmas podem ser contornadas ou não.
\begin{longtable}{| p{.09\textwidth} | p{.53\textwidth}| p{.11\textwidth} |}
\hline
Código & Descrição & Existe solução ou contorno? \\ \hline
\Cref{limitation:multipleTesting} & Para garantir que algo de fato funcione é necessário testar em múltiplos dispositivos & Parcial \\ \hline
\Cref{limitation:hardToBuildGuis} & Dificuldade em construir interfaces nativas para os diversos dispositivos & Sim \\ \hline
\Cref{limitation:differentScreenSizesMayPutSomeUsersInDisvantage} &  Desvantagem  para alguns jogadores devido ao tamanho da tela  & Parcial \\ \hline
\Cref{limitation:cssPrefixes} & Tecnologias equivalentes em fase experimental levam o prefixo do motor de renderização  & Sim \\ \hline
\Cref{limitation:jsSpecificationCycle} & Demora para chegar a um consenso na criação de especificações do JavaScript & Não \\ \hline
\Cref{limitation:jsImplementaionCycle} & Demora para implementar novas especificações do JavaScript & Sim \\ \hline
\Cref{limitation:discoverErrorsWhileRunning} & Por ser interpretado, erros no código JavaScript só podem ser descobertos em tempo de execução & Parcial \\ \hline
\Cref{limitation:complexBuild} & Requirimento de diversos passos para testar ou disponibilizar aplicações triviais & Sim  \\ \hline
\Cref{limitation:passCompleteObjectsOnSockets} & Impossibilidade de passar objetos completos em WebSockets & Parcial \\ \hline
\Cref{limitation:runScriptsOnlyOnTheEndOfTheProcessment} & Scripts só podem ser rodados com segurança no final da renderização & Parcial \\ \hline
\Cref{limitation:harderToDoMaintainence} & A manutenibilidade de JavaScript tende a ser pior do que a de outras linguagens comuns como C\# & Sim \\ \hline
\Cref{limitation:NANPropagation} & Propagação de valores inválidos NAN & Parcial \\ \hline
\Cref{limitation:typesCheck} & Checagem de alguns tipos retorna resultados inesperados & Parcial \\ \hline
\Cref{limitation:JSanimations} & É pouco eficiente produzir animações em JavaScript & Parcial  \\ \hline
\Cref{limitation:fullscreenManagement} & Impossibilidade de detectar e manipular \textit{fullscreen} de forma padronizada &Não \\ \hline
\Cref{limitation:svgDomPerformance} & Problemas de performance em arquivos SVG muito grandes & Não \\ \hline
\Cref{limitation:svgRefinendControl} & É praticamente impossível obter posicionamento preciso com SVG & Não \\ \hline
\Cref{limitation:noCanvasIntegration} & A integração do Canvas com as demais tecnologias da Web é pobre & Não \\ \hline
\Cref{limitation:noWebglDesktopFunctions} & Falta de alguns recursos disponíveis para OpenGL em ambientes que os permitiriam & Não \\ \hline
\Cref{limitation:hardToUseWebGL} & Para construir jogos é muito difícil utilizar WebGL diretamente & Sim \\ \hline
\Cref{limitation:incompleteSupportWebgl} & Em muitas plataformas o suporte a WebGl é incompleto & Parcial \\ \hline
\Cref{limitation:limitedToRecentThingsWebgl} & WebGl não funciona em placas gráficas antigas  & Não \\ \hline
\Cref{limitation:soundAPIConflicts} & A API de Audio é problemática com sons frequentes no Android & Não \\ \hline
\Cref{limitation:bestAudioCompressionNotSupportedByAllBrowsers} & Não existe um único codec moderno suportado em todos os navegadores & Parcial \\ \hline
\Cref{limitation:limitedMultimidiaControlOnSafari} & O Safari conta com problemas de controle de execução de multimídia & Não \\ \hline
\Cref{limitation:noEffectsOnVideo} & Não é possível aplicar shaders diretamente em vídeo & Sim \\ \hline
\Cref{limitation:videoFullscreenControl} & Falta de forma padronizada de controlar o modo tela cheia para vídeos & Não \\ \hline
\Cref{limitation:videoCodecs} & Falta de consenso sobre um codec de vídeo padrão & \\ Não \hline
\Cref{limitation:safariVideoMissingControlAndCanvas} & A integração de vídeo e canvas do Safari é problemática  e vídeo só pode ser carregado pelo usuário & Não \\ \hline
\Cref{limitation:noSqlSupport} & Não existência de tecnologia oficial de SQL para Web & Parcial \\ \hline
\Cref{limitation:webStorageLimit} & Em muitos navegadores não é possível armazenar mais de 5MB em Web Storage & Não \\ \hline
\Cref{limitation:webStorageQueryLimit} & Não existe forma padronizada de descobrir quando espaço ainda resta em Web Storage & Parcial \\ \hline
\Cref{limitation:webStorageStringOnly} & Web Storage só permite armazenamento de  dados como texto & Não \\ \hline
\Cref{limitation:indexedDbAnonymousBehaviour} & O comportamento do IndexedDB não está especificado & Não \\ \hline
\Cref{limitation:indexedDbNoLike} & Não é possível realizar buscas em texto com o IndexedDB & Não \\ \hline
\Cref{limitation:noErrorMessagesOffline} & Erros nas configurações do manifesto offline não contém mensagens descritivas & Não \\ \hline
\Cref{limitation:orientationIsntReady} & A implementação da API de orientação é divergente entre os navegadores & Sim \\ \hline
\Cref{limitation:checkResourcesOnlyOnJavascrit} & Checagem de recursos só por JavaScript força implementação de markup em scripts & Não \\ \hline
\Cref{limitation:noCanvas2DIsnpectorOnipresent} & Não existe inspetor Canvas 2D multiplataforma para navegadores & Parcial \\ \hline
\Cref{limitation:multiTouch} & Scroll juntamente com outros gestos não funciona corretamente na maioria dos navegadores & Sim \\ \hline
\Cref{limitation:gamepadObject} & Não existe forma padronizada de detectar a entrada e saída de controles nos navegadores & Não \\ \hline
\Cref{limitation:desktopIcon} & Não existe forma padronizada de adicionar ícones de aplicativos Web a área de trabaho & Não \\ \hline
\Cref{limitation:monetizationDifferent} & Jogos exclusivamente Web dificilmente são monetizados da forma convencional & Parcial \\ \hline
\caption{Lista de problemas e limitações}
\label{table:technologies}
\end{longtable}

Das limitações dispostas no quadro \ref{table:technologies}
aquelas resolvidas parcialmente identificam que o problema
pode ser contornado em partes, mas também quando o problema é
contornado de forma inesperada e não intuitiva. Como a utilização
de um depurador internamente no código no caso da limitação
\Cref{limitation:noCanvas2DIsnpectorOnipresent}.

As limitações as quais não tenham solução, em muitos casos, são
problemas temporais e que podem ser solucionados com o tempo. Como no
caso da limitação \Cref{limitation:gamepadObject}, que as diferenças
se dão em grande parte por a especificação não estar pronta. Outras
limitações são intrínssecas da característica do HTML e pode
demorar muito para serem resolvidas.

Algumas possibilidades que são triviais em ambientes desktop podem
tornar-se fatores limitante na plataforma Web. Como é o caso do
controle de Fullscreen via JavaScript. Em plataformas nativas esse
controle é geralmente garantido, já na Web não se permite esse
tipo de gestão. %pesquisar mais sobre isso

Apesar de pretender resolver os problemas de múltiplas tecnologias em
multiplataformas, HTML5 introduz uma gama de incompatiblidade entre suas
próprias versões e implementações.

%}}}
\chapter{CONCLUSÕES}
%{{{
\thispagestyle{myheadings}

Neste trabalho revisamos tecnologias relevantes no desenvolvimento
de jogos multiplataforma em HTML5 e algumas das limitações a elas
relacionadas.

Para tanto, elaborou-se uma pesquisa bibliográfica
dos assuntos relevantes. Em seguida, criou-se um protótipo de jogo
para avaliar experimentalmente algumas das tecnologias e detectar
limitações. Apesar de simples, o protótipo desenvolvido ajudou
bastante na validação de algumas limitações previamente encontradas
e na detecção de alguns problemas adicionais os quais foram citados
previamente.

As limitações encontradas através do processo mencionado foram
registradas e avaliadas. Os resultados obtidos nos levam a crer que,
apesar de o número de limitações ser substancial, a grande maioria
destas limitações podem ser contornadas pelo programador ou já estão
no processo de serem solucionadas pelas especificações e navegadores.

As tecnologias da Web vem evoluindo a grande passo e número de pontos
positivos da abordagem HTML para desenvolvimento de jogos é muito
grande em comparação as limitações.

É seguro dizer que o HTML ainda não está pronto para substituir todas
as estratégias de desenvolvimento de jogos. Não obstante, as tecnologias
da Web podem ser aplicadas ao desenvolvimento de jogos em uma gama
crescente de casos.

No contexto multiplataforma, apesar de ser difícil de prover a
mesma experiência em duas plataformas diferentes, é possível
utilizar algumas plataformas com funcionalidades limitadas enquanto
outras recebem a experiência completa dos jogos \autocite[p.
1]{currentStateCrossPlatform}.

%provavelmente os problemas da Web mais difíceis de resolver são 
%os que conflitam com os princípios da mesma.

Da mesma maneira, existem várias funcionalidades interessantes para
jogos que ainda não estão prontas ou, por hora, não funcionam
em todos os navegadores como: WebVR, ES6, HTTP/2, Gamepad, entre
outras. Mas estes problemas, se considerarmos o histórico do HTML -
principalmente após o lançamento do HTML5, são passageiros. A figura
\ref{fig:htmlSupport} corrobora com esta hipótese, o crescimento do
suporte das tecnologias nesta imagem apresenta uma curva aparentemente
exponencial. Ao que tudo indica o futuro dos jogos em HTML5 parece
brilhante.

\subsection{TRABALHOS FUTUROS}

Visto que a especificação do HTML é viva, checar suas limitações
é uma tarefa que deveria ser feita de tempos em tempos. Tecnologias
como o EMACScript 6 e Web Assembly podem mudar completamente o cenário
de desenvolvimento de jogos Web assim que se tornarem opções viáveis,
comercialmente falando.

Seria interessante que assuntos como o WebGL e WebVR, levemente
abordados neste trabalho, fossem estudado com profundidade visto que
são deveras importantes para a criação de jogos cada vez mais
iterativos. O suporte a mais versões de navegadores e plataformas
bem como outras metodologias de desenvolvimento também poderiam ser
estudadas.

A utilização de plugins foi intencionalmente ignorada deste
trabalho. Não obstante, em jogos comerciais estas ferramentas são
imprescindíveis. Trabalhos que analisem a construção de um jogo sob
a perspectiva comercial, utilizando plugins, da forma mais aproximada
possível da realidade mercadológica, seriam igualmente interessantes.


%}}}

%Bibliograpy %{{{ 
\chapter{REFERENCIAS BIBLIOGRÁFICAS}
\markboth{}{}
\printbibliography[heading=none]
% \bibliographystyle{abnt}
% \bibliography{tcc}
\markboth{}{}
%}}}
\begin{appendices}
\chapter{Diagrama de classes}

A figura \ref{fig:fullDiagram} é o diagrama de classes detalhado do protótipo.
\begin{figure}[H]
    \centering
    \includegraphics[width=0.8\textwidth,natwidth=610,natheight=642]{ClassesFullView.png}
	\caption{Diagrama de classes completo}
    \label{fig:fullDiagram}
\end{figure}

\chapter{Bibliotecas relevantes no desenvolvimento de jogos Web}

A quantidade de ferramentas Web a disposição dos usuários é enorme.
Segundo \citet{html5mostwanted} desenvolvedores da Web são geralmente
de mente aberta e criaram uma variedade de bibliotecas e frameworks
pela internet. Sabendo disso, é uma tarefa praticamente impossível
ter habilidade em todas as ferramentas existentes. Entretanto, é
importante que os desenvolvedores ao menos conheçam as ferramentas
mais importantes disponíveis para que, no momento necessário, saibam
qual aprender. Todavia, \citet{creatingFun} ressalta a importância
de sermos moderados quanto a escolha de bibliotecas no contexto Web
multiplataforma.

\begin{quote}
Muitos desenvolvedores da Web utilizam bibliotecas como jQuery e o
Prototype de modo a se verem livres de ter que lidar com
partes triviais do desenvolvimento Web, como selecionar e
manipular elementos do DOM. Muitas vezes essas bibliotecas incluem
várias funcionalidades que não são utilizadas. É recomendável
cautela para verificar se realmente é necessário adicionar 50-100k
de bibliotecas, ou se alguma coisa mais simples e menor não trará
os mesmo benefícios, especialmente quando desenvolvendo
multiplataforma onde uma rápida conexão a internet nem sempre é
garantida.
\end{quote}

Esta preocupação aumenta no contexto de desenvolvimento
de jogos. Ainda segundo \citet{creatingFun} o site MicroJS
\url{https://microjs.com} oferece uma coleção de micro bibliotecas
focadas em áreas particulares em detrimento de grandes bibliotecas
cheias de funcionalidades.

\section{--prefix-free}

A biblioteca \textit{-prefix-free} \url{http://leaverou.github.io/prefixfree/}
é um polyfill que possibilita os desenvolvedores utilizarem CSS sem
a adição de prefixos, o que torna o trabalho de desenvolvedores bem
menos redundante. A biblioteca é razoavelmente leve (2KB), e quando o
sistema desenvolvido com ela estiver pronto, e o CSS evoluir removendo os
prefixos utilizados, ela pode ser removida sem ter que mudar o resto do código.

\section{Crosswalk} \label{crosswalk}

Crosswalk empacota os fontes juntamente com uma versão do Chromium e do
motor de renderização Blink. Isso faz com que o software se comporte
da mesma forma para todas as versões de dispositivos Android.
Novas versões do Crosswalk são disponibilizadas a cada 6 semanas
encorporando as últimas modificações do Chromium \autocite{crosswalk}.

\section{PhoneGap}

PhoneGap é um framework para desenvolvimento de sistemas híbridos que
empacota uma aplicação Web dentro de um contêiner nativo e provê um
conjunto de funcionalidades nativas via JavaScript. PhoneGap permite
acesso as APIs de câmera, geolocalização, contatos calendário,
etc \autocite[p. 3]{crossPlatformAppsAnimations}. Todos os sistemas
populares são suportados pelo PhoneGap: Android, IOS, Windows Phone,
etc. Diferentemente de outros frameworks, PhoneGap não provê uma
forma para criar funcionalidades de interface como animações e listas
\autocite[p. 15]{viabilityBusinessApplications}. O framework 
geralmente é combinado com outros para obter-se este tipo de funcionalidade.

PhoneGap também conta com um serviço para empacotamento online o
PhoneGap Build. Este serviço possibilita que se carregue de um arquivo
compactado contendo os fontes em um padrão especificado que o PhoneGap
Build se encarrega de gerar os binários para as plataformas requeridas.
Este recurso é valioso pois a alternativa é configurar o computador
de desenvolvimento para compilar para as plataformas alvo, um processo
entediante e complexo.

\section{Unity}

Unity é um motor de jogos multiplataforma utilizado para criar jogos
para PC's consoles, dispositivos móveis e Websites \autocite{unity}.
Mais do que um conjunto de códigos, o Unity disponibiliza também
um ambientes integrado de desenvolvimento onde é possível alterar
aspectos de jogos através de interfaces. Com Unity é possível
desenvolver em um dialeto JavaScript, entre outras linguagens, e
exportar para diversas plataformas. Para a Web o Unity possibilita
exportar em WebGl e Web Assembly, sendo uma combinação poderosa e que
pode garantir grande performance.

\section{Tree.js}

Treejs é um framework popular para o desenvolvimento em WebGL.
Consistem em uma abstração sobre WebGL que permite os autores se
focarem na criação de conteúdo para Web, ao invés de dispenderem
tempo manipulando os detalhes da WebGL. Possibilita trabalhar com
efeitos, luzes, cenas e outras abstrações em detrimento de shaders,
vértices, e conceitos primitivos.

\section{Appcelerator Titanium}

Appcelerator Titanium é um framework JavaScript que possibilita a
construção de aplicativos mobile nativos para Android, IOS e outras
plataformas. Titanium oferece uma vasta quantidade de APIs; sendo
bem documentadas, contendo descrições de métodos, parâmetros de
entrada e saída e algumas vezes exemplos de utilização \autocite[p.
2]{crossPlatformAppsAnimations}. A tecnologia também conta com uma IDE
especializada para o desenvolvimento em Titanium, enquadrando-se também
em um ambiente desenvolvimento de jogos.

\section{jQuery Mobile}

jQuery Mobile é um dos mais famosos frameworks mobile da Web, isso se
dá, parcialmente, pela popularidade do jQuery em si \autocite[p.
14]{viabilityBusinessApplications}.

\autocite[p. 2]{crossPlatformAppsAnimations} cita que:
\begin{quote}
j0uery prove uma grande game de APIs para muitos propósitos, por
exemplo adicionar ou remover elementos, gestão de eventos de clique,
manipulação de estilo,etc. Também provê APIs para animações, por
exemplo aparecer/desaparecer, etc, apesar de funcionar bem com desktops,
\end{quote}

jQuery Mobile não é um framework para todos as necessidades mobile.
Focando-se principalmente na interface; acesso ao hardware, instalação
nativas e outros aspectos do desenvolvimento multiplataforma
são responsabilidades do programador. jQuery mobile sofre com
problemas de performance em ambientes móveis, em partes por não
utilizar aceleração de hardware para criar suas interfaces,
como fazem alguns concorrentes como o Sencha Touch\autocite[p.
14]{viabilityBusinessApplications}.

\section{Kendo UI Mobile}

É um framework baseado em jQuery que provê componentes para a
construção de interfaces semelhantes as nativas através da
utilização de ferramentas da Web. Existem interfaces especializadas
para IOS, Android, Windows Phone e Blackberry \autocite{kendoui}.

\section{Sencha Touch}

Sencha Touch é um framework para desenvolvimento multiplataforma
que fornece um conjunto de componentes para criação de
interfaces gráficas, estruturas MVC e empacotamento. \citet[p.
14]{viabilityBusinessApplications} cita que o Sencha Touch é um dos
mais rápidos frameworks disponíveis. Com o Sencha Touch desenvolve-se
em JavaScript e nas demais tecnologias da Web e cria-se binários para
Android, IOS, Windows Phone, Tizen, etc.

\chapter{Ambientes de desenvolvimento de jogos em HTML}

\section{PlayCanvas}

PlayCanvas é uma plataformas para a construção de jogos 3D
na nuvem, desenvolvida com foco em performance. Permite a hospedagem,
controle de versão e publicação dos aplicativos nela criados,
possibilita também a importação de modelos 3D de softwares populares
como: Maya, 3ds Max e Blender;

\section{Intel® HTML5 Development Environment}

Fornece uma solução na nuvem, completa para o desenvolvimento em
múltiplas plataformas. Com serviços de empacotamento, serviços
para a criação e testes de aplicativos com montagem de interfaces
\textit{drag-and-drop} e bibliotecas para a construção de jogos
utilizando aceleração de hardware, o que garante até duas vezes mais
performance que aplicativos mobile baseados em Web tradicionais. Esta
solução é gratuita, open-source e funciona através de um plugin
para o Google Chrome. Devido ao fato de os binários ficarem hospedados
na nuvem, possibilitou a Intel criar compiladores para cada uma das
plataformas disponibilizadas pelo PhoneGap.

\chapter{Tecnologias de Compilação multiplataforma}

\section{GWT}
Segundo \citet[p. 29]{gwt}
\begin{quote}
O GWT é um framework essencialmente para o lado do cliente que dá
suporte à comunicação com o servidor através de RPCs (\textit{Remote
Procedure Calls}). Ele não é um framework para aplicações clássicas
da Web, pois deixa a implementação da aplicação Web parecida com
implementações em desktop.
\end{quote}

Com o GWT se programa em Java e exporta-se com código otimizado para
as plataformas alvo. Sua licença é Apache 2 e, como o framework é em
Java, pode ser rodado em todos os sistemas operacionais.

\section{Libgdx}

Libgdx \url{https://libgdx.badlogicgames.com/} é um framework
focado no desenvolvimento de jogos nativos multiplataforma. O
código é escrito uma vez e a aplicação pode ser portada para
todas as plataformas sem nenhuma modificação \autocite[p.
8]{crossPlatformMobileGameDevelopment}. A linguagem do Libgdx é Java, e
é possível compilar para Android, IOS, HTML5, entre outros. Também é
possível desenvolver em todos os desktops comuns.

\chapter{Sistemas de Building}

Segundo \citet{gruntTutorial}
\begin{quote}
Durante o desenvolvimento de aplicações Web, existem muitas tarefas
que tem que ser feitas repetidamente. Estas tarefas incluem minificar o
JavaScript e CSS, rodar testes unitários, aplicar \textit{linters} nos
arquivos para checar erros, compilar os pré processadores de CSS (LESS,
SASS), e muito mais.
\end{quote}

\section{Grunt}

O Grunt é um automatizador destas tarefas quotidianas. Para realizar
a automação o Grunt conta com um grande ecossistema de plugins que
podem ser compostos para realizar as tarefas desejadas para determinada
aplicação.

No protótipo o Grunt foi utilizado com o intuito primário de realizar
a minificação e disponibilização da aplicação como um conjunto de
arquivos prontos para a produção.

Para tanto os seguintes plugins foram utilizados:
\begin{itemize}
    \item grunt-contrib-uglify: responsável por minificar os arquivos JavaScript;
    \item grunt-contrib-cssmin: responsável por minificar arquivos CSS;
    \item grunt-contrib-copy: responsável por copiar os arquivos de desenvolvimento para a pasta de distribuição;
    \item grunt-contrib-watch: observar modificações nos arquivos de desenvolvimento e iniciar o processamento dos plugins acima;
\end{itemize}

\section{Gulp}

O Gulp utiliza o conceito de \textit{streams} para aplicar todas as
modificações sobre um arquivo de uma vez só.

\citet{crossPlatformMobileGame} afirma que:
\begin{quote}
Essencialmente, Gulp trabalha com streams de arquivos para dentro e fora
dos módulos do node possibilitando que a aplicação se transforme
em uma variedade de combinações. Uma vez que a transformação é
completa, a saída pode ser escrita em um arquivo no sistema. Essa forma
de trabalhar com streams é possibilitada pelo Node.js, sendo altamente
eficiente comparado com a maioria dos sistemas de build, visto que não
existem arquivos temporários para serem escritos em disco (tudo é
feito em tempo de execução utilizando objetos de virtuais).
\end{quote}


\end{appendices}
%}}}

\end{document}
