%{{{
\documentclass[11pt,a4paper]{article}
\author{Jean Carlo Machado}
\title{}
\usepackage[top=3cm,left=3cm,bottom=2cm,right=2cm]{geometry}
\pagestyle{plain}
\begin{document}
%}}}

\chapter{CONTEXTUALIZAÇÃO}
%{{{
\section{JOGOS}

\subsection{BENEFÍCIOS}

São variadas as pesquisas que apontam os benefícios dos jogos
eletrônicos para as pessoas, coordenação motora, maior
concentração.

Muitos jogos ajudam a desenvolver habilidades práticas, servem como uma
forma de exercício, ou de alguma forma executam um papel educacional,
de simulação ou psicológico. Jogos são uma parte universal do
experiência humana, presentes em todas as culturas.

\subsection{O MERCADO}
No início os jogos nativos dominavam...
Estatísticas sobre os jogos...
A muito a indústria de jogos superou a cinematográfica...

Com a ascensão dos dispositivos inteligentes a massiva quantidade de
dispositivos se tornou atraente para a indústria e consumidores de
jogos.

A maior dificuldade em capturar uma base de usuários é que o mercado
de dispositivos móveis é muito fragmentado e não existe uma única
plataforma popular. (HASAN, 2012)

\subsection{JOGOS E MULTIPLATAFORMA}

É laboriosa a tarefa dos produtores de software em um panorama tão
diversificado como o atual, existem muitas plataformas, muitas versões
e hardwares diferenciados.

Uma alterativa para mitigar os problemas oriundos da multiplicidade de
plataformas é o HTML.

\subsection{HTML E MULTIPLATAFORMA}

Desenvolvedores de jogos web podem rapidamente satisfazer as
necessidades de seus jogadores, mantendo-os leais a tecnologia HTML5
(ZHANG, 2012).

>A maioria dos desenvolvedores demonstra interesse para o HTML5 

Entretanto, o HTML em sua especificação e implementações atuais
costa com algumas limitações que precisam ser compreendidas por
aqueles interessados em criar jogos em HTML5.

> O tempo de desenvolvimento de uma aplicação em HTML5 é 67% menor
que aplicações nativas. Isso mostra o custo efetivo de aplicações
baseadas em HTML5. A real vantagem de aplicações em HTML5 é o suporte
horizontal entre as plataformas - que é a maior razão por trás do
custo efetivo. (HASAN et al, 2012)

\subsection{LIMITAÇÕES DE JOGOS MULTIPLATAFORMA COM HTML5}

O HTML vem sendo desenvolvido por muitos anos e por pessoas que não
conheciam umas as outras, muitas funcionalidades foram construídas de
maneiras inconsistentes.

> Funcionalidades foram disponibilizadas de diversas fontes e não foram
construídas de forma especialmente consistente com as demais. Além
disso, devida a única característica da Web, erros de implementação
se tornam frequentes, e muitas vezes se tornam o padrão, pois outras
funcionalidades dependem destas primeiras antes que elas estejam
estáveis. (W3C manual)

Enquanto o HTML é desenvolvido muitas das funcionalidades
disponibilizadas são testadas em apenas um pequeno conjunto de
navegadores para um pequeno conjunto de versões (referência 2). Isso
acarreta em suporte inconsistente. A forma mais segura de garantir
suporte é testando em todas as versões alvo, todavia essa solução
não é prática. (ref. 2)

Os desenvolvedores de navegadores podem interpretar/implementar
as especificações erroneamente aumentando os problemas de
compatibilidade.

Nem todos os recursos disponíveis através das SDK's nativas estão
presentes através do HTML5.

\section{ ESTE TRABALHO}

Este projeto propõe analisar as limitações do HTML5 quanto relativo
a construção de jogos multiplataforma. Através de revisão
bibliográfica e da criação de um protótipo de jogo multiplataforma.

Um tratado completo sobre o assunto requiriria um comparativo entre
jogos desenvolvidos nativamente e jogos em HTML5.

Não é objetivo deste trabalho demonstrar onde o HTML5 se sobressai,
apenas suas limitações. Também não é objetivo deste trabalho
comparar o HTML com outras
tecnologias de desenvolvimento de jogos, como Flash Player, Silverlight
ou alternativas Desktop.

\subsection{O JOGO}

Para a análise prática das limitações foi escolhido um jogo de
matemática simples. Consistindo na geração de equações com um
candidato de resposta. Cabe ao usuário informar se o resultado apontado
pelo jogo está correto ou não.

Porquê escolhi esse tipo de jogo?
%}}}

\chapter{PROBLEMA}
%{{{
A carência de definições concretas sobre a viabilidade da atual
versão do HTML5 - quando utilizado no desenvolvimento de jogos e o
senso comum, acabam por monopolizar à construção de jogos nativos as
plataformas alvo.

Os custos introduzidos no ciclo vida de um jogo, para diversas
plataformas, é muito alto para ser considerado trivial. Cerca de 65%
mais altos (segundo trabalho 2)
%}}}

\section{ OBJETIVOS}
%{{{

Abaixo seguem os objetivos deste trabalho.

\subsection{  OBJETIVO GERAL}

Identificar possíveis limitações no processo de desenvolvimento
de jogos multiplataforma oriundas do atual estado de definição e
implementação do HTML5.

\subsection{  OBJETIVOS ESPECÍFICOS}

Estudar as limitações de desenvolvimento de jogos nas plataformas de
dispositivos inteligentes Android e navegadores Desktop Google Chrome 42
e Firefox 37. Optamos por Android, e não IOS, pois o primeiro contém
a vasta maioria do mercado de dispositivos inteligentes, e por termos
maior experiência na já mencionada plataforma.

Pretende-se também estudar os seguintes tópicos do desenvolvimento de
jogos, relativos ao HTML5:

- Depuração
- Diferenças em tamanho de tela
- Canvas
1. Troca de tamanhos via Canvas vs DOM
2. Aceleração de GPU
3. API de Áudio (referência 2)
- Performance
- Empacotadores HTML5
- Eventos de entrada
- Vibração
- Acelerômetro
- Armazenamento
- Disponibilização de assets (controle de tamanhos, cache, etc)
- Aplicações offline
- CSS media queries

Elaborar uma lista de limitações e correlacionar os dados de acordo
com as plataformas.

<!-- Pensamento: talvez seja interessante concluir se é viável
produzir jogos com HTML5 do ponto de vista mercadológico --> %}}}

\chapter{JUSTIFICATIVA}
%{{{

Tendo em vista que este trabalho busca mapear possíveis problemas
do desenvolvimento multiplataforma em HTML ele serve para apoiar
e justificar decisões relativas ao desenvolvimento de jogos
multiplataforma; Por tratar cientificamente de aspectos importantes do
HTML, este trabalho tem potencial apontar os pontos chave que necessitam
de melhorias nas plataformas alvo, colateralmente colaborando para a
melhoria do próprio HTML.
A opinião comum tende para soluções nativas em detrimento do
desenvolvimento de jogos, este trabalho pretende desafiar esta
concepção. (REFERENCIAR) Muitos desenvolvedores estão familiarizados
com as tecnologias da WEB ou apontam interesse na tecnologia. <!--
referenciar -->
Estimular e avançar o estudo da implementação da Open Web;
%}}}

\chapter{REVISÃO BIBLIOGRÁFICA}
%{{{

\section{JOGOS}
%{{{
Segundo LEMES (2009, pág. 126) > jogo digital constitui-se em uma
atividade lúdica composta por uma série de ações e decisões,
limitada por regras e pelo universo do game, que resultam em uma
condição final.

Essa característica interativa é a dependência comandos
sobre uma interface digital, que faz com que o projeto digital desta
natureza não seja um filme ou uma animação, e sim um game.

Quando desenvolvendo qualquer jogo, o desenvolvedor tem que considerar
seu usuário. O objetivo é maximizar a satisfação de seu usuário.
Jogos em plataformas móveis trazem um novo conjunto de desafios para
produtores de jogos. Um destes desafios é fornecer feedback suficiente
para o player pois o dispositivo é limitado em proporções, som, tela
etc. Já jogos multiplataforma em HTML5 tem a dificuldade adicional
de ter que comportar, na mesma base, o feedback adequando para cada
plataforma móvel.

A interface tem que ser o mais intuitiva o possível. No caso de       
dispositivos móveis, quanto menos gestos necessários melhor Tornar   
previsível causa e efeito é uma boa característica para os jogos    
Os desenvolvedores tem que evitar fazer o jogo para eles mesmos. E     
pela falta de crítica os designs tendem a ser ruins. Afinal o que os  
jogadores querem? LEMES (2009, pg XX) aponta alguns fatores procurados
pelos usuários de jogos: Desafio, socializar, experiência solitária,.
respeito e fantasia                                                    .

Mencionar algum jogo (como WOW) e como ele faz para prender a
atenção dos usuários. Candy crush saga 

\subsection{GÊNEROS}

LEMES (2009, p. 43) aponta os seguintes gêneros de jogos.

- Adventure Ação RPG Estratégia Simuladores Esportes Luta Casuais
- 'God' Games Educacionais Puzzle Online / Massive Multiplayer

\subsection{MECÂNICA}

A mecânica é composta pelas regras do jogo. Quais as ações
disponíveis aos usuários, é fortemente influenciada pela categoria do
jogo em questão.

\subsection{ARQUITETURA}

Existem algumas estratégias relativas às plataformas alvo de como
efetuar construção de jogos.

\subsection{DESENVOLVIMENTO DE JOGOS NATIVOS}

Habilita a melhor experiência de usuário pois permite utilizar ao
máximo os recursos e funcionalidades dos aparelhos. Porém, devido
a cada plataforma conter seu próprio sistema operacional, com seus
próprios *SDK's* totalmente incompatíveis, os desenvolvedores são
forçados a desenvolver uma versão do jogo para cada plataforma alvo.
Além da replicação dos fontes, esta abordagem requer mais pessoas, e
maior custo com possivelmente parte do mercado não atendido de qualquer
forma.

\subsection{DESENVOLVIMENTO DE JOGOS WEB}

Necessitam de apenas uma base de código e pode rodar em todas as      
plataformas. Contém a mais vasta gama de desenvolvedores e muitos      
interessados em aprendê-la. Seus custos também são inferiores, aos  
do desenvolvimento nativo – pois demandam menos trabalhadores/hora   
devido a inexistência de duplicação da base. Não obstante, esta     
opção – devido a incompletude da especificação de padrões       
– carece de alguns recursos e outros não estão completamente       
implementados. Performance também pode ser um limitador, visto que    
estas tecnologias são executadas através de um navegador, criando uma
camada de abstração superior à das API's nativas que fazem chamadas 
ao sistema diretamente.                                                 

\subsection{DESENVOLVIMENTO DE JOGOS HÍBRIDOS}

Jogos híbridos são jogos geralmente desenvolvidos com tecnologias da .
web: beneficiando-se da não necessidade de duplicação. Rodam dentro .
de um *container* nativo – possibilitando o acesso à chamadas do    .
sistema, recursos de hardware, eliminando muitias das dificuldades da  .
web Em certo sentido, beneficiam-se do melhor de ambas as metodologias .
anteriores Phone game é uma ferramente deste tipo. Permite acessar    .
os dispositivos utilizando sua API JavaScript. Funciona encapsulando   .
todo o código HTML5. Este tipo de abordagem permite acessar câmera,  .
acelerômetro, GPS, etc                                                .
%}}}

\section{WEB}
%{{{

\subsection{OPEN WEB}

Mais do que um conjunto de tecnologias Open Web, é um conjunto de
filosofias as quais a web se baseia.

Neste conjunto inclui-se:

- Descentralização;
- Transparência;
- Relevância;
- Imparcialidade;
- Consenso;
- Disponibilidade;
- Manutibilidade;

<!-- Ref
-->

A OWP (Open Web Platform), uma coleção de tecnologias livres,
amplamente utilizadas e padronizadas as quais adotam a postura da Open
Web. Quando uma tecnologia se torna amplamente popular, através da
adoção de grandes empresas e desenvolvedores ela se torna candidata a
adoção pela OWP.

A tecnologia chave que inaugurou e alavancou este processo é o HTML. 
%}}}

\section{HTML}
%{{{

*Hyper Text Markup Language* (HTML) é uma linguagem de marcação
que define a estrutura semântica do conteúdo das páginas da
web, criada por Tim Berners Lee e oficialmente mantida pela W3C.
O "Hyper Text" refere-se a links que conectam páginas umas as
outras, fazendo a Web como conhecemos hoje (MDN, 2015). <!--
https://developer.mozilla.org/en-US/docs/Web/HTML --> HTML foi
especificado baseado-se em SGML *Standard Generalized Markup Language*,
abraçando assim suas premissas:

- Deve ser declarativo, descrevendo estrutura e outros atributos,
- Deve ser ao invés de definir o processamento a ser efetuado no
- Deve ser documento: rigoroso de modo que as mesmas técnicas de
- Deve ser mineração de dados em objetos e bancos de dados possam ser
- Deve ser utilizadas;

Não obstante, os criadores de navegadores constantemente introduziam
elementos de apresentação com o *blink*, *<i>* itálico, *<b>* bold,
que eventualmente acabavam por serem inclusos na especificação. Foi
somente nas últimas versões que elementos de apresentação voltaram
a ser proibidos reforçando as propostas chave HTML como uma linguagem
de conteúdo semântico, abrindo espaço para a expansão de outras
tecnologias como o CSS para responder as demandas de apresentação.

A atual versão do HTML é o HTML5, desenvolvido como um trabalho em
conjunto entre a WHATWG e a W3C, seu rascunho foi proposto em 2008 e
apenas ratificado em 2014. O HTML5 introduziu elementos interativos que
viabilizaram a construção de jogos para a plataforma como: Canvas,
áudio, vídeo.

Cada elemento HTML informa algo ao navegador sobre a informação que
reside entre o abrir e fechar da tag. <!-- ducket pág 20 -->

Um elemento é o abrir fechar de uma tag e todo o conteúdo que dentro
dele reside.<!-- ducket pág 24 -->

O HTML5 é muitas vezes interpretado como um conceito guarda chuva para
designar as tecnologias da web HTML, JavaScript e CSS3.

%}}}

\section{CSS}
%{{{

É uma linguagem de folhas de estilo, criada por Håkon Wium Lie em
1994, com intuito de definir a apresentação de páginas HTML.

>  Possibilitam ligação tardia *late biding*. Essa característica
é atrativa para os publicadores por dois motivos. Primeiramente pois
permite o mesmo estilo em várias publicações, segundo pois os
publicadores podem focar-se no conteúdo. Lie

O CSS é dividido em módulos, contendo aproximadamente 50 deles, cada
qual evoluindo separadamente.

Sua última versão, o CSS3, introduziu várias funcionalidades
relevantes para jogos, como *media-queries*:possibilitam regras para
tamanhos de tela, transformações 3D e animações.

Os navegadores interpretam CSS através da tag ``<style>``.

<!-- Cascading Style Sheets, PhD thesis, by Håkon Wium Lie – provides
an authoritative historical reference of CSS pág 23 -->

<!-- Muitos navegadores também suportam aceleração de GPU (Unidade de
processamento gráfico) para elementos que tenham transformações 3D.
-->

<!!-- Flow de documento, ordem e posição em que os elementos tem que
aparecer na página. Modelo de caixa o que encapsula o conteúdo em um
elementos. -->

<!-- falar do suporte a variáveis do CSS -->

%}}}

\section{JAVASCRIPT}
%{{{

EMACScript, melhor conhecido como JavaScript, criada por Brendan Eich
em 1992, é a linguagem da Web. Devido a tremenda popularidade entre
comunidade de desenvolvedores a linguagem foi abraçada pela W3C e
atualmente é um dos componentes da *Open Web Platform*.

As definições da linguagem são descritas na especificação ECMA-262.
Esta possibilitou o desenvolvimento de outras implementações além da
original - *SpiderMonkey* - como o Rhino, V8 e TraceMonkey; bem como
outras linguagens similares como JScript da Microsoft e o ActionScript
da Adobe.

JavaScript é uma linguagem de script. Segundo a Ecma Internacional
2012:

> "Uma linguagem de script é uma linguagem de programação que é
usada para manipular e automatizar os recursos presentes em um dado
sistema. Nesses sistemas funcionalidades já estão disponíveis
através de uma interface de usuário, uma linguagem de script é
um mecanismo para expor essas funcionalidades para um programa
protocolado."

A intenção original era utilizar o JavaScript para dar suporte aos já
bem estabelecidos recursos do HTML, como para validação, alteração
de estado de elementos, etc. Em outras palavras, a utilização do
JavaScript era opcional e as páginas da web deveriam continuar
operantes sem a presença da linguagem.

Entretanto, com a construção de projetos Web cada vez mais complexos,
as responsabilidades delegadas ao JavaScript aumentaram a ponto que a
grande maioria dos sistemas web não funcionarem sem ele. Não obstante,
JavaScript não evoluiu ao passo da demanda e muitas vezes carece de
definições expressivas, completude teórica, e outras características
de linguagens de programação mais bem estabelecidas, como o C++ ou
Java (Barnett, 2013). A nova versão do JavaScript, o JavaScript 6, é
um esforço nessa direção. JavaScript 6 ou *EMACScript Harmonia*,
contempla vários conceitos de orientação a objetos como classes,
interfaces, herança, tipos, etc.

Estes esforços de padronização muitas vezes não são rápidos
o suficiente para produtores de software web, demora-se muito até
obter-se um consenso sobre quais as funcionalidades desejadas em
determinada versão e seus detalhes de implementação. Outrossim, uma
vez definidas as especificações, é necessário que os distribuidores
do JavaScript implementem o especificado.

Alternativamente, existe uma vasta gama de conversores de código -
*transpilers* - para JavaScript; possibilitando programar em linguagens
formais e posteriormente gerar código JavaScript. Não obstante,
essa alternativa tem seus pontos fracos, necessita-se de mais tempo
de depuração , visto que o JavaScript gerado não é conhecido
pelo desenvolvedor, e provavelmente o código gerado não será tão
otimizado, nem utilizará os recursos mais recentes do JavaScript.

Mesmo com suas fraquezas amplamente conhecidas, JavaScript está
presente em praticamente todo navegador atual. Sendo uma espécie de
denominador comum entre as plataformas. Essa onipresença torna-o
integrante vital no processo de desenvolvimento de jogos multiplataforma
em HTML5. Vários títulos renomeados já foram produzidos que fazem
extensivo uso de JavaScript, são exemplos: Candy Crush Saga, Angry
Birds, Dune II, etc.

Jogos Web são escritos na arquitetura cliente servidor, JavaScript pode
rodar em ambos estes contextos, para tanto, sua especificação não
define recursos de plataforma. Distribuidores do JavaScript complementam
a o JavaScript com recursos específicos para suas plataformas alvo.
Por exemplo, para servidores, define-se objetos de terminal, acesso a
arquivos e dispositivos, etc. No contexto de cliente, são definidos
objetos como janelas, frames, DOM, etc.

Para o navegador o código JavaScript geralmente é disposto no elemento
``script`` dentro de arquivos HTML. Quando os navegadores encontram esse
elemento eles fazem a requisição para o servidor e injetam o código
retornado no documento, e a não ser que especificado de outra forma,
iniciam sua execução.

\subsection{ ASM.JS}

Asm.js é um subconjunto da sintaxe do JavaScript a qual permite grandes
aumentos de performance quando em comparação com JavaScript normal.
No contexto dos jogos performance é usualmente um recurso estimável,
asm.js consegue-o supra utilizando recursos que permitam otimizações
antes do tempo *ahead of time optimizations*. Entretanto, não é
trivial escrever código em asm.js e geralmente a geração de código
asm.js é feita através da transpilação de outras linhagens como C.

> Muita da performance adicional em relação ao JavaScript é devido
a consistência de tipo e a não existência de um coletor de lixo
(memória é gerenciada manualmente através de um grande vetor). Esse
modelo simples desprovido de comportamento dinâmico, sem alocação
e desalocação de memória, apenas um bem definido conjunto de
operações de inteiros e flutuantes possibilita grade performance e
abre espaço para otimizações.

\subsection{AJAX}

\subsection{ ALTERNATIVAS AO JAVASCRIPT}

Abaixo seguem algumas tecnologias que servem de alternativa ao
JavaScript.

\subsection{TYPESCRIPT}

Conhecido como uma versão estendida do JavaScript que compila para
JavaScript normal.

\subsection{DART}

Google. DartVM é uma máquina virtual que está embebido no Google
Chrome. Significante melhorias em performance quando comparado
ao JavaScript. Existe o dart2js que compila código em Dart para
JavaScript.
%}}}


\section{ DOCUMENT OBJECT MODEL (DOM)}

É uma plataforma e interface agnóstica a linguagem que permite os
programas e scripts dinamicamente acessar e atualizar o conteúdo,
estrutura e estilo de documentos. Pode ser novamente processado e o
resultado aparecer na tela. O navegador cria um DOM quando ele processa
os elementos e tags encontrados em um documento HTML. Gmail é uma
aplicação de única página (single-page) que se baseia fortemente no
DOM para gerar conteúdo dinâmico e interativo oferecido pelo DOM.

\subsection{CANVAS}

A nova tag <canvas> define um layer gráfico em documentos HTML que pode
ser desenhado através de JavaScript.

Permite desenhar diagramas, gráficos e animações [7]. É baseado em
bitmap.

O suporte ainda é escasso.

Muitas vezes lento. Algumas soluções tentam arrumar isso através da .
utilização de GPU Apache Cordova utiliza o FastCanvas                .

CocoonJS é uma aplicativo híbrido que preenche a fraca implementação
de OPENGL nos dispositivos móveis possibilitando se desenvolver em
WEBGL.

\section{WEBGL}

Baseado no OpenGL.

Web GL não foi utilizada no trabalho apesar de ser de grande
relevância no processo de jogos pois ainda não está completamente
especificada e a dificuldade e escopo do projeto aumentariam muito se
tivessem de incluir um jogos 3D. Versão da especificação atual?

\section{VIDEO}

\section{AUDIO}

Audio é um componente vital para oferecer grande satisfação aos
usuários de jogos. Provê feedback e imerge o usuário. Efeitos de som
e música podem servir como mecanismo. Jogadores tem baixa tolerância a
volume, deve ser utilizado com cautela.

\subsection{TAG AUDIO}

A tag <audio> define um som dentro de um documento html. Quando o
elemento é renderizado pelos navegadores, ele carrega o conteúdo
que pode ser reproduzido pelo player de audio do navegador. Existem
muitas discrepâncias entre os formados aceitáveis pelos navegadores.
È um tanto limitada quanto comparada ao áudio de múltiplos canais
disponibilizados por SDKs nativas.

\subsection{API DE AUDIO}

É uma interface de audio experimental para JavaScript. Provê maior
flexibilidade na manipulação de audio. Essa tecnologia é muito mais
nova do que a tag audio. 

FORMATOS DE ÁUDIO

\section{CAMERA}

\section{ ENTRADA DE COMANDOS}

Na construção da grande maioria dos jogos é muitas vezes
imprescindível alta flexibilidade na gestão de entrada de dados.
Este fator muito se amplia na criação de jogos multiplataforma,
seja através de teclado, tela sensível ou sensor de movimentos, o
importante é oferecer a melhor experiência possível por plataforma.
O HTML5 trata todos estes casos abstratamente na forma de eventos, os
quais podem ser escutados através de listeners. Os eventos básicos
são: keydown (tecla baixa), keyup (tecla solta) e keypress (tecla
pressionada).

Para detectar suporte aos mais variados recursos do HTML5 no navegador
do cliente existem duas possibilidades. Pode-se implementar testes para
cada funcionalidade utilizada abordando os detalhes de implementação
de cada uma ou então fazer uso de alguma biblioteca especializada
neste processo, o Modernizr é uma opção open-source deste tipo de
biblioteca, este gera uma lista de booleanos sobre grande variedade dos
recursos HTML5, dentre estes, geolocalização, canvas, áudio, vídeo e
local storage.

\section{CACHE}

Aplicações offline.

Algumas tecnologias desta classe são:

\subsection{ OFFLINE E ARMAZENAMENTO}

> Uma das grades limitações do HTML era a ausência de capacidade
de armazenamento de dados. Armazenamento no lado do cliente é um
requerimento básico para qualquer aplicação moderna. Essa área
era ode as aplicações nativas detinham grande vantagem sobre as
aplicações web. O HTML5 solucionou este problema introduzindo várias
formas de armazenamento de dados. (HASAN et all, 2012)

\subsection{ LOCAL STORAGE}

Também conhecido como WebStorage na especificação do HTML5. Provê
uma forma de armazenar os dados como chave valor dentro do navegador. Os
dados são persistido mesmo que o navegador seja fechado.

\subsection{WEB SQL}

Simplesmente um banco de dados SQLite embebido no navegador. Permite
tabelas relacionais. O tamanho padrão do banco de dados é 5 megabytes
e pode ser estendido pelo usuário.

\subsection{ RECURSOS NATIVOS ATUALMENTE INDISPONÍVEIS PARA O HTML5}

- Suporte à câmera;
- Suporte à calendário;

\subsection{DEBUG}

\subsection{ WEINRE}

 5.21  TECNOLOGIAS POLYFILL
Acarretando assim, que muitos navegadores não implementam algumas
funcionalidades, completa ou parcialmente especificadas, daí surge a
necessidade dos polyfills (tecnologias de preenchimento de lacunas) para
implementar estas camadas.

Uma das soluções mais promissoras polyfill é o PhoneGap ou Apache
Cordova, esta ferramenta é Open-source e possibilita utilizar de
inúmeros recursos de hardware da grande maioria das produtoras de
dispositivos móveis.

\subsection{FERRAMENTAS}

\subsection{NODEJS}

Permite rodar JavaScript fora do navegador. Utiliza um modelo dirigido
à eventos sem bloqueio, tornando-o rápido e eficiente.

\subsection{SISTEMAS DE BUILDING}

Aquivos JavaScript são requisitados do servidor assincronamente. Isso
pode levar a tempos de requisição pouco desejáveis. Uma saída seria
escrever o código em apenas um arquivo mais isso leva a gerência de
código bagunçada. A saída mais comum entre desenvolvedores é utiliza
ruma ferramenta que junta todos os arquivos e disponibiliza apenas um
para o usuário.

\subsection{GRUNT}

Aplica as modificações separadamente em cada arquivo.

\subsection{GULP}

Utiliza o conceito de streams para aplicar todas as modificações sobre
um arquivo de uma vez só.

\subsection{SOURCE MAPS}

Para encontrar os arquivos minificados a fim de ajudar o desenvolvedor a
debugar a aplicação.

\subsection{Minificar}

Remover caracteres desnecessários do JavaScript como espaços vazios,
diminuindo o tamanho dos nomes, fazendo o tempo de loading diminuir.

\subsection{GERENCIADORES DE PACOTES}

\subsection{BOWER}


\subsection{NPM}


\subsection{DISPONIBILIZAÇÂO DA APLICAÇÂO}

Links com manifestos

\section{ INSTALAÇÃO}

Este método é benéfico pois possibilita ao usuário a mesma
experiência ao adquirir uma aplicação normal. Este tipo de
aplicação é comummente referido como "híbrido".

\section{ CROSSWALK}

Crosswalk empacota os fontes juntamente com uma versão do Chromium, a
versão Open-source do Google Chrome. Isso faz com que o software se
comporte da mesma forma para todas as versões de dispositivos Android.

\section{ PHONEGAP}

\section{ PHONEGAP CLOUD}

Este serviço possibilita que se faça upload de um arquivo compactado
contendo os fontes – ou apontando para um repositório no GitHub –
que no tempo desta pesquisa não estava funcionando; e se gere o APK
para o Android nativamente.

\subsection{O JOGO}

Devido ao fato deste trabalho explorar as limitações dos jogos em
HTML5, optei por evitar a utilização de plugins e ferramentas de
terceiros que pudessem ocultar alguma limitação.

Escolhi a simplicidade para não precisar ficar muito tempo aprendendo
as coisas em detrimento do refinamento da pesquisa.

\section{MECÂNICA}

O jogo consiste em simplesmente em uma tela que apresenta equações e
um possível resultado. Cabe ao jogador decidir se o resultado está
certo ou errado. O tempo é um fator levado em consideração, quão
mais rápido o jogador acertar se a afirmação está correta ou não,
mais pontos ele receberá.

Argumentos à favor da escolha do game: Tem profundidade, permite a
adição de novos recursos no futuro;
É facilmente traduzível em tamanhos de telas diferentes e tipos de
entrada de dados diferentes;

\section{IMPLEMENTAÇÃO}

Não tenho grande experiência com o desenvolvimento de jogos nem com
o desenvolvimento em HTML5. Também para não interferir na pesquisa
busquei não me distanciar do que é considerado padrão em ferramentas
e métodos.
Comecei escrevendo o aplicativo para o Navegador do desktop pois era o
que estava mais acessível no momento. Mais tarde descobri que de fato
é assim que de desenvolve.

\section{ TRABALHOS SIMILARES}

(Referência 2) Faz uma revisão de aspectos do HTML5 através da
construção de um jogo. O autor foca muito nos aspectos de criação
de jogos e feedback do desenvolvimento. Troca de tecnologias e não
especificamente nas limitações conforme o meu trabalho. Em outras
palavras seu escopo é mais genérico e não tão preciso quanto este

\section{ANDROID}
%{{{

É um sistema operacional *open-source* desenvolvido pela Google.
Utiliza o kernel Linux .
Softwares para Android são geralmente escritos em Java e executados
através da máquina virtual Dalvik.

É similar a máquina virtual Java, mas roda um  .
formato de arquivos diferenciado (dex), otimizados para consumir pouca .
memória, que são agrupados em um único Android Package (apk) Android.
permite a renderização de documentos HTML através de sua própria   .
API WEBVIEW. Ou através do navegador disponibilizado por padrão, ou  .
outros de terceiros como o Google Chrome, Firefox, Opera, etc          .

No quesito jogos para dispositivos móveis é preferível disponibilizar
os jogos através da interface nativa pois dá a sensação de
continuidade para com os demais aplicativos instalados no dispositivo.

%}}}


\section{NAVEGADORES}
%{{{

Aplicações do lado do cliente geralmente se comunicam com um
servidor através de documentos em HTTP. Quado o navegador recebe um
destes pacotes em HTML ele começa o processo de renderização. A
renderização pode requisitar outros arquivos a fim de completar a
experiência desenvolvida para o endereço em questão.

Nos navegadores os usuários necessitam localizar a página que desejam,
sabendo o endereço, ou pesquisando em buscadores. Isso é um processo
árduo para a plataformas móveis pois necessitam maior interação
do usuários e não são “naturais” se comparado ao modo normal
de consumir aplicativos nestas mesmas plataformas – simplesmente
adquirindo o aplicativo na loja e abrindo-o no sistema operacional.
Alguns contornos para este problema serão descritos nas tecnologias
offline.

Para transformar as instruções retornadas pelo servidor em algo útil
para o usuário final os navegadores geralmente fazem uso de bibliotecas
externas capazes de interpretar HTML5 e gerar o conteúdo iterativo.

\subsection{BIBLIOTECAS WEB}

O Google Chrome utiliza o Webkit para renderizar seu conteúdo HTML5. O
webkit foi criado pela Apple baseando-se no motor de renderização do
Konkeror do projeto KDE. Safari e Opera também fazem uso do Webkit. V8
para JavaScript.

O motor de renderização do HTML5 do Firefox é o XXX. O motor de
JavaScript é o.

%}}}

\chapter{METODOLOGIA}
%{{{

Este trabalho procura investigar a estratégia web, sendo assim não
será explorado em detalhes as alternativas nativas.

Também foi dada primazia ao HTML puro, pois ao utilizar frameworks
muitos dos problemas da web podem já serem resolvidos por essas
ferramentas e as limitações ficarem escondidas.


O primeiro passo consiste em definir as plataformas alvo do trabalho;
devem ser plataformas mercadologicamente relevantes ao desenvolvimento
de jogos, que possibilitem a criação de aplicativos em HTML e que
acentuem o antagonismo de características.

Segue-se com a construção de uma lista com os recursos relevantes
aos jogos que, empiricamente, sofrem ou são comummente ligados à
limitações multiplataforma. Segue-se uma pesquisa para aprofundar
teoricamente cada um dos recursos, possivelmente elegendo novos.

Com um baseamento teórico substancial, o próximo passo é a criação
do protótipo de um jogo multiplataforma que utilize recursos
potencialmente limitados. Para ser considerado pronto, o protótipo deve
ser testado, e estar funcional, com adaptações ou não, em cada uma
das plataformas alvo definidas.

Com o protótipo concebido, o passo que segue é a enumeração, e
descrição das limitações detectadas no processo de desenvolvimento e
testes do jogo. Este detalhamento deve responder as seguintes perguntas:

- Quais as limitações foram encontradas no jogo?
- Em quais plataformas?
- Sob quais circunstâncias?
- As limitações puderam ser contornadas?
- Algum efeito colateral das limitações no jogo?
- Qual a categoria do problema: usabilidade, funcionalidade,
manutibilidade, portabilidade ou performance? (segundo ISO) %}}}

%}}}

\chapter{RESULTADOS}
%{{{

Abaixo constam as limitações encontradas durante a pesquisa e concepção do jogo

Durante a construção do jogo utilizei a estratégia de declarar todos
os objetos relativos ao window e limitar o escopo. Isso se demonstrou
uma boa forma de separar as responsabilidades.

\subsection{8.1  LIMITAÇÕES}

> Apesar da grande maioria dos recursos dos dispositivos estar presente
em HTML5 ainda existem muitas funcionalidades faltando para este tipo
de aplicação. Por exemplo, não podemos mudar a imagem de fundo do
dispositivo, ou adicionar toques etc. Similarmente, existem muitas
APIs de nuvem como os serviços de impressão do ICloud ou Google
cloud que estão disponíveis para aplicações nativas mas não para
HTML5. Outros serviços utilitários como o C2DM do Google que está
disponível para desenvolvedores Android para utilizar serviços de push
também não estão disponíveis para o HTML5. (HASAN, 2012)

1.  VERSÕES
A grande maioria dos dispositivos atualmente no mercado utilizam
obsoletas de seus softwares. Isso dificulta o desenvolvimento. Se a
tecnologia de tradução para o navegador utilizar o a classe Webview do
Android - como o Apache cordova faz - as versões mais antigas podem ser
penalizadas com problemas de performance ou falta de recursos.

2. OFFLINE

Refresh duplo para ver assets cacheados. Ver:
http://buildnewgames.com/game-asset-management/

3. AUDIO
Api de som quebra quando executado diversas vezes.
Os navegadores variam na disponibilização de formatos aceitáveis
Somente um áudio pode ser tocado no Navegador do Android
Não é possível trocar o volume no IOS.
Alguns navegadores favorecem formatos ogg (vorbis) e outros, como o
Safari, favorecem o MP3.

> O maior problema com as API's de áudio e de vídeo do HTML5 é
a disputa entre os codecs dos navegadores. Por exemplo, Mozilla e
Opera suportam Theora, já o Safari suporta H.264 que também é
suportado pelo IE9. Ambos, Iphone e Android suportam H.264 em seus
navegadores. A W3C recomenda OggVorbis e OggTheora para áudio e vídeo
respectivamente. (HASAN et al, 2012)

3. VIDEO

Codecs

4. ASSETS

Trafegar muitos assets deixa o sistema lento.

 Contorno
Utilizando páginas de carregamento e/ou cache;

5. UI

É muito custoso desenvolver uma interfaces que pareçam nativas
para cada dispositivo sem a utilização de plugins e ferramentas
especializadas. Em termos gerais, trabalhar com proporções é
positivo. Não obstante
há casos, como o dos botões de certo e errado que a proporções ficam
exageradas, nesses casos a utilizada de max-width é uma solução
conveniente.

6. PERFORMANCE

De acordo com uma pesquisa, para um usuário uma tarefa é instantânea
se ele leva até 0.1 segundos para ser executada. Se a tarefa toma
aproximadamente um segundo então a demora será notada mas o
usuário não se incomodará com ela. Entretanto, se a tarefa leva
aproximadamente 10 segundos para terminar o usuário então começa a
ficar aborrecido e esse é o limite que algum feedback deve ser dado
para um usuário.

ACELERAÇÃO DE GPU

7. Acelerômetro

8. IMPLEMENTAÇÃO INCONSISTENTE DE APIs

9.  TAMANHO DE TELA
Em alguns casos o tamanho das telas pode ser um fator limitante – como
no caso de jogos de estratégia. Jogadores com telas menores podem sair
em desvantagem. 9. CÂMERA

10 . JavaScript
Ciclo de vida demorado pois necessita que todos os consumidores da
especificação entrem em consenso e implementem a.

Desktop/Firefox
Desktop/Google Chrome
Smatphone/Android
%}}}

\chapter{CONCLUSÕES}
%{{{

Não pude testar todos os métodos e ferramentas e versões à
disposição, um trabalho completo demandaria esforços conjuntos de
muitos indivíduos ou um período de tempo bem mais extenso. Se uma
empresa deseja produzir jogos nativos elas precisarão de vários
desenvolvedores. Eu sozinho fui capaz de produzir um jogo em tempo
razoável trabalhando apenas com a plataforma web.

Por não utilizar frameworks e bibliotecas estou me distanciando
dos casos da vida real. Só poderemos considerar o HTML como uma
especificação pronta quando
for possível fazer tudo o que se faz nativamente com os dispositivos
através de uma API web padronizada.

> Conforme JavaScript vai ganhando importância rápido progresso é
feito por diferentes empresas a fim de prover boas ferramentas de debug
e inspecionamento para JavaScript.

Baixa fricção quer dizer que você pode ir de um site para outro sem
ter que instalar, o serviço está em demanda, você não é obrigado a
tê-lo em sua home screen.


\subsection{ TRABALHOS FUTUROS}

EMACSCRIPT 7

%}}}

\chapter{ANEXOS}
%{{{

\subsection{ CONVERSORES PARA HTML5}
Além da possibilidade de escrever em HTML, pode-se optar pela
alternativa de utilizar-se um conversor de linguagens.

\subsection{ METODOLOGIA DE DESENVOLVIMENTO DE SOFTWARE PARA A CONSTRUÇÃO DE GAMES}

Como o jogo é um software complexo demanda-se a utilização de
metodologias de engenharia de software, dentre os processos de software
mais conhecidos academicamente destacamos:

- OpenUP: este é bem detalhado e de característica iterativa e
incremental. Gerando assim, um levantamento mais apurado dos riscos,
requisitos e outros detalhes do sistema e a criação incremental do
sistema, com requisitos maleáveis;

- Cascata: processo antigo, caracteriza-se por ser pouco maleável aos
requisitos mapeados posteriormente ao processo de análise;

- Processo ágil - SCRUM: sua utilização é flexível e sendo
um método ágil especifica pouca documentação, ou como dizem,
somente a documentação necessária, este processo é bem conhecido e
aceito na comunidade de desenvolvimento de software. Suas principais
características são: divisão do processo de desenvolvimento através
uma série de iterações chamadas sprints. Cada sprint consiste
tipicamente em duas a quatro semanas. É bem aplicado a projetos que
mudam constantemente e que demandam rápidas adaptações;

- Processo ágil – XP: tem muitas características similares ao SCRUM
por este também ser um processo ágil. Dentre suas especifidades
destaca-se: versões frequentes, pequenos ciclos de desenvolvimento que
buscam aumentar a produtividade, introduzem checkpoints onde os clientes
podem agregar novas funcionalidades;

\subsection{ AMBIENTES PARA DESENVOLVIMENTO HTML5}

Na pesquisa efetuada sobre estes frameworks full-stack foram
identificadas as seguintes tecnologias:

    - segundo (PRADO, 2012) o GWT é um framework essencialmente para
o lado do cliente (client side) e dá suporte à comunicação com
o servidor através de RPCs Remote Procedure Calls (ou procedimento
de chamadas remotas). Ele não é um framework para aplicações
clássicas da web, pois deixa a implementação da aplicação web
parecida com implementações em desktop. Este é utilizado em muitos
produtos de grande porte como o Google Adwords e Google Wallet. Outra
característica interessante é que a plataforma opera sobre a licença
Apache versão 2;

    - construct 2 - é um editor na nuvem focado para usuários sem
    - conhecimento prévio em programação orientado a comportamento;
    - PlayCanvas - é uma plataformas para a construção de jogos 3D
na nuvem, desenvolvida com foco em performance. Permite a hospedagem,
controle de versão e publicação dos aplicativos nela criados,
possibilita também a importação de modelos 3D de softwares populares
como: Maya, 3ds Max e Blender;

    - o ambiente HTML5 Development Environment (ambiente de
desenvolvimento HTML5) da Intel, este fornece uma solução na nuvem,
completa para o desenvolvimento em plataforma cruzada, com serviços de
empacotamento, serviços para a criação e testes de aplicativos com
montagem de interfaces drag and drop (Intel XDK) e bibliotecas para a
construção de jogos utilizando aceleração de hardware, o que garante
até duas vezes mais performance que aplicativos mobile baseados em
Web tradicionais. Esta solução é gratuita, open-source e funciona
através de um plugin para o Google Chrome, ou seja, o desenvolvimento
também é multiplataforma e devido ao fato de os binários ficarem
hospedados na nuvem, possibilitou a Intel criar compiladores para cada
uma das plataformas disponibilizadas pelo PhoneGap, que é o framework
polyfill utilizado na solução.

\subsection{ HTTP}

\subsection{FRAMEWORKS DE DESENVOLVIMENTO DE JOGOS EM HTML5;}


\subsection{ FRAMEWORKS PARA DESENVOLVIMENTO DE JOGOS HTML5}
Com o intuito de simplificar o processo para os desenvolvedores,
auxiliando-os a focarem-se apenas nas soluções que estão
desenvolvendo, foram criados os frameworks para desenvolvimento de
jogos. Não obstante, o intuito deste trabalho é desenvolver um jogo
sem auxilio de frameworks pois estes muitas vezes escondem possíveis
limitações, desenvolvendo soluções próprios.

- enchant.js: dentre suas funcionalidades constam: orientação à     ;
- objetos, orientado à eventos, contém um motor de animação,       ;
- suporta WebGL e Canvas, etc three.js: considerada leve, renderiza    ;
- WebGL e Canvas, arquitetura procedural                               ;
- quintus: bom para plataformas 2D
- limeJs: bom para 2d

\subsection{ INTERFACE E ESCOLHAS DE DESIGN}

\subsection{PROGRESSÃO CONTÍNUA}

\subsection{JAVASCRIPT NÃO OBSTRUTIVO}

\subsection{Arquitetura Cliente Servidor}
%}}}

\end{document}
